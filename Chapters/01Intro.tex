%*****************************************
\chapter{Introduction}\label{ch01:introduction}
%*****************************************
%TODO Status Text Draft: Checked against the LO Version

\begin{center}
	\rowcolors{1}{gray!30}{gray!10}
	\begin{tabularx}{.90\linewidth}{X}
		\hline 
		\textbf{ Objectives } \\ 
		\hline
		1. Identify the various sources of knowledge\\ 
		2. Define "science"\\ 
		3. Describe the scientific method and relate that to business research\\ 
		4. Identify the three types of science research (exploratory, descriptive, and explanatory)\\ 
		\hline
	\end{tabularx}
\end{center}
\medskip

\section{Knowing}

In general, people want to know about things. Most people are curious about the world around them but business owners are interested in specifically how people can be persuaded to make a purchase. Understanding how one person can walk past a candy store without even the slightest thought about going inside while another cannot seem to walk in the same block without stopping in for a treat is valuable information for the owner of the candy store. In general, business owners are eager to know about people and what drives their behavior.

The goal of this book is to teach students how research can be used to help business owners make good decisions. More specifically, the book examines the ways that researchers come to understand the impetus that drives purchases. The research methods considered in this book are a systematic process of inquiry designed to learn something of value about a business problem. Before considering research methods, though, it is useful to contemplate other sources of knowledge.

\subsection{Different Sources of Knowledge}\label{ch01.01}

As an introduction to the field of research, it is useful to briefly consider common sources of knowledge. 

\begin{itemize}
	\item \textbf{Assumptions}. Many people assume that children without siblings are rather spoiled and unpleasant. In fact, many people believe that the social skills of only children will not be as well developed as those of people who were reared with siblings. However, sociological research shows that children who grow up without siblings are no worse off than those with siblings when it comes to developing good social skills\cite{bobbitt2013number}. Researchers consider precisely these types of assumptions that ``everyone knows'' when investigating their worlds. Sometimes the assumptions are correct and other times not so much.
	
	\item \textbf{Direct Experience}. One source of knowledge is direct experience. Mark Twain observed that ``... the cat that sits down on a hot stove-lid ... will never sit down on a hot stove-lid again...''\cite{twain2014following}. Direct experience may be a source of accurate information, but only for those who experience it. The problem is that the observation is not deliberate or formal; rather, it comes as an accidental by-product of life. Even worse, the lesson learned may be wrong. Without a systematic process for observing and evaluating those observations any conclusions drawn are suspect.

	\item \textbf{Tradition}. Another source of knowledge is tradition. There is an urban legend about a woman who for years used to cut both ends off of a ham before putting it in the oven\textbackslash\{\}footnote\{See Snopes: http://www.snopes.com/weddings/newlywed/secret.asp\}. She baked ham that way because that was the way her mother did it, so clearly that was the way it was supposed to be done. Her mother was the authority, after all. After years of tossing cuts of perfectly good ham into the trash, however, she learned that the only reason her mother ever cut the ends off ham before cooking it was that her baking pan was not large enough to accommodate the ham without trimming it. Tradition may or may not be a good source of knowledge.
	
	\item \textbf{Authority}. Many people rely on the government, teachers, and other authority figures to dispense knowledge. Unfortunately, authority figures may or may not be a source of accurate knowledge.
	
	\item \textbf{Observation}. People rely on their own informal observations of their worlds. Occasionally, someone will decide to ``investigate'' something, perhaps an odd sound, and their observations will become more selective. Unfortunately, these types of observations are not systematic and may easily lead to incorrect conclusions.
	
	\item \textbf{Generalization}. Often a broad pattern is observed and people draw a conclusion that the pattern is true for all instances. This can be the source of prejudice where the actions of a few bad actors may bias peoples' knowledge of the whole.
	
\end{itemize} 

While there are many ways that people come to know what they know, some of those ways are more reliable than others. The goal of formal research is to ferret out an accurate answer to the questions people have \textemdash \; to provide a reliable source of knowledge. 

\subsection{What is science?}

Most research methods used for business and marketing are based on methods used in the various social sciences and this section of the book describes how that scientific research is conducted.

Many students assume that ``science'' is a craft practiced by highly educated experts wearing white lab coats and pouring boiling liquids into test tubes. Unfortunately, that is not an accurate definition of ``science.'' Etymologically, the word ``science'' is derived from the Latin word \textit{scientia}, which means knowledge. ``Science,'' then, is a systematic and organized body of knowledge acquired by using a specific, rigorous method in any field of inquiry. The sciences can be grouped into two broad categories: natural and social. Natural science is the science of naturally occurring objects or phenomena, such as light, objects, matter, earth, celestial bodies, or the human body. Natural sciences are further classified into the physical sciences, earth sciences, life sciences, and others. In contrast, social science is the science of people or collections of people, such as groups, firms, societies, or economies, and their individual or collective behaviors. Social sciences can be classified into disciplines such as psychology (the science of human behaviors), sociology (the science of social groups), and economics (the science of markets and economies).

Sciences are also classified by their purpose. Basic sciences, also called pure sciences, are those that explain the most basic objects and forces, relationships between them, and laws governing them. Examples include physics, mathematics, and biology. Applied sciences, also called practical sciences, are sciences that apply scientific knowledge from basic sciences to a physical environment. For instance, engineering is an applied science that applies the laws of physics and chemistry to practical applications such as building stronger bridges or fuel efficient combustion engines, while medicine is an applied science that applies the laws of biology to relieving human ailments.

Scientific knowledge is a generalized body of laws and theories acquired using the scientific method to explain a phenomenon or behavior of interest. Closely related to laws and theories are hypotheses.

\begin{itemize}
	\item Laws are observed patterns of phenomena or behaviors and are based on repeated experimental observations. They are generalized rules that explain observations and are, typically, theories that have been repeatedly tested and believed to be true. As an example, the Newtonian Laws of Motion describe what happens when an object is in a state of rest or motion (Newton's First Law), what force is needed to move a stationary object or stop a moving object (Newton's Second Law), and what happens when two objects collide (Newton's Third Law). Collectively, the three laws constitute the basis of classical mechanics \textemdash \; a theory of moving objects. 
	\item Theories are systematic explanations of underlying phenomenon or behavior. Theories are typically based on hypotheses that have been tested and found to be true, but the testing has been incomplete or not rigorous enough to classify the theory as a law. It is important to note that theories are not ``wild guesses'' but are, instead, the result of experimental observations that found to be true in the instances tested. It is also important to note that theories can be falsifiable, that is, there are ways to prove that the theory is not true. As examples, the theory of optics explains the properties of light and how it behaves in different media, electromagnetic theory explains the properties of electricity and how to generate it, quantum mechanics explains the properties of subatomic particles, and thermodynamics explains the properties of energy and mechanical work. 
	\item Hypotheses are a well-guessed explanation of some phenomena or a prediction about what will happen in the future. Hypotheses are generally the beginning of an investigation that will either support or reject the hypotheses. As an example, a researcher may hypothesize that products in red boxes sell better than products in blue boxes. To test the hypothesis an experiment can be set up where the same product is sold in two identical boxes, except that one box is red and the other blue.
\end{itemize}

The pure science of economics and its applied science of business includes a body of both laws and theories. For example:

\begin{itemize}
	\item Law of Supply and Demand. While this is often described as a ``model'' it is also usually categorized as a law since it has been shown to be true in repeated observations. This law basically states that there is a relationship between a product's demand and its supply.
	\item Law of Diminishing Returns. This law states that at some point increasing a single production factor will yield less profit-per-unit produced. In other words, the return on the investment is not worth the cost.
	\item The 2009 Nobel Prize for economics was for the theory that groups work together to manage common resources, like water, by using collective property rights.
	\item The theory of marginalism attempts to explain the discrepancy in the value of goods by looking at their secondary, or marginal, utility. The price of diamonds is greater than water because of a marginal ``satisfaction'' of owning diamonds when compared to water, even though water is far more utilitarian.  
\end{itemize}

The goal of scientific research is to discover laws and postulate theories that can explain natural or social phenomena, or in other words, build scientific knowledge. It is important to understand that this knowledge may be imperfect or even quite far from the truth. It is important to understand that theories, upon which scientific knowledge is based, are explanations of a particular phenomenon and some tend to fit the observations better than others. The progress of science is marked by progression over time from poorer theories to better theories through enhanced observations using more accurate instruments and more informed logical reasoning.

Scientific laws or theories are derived through a process of logic and evidence. Logic (theory) and evidence (observations) are the two, and only two, pillars upon which scientific knowledge is based. In science, theories and observations are interrelated and one cannot exist without the other. Theories provide meaning and significance to what we observe and observations help validate or refine existing theory or construct new theory. Any other means of knowledge acquisition, such as faith or authority, cannot be considered science.

\subsection{Scientific Research}

Scientific research moves easily between theory and observations, each reinforcing the other. Theory drives the research of some phenomenon but observations made by the research further refine the underlying theory. Relying solely on observations for making inferences while ignoring theory is not scientific research, it is simple observation. The application of theories and observations lead to two primary types of scientific research: theoretical and empirical. Theoretical research is concerned with developing abstract concepts about natural or social phenomena while empirical research is concerned with testing theoretical concepts to see how well they reflect reality in our observations. 

Depending on a researcher's training and interest, scientific inquiry may take one of two forms: \textbf{\texttt{inductive research}} or \textbf{\texttt{deductive research}}. The goal of inductive research is to infer theoretical concepts and patterns from observed data. In contrast, the goal of deductive research, is to test theory using empirical data. Hence, inductive research is sometimes called theory-building research while deductive research is called theory-testing research. Note here that the goal of theory-testing is not just to test a theory, but to refine, improve, and extend it. 

\begin{center}
	\begin{figure}[H]
		\tikzstyle{topbox}=[rectangle,draw=blue!50,fill=blue!20,ultra thick,
			inner sep=10pt,minimum width=3cm,rounded corners=.25cm]
		\tikzstyle{botbox}=[rectangle,draw=red!50,fill=red!20,ultra thick,
			inner sep=10pt,minimum width=3cm,rounded corners=.25cm]
		\tikzstyle{every label}=[red]
		\begin{tikzpicture}
			\node[topbox] (theory)                        {Theory};
			\node[botbox] (observe) [below=2cm of theory] {Observation};
			\node (righttext) [below right=1cm of theory,yshift=.4cm,text width=2cm] {Test Hypotheses};
			\node (lefttext)  [below left= 1cm of theory,yshift=.6cm,text width=2cm] {Generalize From Observations};
			\node (deduction) [below right=1cm of theory,yshift=-1.25cm,xshift=2.0cm,rotate=90] {Deduction};
			\node (induction) [below left=1cm of theory,yshift=-1.15cm,xshift=-2.5cm,rotate=270] {Induction};		
	
			% Draw the red arrows		
			\draw [-,red,line width=3pt] [out=0]  (theory) 
				to [in=90] (righttext);
			\draw [->,red,line width=3pt] [out=270]  (righttext) 
				to [in=0] (observe);
	
			% Draw the blue arrows
			\draw [-,blue,line width=3pt] [out=180]  (observe) 
				to [in=270] (lefttext);
			\draw [->,blue,line width=3pt] [out=90]  (lefttext) 
				to [in=180] (theory);
	
		\end{tikzpicture}
		\caption{Scientific Research Model}
		\label{fig01.01}
	\end{figure}
\end{center}

Figure \ref{fig01.01} illustrates the complementary nature of inductive and deductive research; they are two halves of a research cycle that constantly iterates. It is important to understand that theory-building (inductive research) and theory testing (deductive research) are both critical for the advancement of science\footnote{Both inductive and deductive research is covered more thoroughly in Chapter 2.}. Elegant theories are not valuable if they do not match reality. Likewise, mountains of data are also useless until they can contribute to the construction of meaningful theories. Rather than viewing these two processes in a circular relationship, as shown in Figure \ref{ch01.fig01}, perhaps they can be better viewed as a helix, with each iteration between theory and data contributing to improved observations of the phenomena and the resulting improved theory. Though both inductive and deductive research are important for the advancement of science, it appears that inductive (theory-building) research is more valuable when there are few prior theories or explanations, while deductive (theory-testing) research is more productive when there are many competing theories of the same phenomenon and researchers are interested in knowing which theory works best and under what circumstances.

Theory building and theory testing are particularly difficult in business and marketing, given the imprecise nature of the theoretical concepts and the presence of many unaccounted factors that can influence the phenomenon of interest. It is also very difficult to refute theories that do not work. For instance, Karl Marx's theory of communism as an effective economic engine withstood for decades before it was finally discredited as being inferior to capitalism in promoting growth. Erstwhile communist economies like the Soviet Union and China eventually moved toward more capitalistic economies characterized by profit-maximizing private enterprises. However, the recent collapse of the mortgage and financial industries in the United States demonstrates that capitalism also has its flaws and is not as effective in fostering economic growth and social welfare as previously presumed. Unlike theories in the natural sciences, marketing theories are rarely perfect, which provides numerous opportunities for researchers to improve those theories or build their own alternative theories.

Conducting scientific research, therefore, requires two sets of skills, theoretical and methodological, needed to operate in the theoretical and empirical levels respectively. Methodological skills (``know-how'') are relatively standard, invariant across disciplines, and easily acquired through various educational programs. However, theoretical skills (``know-what'') is considerably harder to master, requiring years of observation and reflection, and are tacit skills that cannot be taught but rather learned though experience. All of the greatest scientists in the history of humanity, such as Galileo, Newton, and Einstein were master theoreticians, and they are honored for the theories they postulated that transformed the course of science.

\subsection{Scientific Method}

If science is knowledge acquired through a scientific method then what is the ``scientific method?'' Scientific method refers to a standardized set of techniques for building scientific knowledge, such as how to make valid observations, how to interpret results, and how to generalize those results. The scientific method allows researchers to independently and impartially test preexisting theories and prior findings, and subject them to open debate, modifications, or enhancements. The scientific method must satisfy four characteristics:

\begin{description}
	\item[\textbf{\texttt{Replicability}}.] Others should be able to independently replicate or repeat a scientific study and obtain similar, if not identical, results.

	\item[\textbf{\texttt{Precision}}.] Theoretical concepts, which are often hard to measure, must be defined with such precision that others can use those definitions to measure those concepts and test that theory.

	\item[\textbf{\texttt{Falsifiability}}.] A theory must be stated in a way that it can be disproven. Theories that cannot be tested or falsified are not scientific theories and any such knowledge is not scientific knowledge. A theory that is specified in imprecise terms or whose concepts are not accurately measurable cannot be tested, and is therefore not scientific. Sigmund Freud's ideas on psychoanalysis fall into this category and is therefore not considered a ``theory'' even though psychoanalysis may have practical utility in treating certain types of ailments.

	\item[\textbf{\texttt{Parsimony}}.] When there are multiple explanations of a phenomenon, scientists must always accept the simplest or logically most economical explanation. This concept is called parsimony or ``Occam's razor.'' Parsimony prevents scientists from pursuing overly complex or outlandish theories with endless number of concepts and relationships that may explain a little bit of everything but nothing in particular.
\end{description}

Any branch of inquiry that does not allow the scientific method to test its basic laws or theories cannot be called ``science.'' For instance, art is not science because artistic ideas (such as the value of perspective) cannot be tested by independent observers using a replicable, precise, falsifiable, and parsimonious method. Similarly, music, literature, humanities, and law are also not considered science, even though they are creative and worthwhile endeavors.

The scientific method, as applied to business and marketing, includes a variety of research approaches, tools, and techniques, such as qualitative and quantitative data, statistical analysis, experiments, field surveys, case research, and so forth. Most of this book is devoted to learning about these different methods. However, recognize that the scientific method operates primarily at the empirical level of research, i.e., how to make observations and analyze and interpret these observations. Very little of this method is directly pertinent to the theoretical level, which is really the more challenging part of scientific research.

Finally, business researchers must bear in mind that the natural sciences are different from the social sciences in several important respects. The natural sciences are very precise, accurate, deterministic, and independent of the person making the observations. For instance, a scientific experiment in physics, such as measuring the speed of sound through a certain medium, should always yield the same results, irrespective of the time or place of the experiment. However, the same cannot be said for the social sciences, which tend to be less accurate and more ambiguous. For instance, an economist may want to measure the impact of some factor on a city's economy. Unfortunately, the outcome of that research may depend on the background and experience of the researcher, the indexes used to measure the impact, and the interpretation of those measures. In other words, there is a high degree variability in all social science research. While natural scientists agree totally on the speed of light or the gravitational attraction of the earth, there is no agreement among economists on questions like the impact of immigration and how much of a nation's economy should be earmarked for reducing carbon emissions. Researchers in business and marketing must be comfortable with handling high levels of ambiguity, uncertainty, and error that come with research in such sciences.

\subsection{Types of Science Research}

Depending on the purpose of research, scientific research projects can be grouped into three types: exploratory, descriptive, and explanatory. 

\subsubsection{Exploratory}

Exploratory research is often conducted in new areas of inquiry, where the goals of the research are: 

\begin{enumerate}
	\item to scope out the magnitude or extent of a particular phenomenon, problem, or behavior
	\item to generate some initial ideas (or ``hunches'') about that phenomenon
	\item to test the feasibility of undertaking a more extensive study regarding that phenomenon. 
\end{enumerate}

For instance, if the citizens of a country are generally dissatisfied with governmental policies during an economic recession, exploratory research may be directed at measuring the extent of citizens' dissatisfaction. It would consider how the dissatisfaction is manifested and the presumed causes of such dissatisfaction. Such research may include examination of publicly reported figures, such as estimates of economic indicators like gross domestic product (GDP), unemployment, and consumer price index. This research may not lead to a very accurate understanding of the target problem, but may be worthwhile in determining the nature and extent of the problem and serve as a useful precursor to more in-depth research.

\subsubsection{Descriptive}

Descriptive research is directed at making careful observations and detailed documentation of a phenomenon of interest. These observations must be based on the scientific method and therefore, are more reliable than casual observations by untrained people. Examples of descriptive research are tabulation of demographic statistics by the United States Census Bureau who use validated instruments for estimating factors like employment by sector. If any changes are made to the measuring instruments, estimates are provided with and without the changed instrumentation to allow the readers to make a fair before-and-after comparison regarding population or employment trends. Other descriptive research may include projects like chronicling reports of gang activities among adolescent youth, the persistence of religious, cultural, or ethnic practices in select communities, and the role of technologies in the spread of democracy movements.

\subsubsection{Explanatory}

Explanatory research seeks explanations of observed phenomena, problems, or behaviors. While descriptive research examines what, where, and when of a phenomenon, explanatory research seeks answers to why and how. It attempts to ``connect the dots'' in research, by identifying causal factors and outcomes of the target phenomenon. Examples include understanding the reasons behind gang violence with the goal of prescribing strategies to overcome such societal ailments. Most academic or doctoral research belongs to the explanation category, though some amount of exploratory and/or descriptive research may also be needed during initial phases of a research project. Seeking explanations for observed events requires strong theoretical and interpretation skills, along with intuition, insights, and personal experience.

\subsection{Specific Considerations for Business/Marketing Research}

It is important to keep in mind that business researchers attempt to explain patterns in the habits of customers. A pattern does not explain every single person's experience, a fact that is both fascinating and frustrating. Individuals who create a pattern may not be the same over time and may not know one another, but they collectively create a pattern. Those new to business research may find these patterns frustrating because they expect various patterns to describe a group's characteristic but that often does not translate into an actual experience. A pattern can exist among a cohort without a specific individual being $ 100\% $ true to that pattern.

As an example of patterns and their exceptions, consider the impact of social class on peoples' educational attainment. In fact, Ellwood \& Kane\cite{ellwood2000getting} found that the percentage of children who did not receive any postsecondary schooling was four times greater among those in the lowest quartile income bracket than those in the upper quartile (that is, children from high-income families were far more likely than low-income children to go to college). These research findings detected patterns in society, but there are certainly many exceptions. Just because a child grows up in a household with little wealth does not keep that child from pursuing a college degree. People who object to research findings tend to cite evidence from their own personal experience, insisting that no patterns actually exists. The problem with this response, however, is that objecting to a social pattern on the grounds that it does not match a specific person's experience misses the point about patterns.

Another matter that social scientists must consider is where they stand on the value of basic as opposed to applied research. In essence, this has to do with questions of for whom and for what purpose research is conducted. We can think of basic and applied research as resting on either end of a continuum. In marketing, basic research studies marketing for marketing's sake \textemdash \; nothing more, nothing less. Sometimes researchers are motivated to conduct research simply because they happen to be interested in a topic and the goal may be to learn more about a topic. Applied research lies at the other end of the continuum. In marketing, applied research studies marketing for some purpose beyond a researcher's interest in a topic. Applied research is often client focused, meaning that the researcher is investigating a question posed by someone other than her or himself.

One final consideration for business and marketing researchers is the difference between qualitative and quantitative methods. Qualitative methods generally involve words (like letters, memos, or policies) or pictures and common methods used include field research, interviews, and focus groups. Quantitative methods, on the other hand, generally involve numbers and common methods include surveys, content analysis, and experimentation. While qualitative methods aim to gain an in-depth understanding of a relatively small number of cases, quantitative methods offer less depth but more breadth because they typically focus on a much larger number of cases.

Sometimes these two methods are presented or discussed in a way that suggests they are somehow in opposition to one another. The qualitative/quantitative debate is fueled by researchers who may prefer one approach over another, either because their own research questions are better suited to one particular approach or because they happened to have been trained in one specific method. While these two methodological approaches differ in goals, strengths, and weaknesses, they both attempt to answer a researcher's question and are equally viable. This text operates from the perspective that qualitative and quantitative methods are complementary rather than competing and both will be covered. 

\subsection{Summary}

\begin{itemize}
	\item There are many different sources of knowledge and some are more valuable than others for formulating theories and practices.

	\item Science is the discipline of using formalized processes to create theories to explain observed phenomena.

	\item Scientific research is a process with a goal of using reproducible methods to create a theory or validate the tenants of an existing theory.

	\item Scientific research can be divided into three types: exploratory, descriptive, and expanatory.

	\item Business research has specific considerations to meet the sometimes disparate objectives of theory-building and practical application.
\end{itemize}

\printbibliography