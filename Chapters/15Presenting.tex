%*****************************************
\chapter{Presenting Research}
%*****************************************
%TODO Status: Pre-draft

\begin{wrapfigure}{r}{0.4\textwidth}
	\centering
	\includegraphics[width=0.4\textwidth]{gfx/15-conduct} 
\end{wrapfigure}

No conductor would rehearse an entire orchestra on a masterpiece like Beethoven's Ninth Symphony and then present that music to an empty theater. That would be analogous to a researcher working months, sometimes years, on a project that brings a sharp focus to a corner of the researcher's world and then failing to share the results of that research with others. Research is normally presented in both written and oral formats and, additionally, there are several levels of formality within each format. For example, an oral presentation can be a formal talk, a round table discussion, or a poster session. This chapter concerns presenting research in both formats and includes practical information that helps inform researchers who are writing or presenting their work.\blfootnote{Photo by xu duo on Unsplash}

\begin{center}
	\begin{objbox}{Objectives}
		\begin{itemize}
			\setlength{\itemsep}{0pt}
			\setlength{\parskip}{0pt}
			\setlength{\parsep}{0pt}
			
			\item What and with whom to share research results.
			\item Oral presentation tips.
			\item Written presentation tips.
			\item Disseminating findings.
		\end{itemize}
	\end{objbox}
\end{center}

\section{Introduction}

Most researchers hope that their work will have relevance to others besides themselves\footnote{Some of the material in this chapter was adopted from McLean, \textit{Business Communication for Success}\cite{mclean2012business}}. As such, research is in some ways a public activity. While the work may be conducted by an individual in a private setting, the knowledge gained from that work should be shared with  peers and other parties who may have an interest. Understanding how to share research is an important aspect of the research process.

\section{What and With Whom to Share}

When preparing to share work with others, researchers must decide what to share, with whom to share it, and in what format(s) to share it. This section considers the ``what'' and ``with whom'' aspects while later sections cover the various formats and mechanisms through which research is shared.

\subsubsection{Sharing It All}

Because conducting research is a scholarly pursuit and because researchers generally aim to reach a true understanding of business and economics processes, it is crucial that all aspects of research, the good, the bad, and the ugly, are shared. Doing so helps ensure that others will understand, be able to build from, and effectively critique the work.

It is also important to share all aspects of a research project for ethical reasons and to permit other researchers to replicate the work. The following questions will aid researchers in preparing to share research with others.

\begin{enumerate}
	\item Why was the research conducted?
	\item How was the research conducted?
	\item For whom was the research conducted?
	\item What conclusions can be reasonably drawn from this research?
	\item How could the research have been improved?
\end{enumerate}

Answering these questions help researchers be honest with themselves and the readers about their own personal interest, investments, or biases with respect to the work. The third question helps identify the major stakeholders, like funders, research participants, or others who share something in common with the research project (\eg, members of the community or social group who were involved in the research). These groups may be interested in the outcome of the research but may also be a source of bias. The last two questions help identify the strengths and weaknesses of the research project and could point the way for future projects.

\subsubsection{Knowing Your Audience}

An important decision for researchers is determining with whom to share the results. Certainly, the most obvious candidates with are other researchers working in the same field. Other potential audiences include stakeholders, reporters and other media representatives, policymakers, and members of the public more generally.

While the findings of a research project would never be altered for different audiences, understanding the audience helps frame the research report in a way that is most meaningful to that group. For example, the report for a project about the spending habits of elderly pensioners may be much different if rendered for a group of business owners, a governmental committee on aging, the funding agency, and a community meeting. In all cases, researchers would share the major findings, but the method of presentation and level of detail would vary by audience.

It would be expected that the greatest amount of detail, including data collection method, sampling, and analytic strategy, would be shared with colleagues and the funding agency. In addition, the funding agency may want information about the exact time line for the project along with any bureaucratic hiccups encountered. With a community meeting, though, a more succinct summary of the important findings using less technical jargon would be appropriate.

\section{Oral Presentations}

\subsection{Settings}

Researchers frequently make presentations to their peers in settings like conferences or departmental meetings. These presentations are excellent means for feedback and help researchers prepare to write up and publish their work. Presentations might be formal talks, either as part of a panel at a professional conference or to some other group of peers or other interested parties; less formal round-table discussions, another common professional conference format; or posters that are displayed in some specially designated area.

\subsubsection{Formal Talk}

When preparing a formal talk, it is very important for researchers to get details well in advance about the time limit for the presentation, requirements for questions from the audience, and whether visual aids, such PowerPoint slides, are expected. At conferences, the typical formal talk is usually expected to last between 15 and 20 minutes. Once researchers start talking about something as as important as their own research, it is common for them to become so engrossed that they forget to watch the clock and they then find themselves running short of time. To avoid this all-too-common occurrence, it is crucial that presenters practice in advance and time those practice sessions.

One common mistake made in formal presentations of research work is in setting up the problem the research addresses. Audience members are usually more interested to hear about the researcher's work work than to hear the results of a long list of previous related studies. While written reports must discuss related previous studies, presentations must use the precious time available to highlight the current research project. Another mistake is to simply read the research paper verbatim. Nothing will bore an audience more quickly than hearing a presenter drone on while reading aloud. Finally, a presentation should highlight only the key points of the study, which, generally, include the research question, methodological approach, major findings, and a few final takeaways.

\subsubsection{Round Table Presentation}

In less formal round table presentations, the aim is usually to help stimulate a conversation about a topic. Normally, several research projects are presented so the time available for each is normally shorter than in a formal presentation. Also, round table presentations always includes time for a conversation following the presentations. Round tables are especially useful when a research project is in the early stages of development. For example, perhaps a researcher has conducted a pilot study and is interested in ideas about where to take the study next. A round table is also an excellent place for a preview of potential objections reviewers may raise with respect to the project's approach or conclusions. Finally, round tables are great places to network and meet other scholars who share common interests and may be engaged in similar research projects.

\subsubsection{Poster Presentation}

Finally, a poster presentation is a visual representation of a research project. Often, poster sessions are tables lined up in a conference area where researchers have a visual display on the table but also stand by to answer questions. A poster should not be just pages from the report pasted onto a poster board, rather, researchers decide how to tell the ``story'' of the work in graphs, charts, tables, and other images. Bulleted points are acceptable as long as the people walking by can quickly read and grasp the major argument and findings. Posters, like round tables, can be quite helpful at the early stages of a research project because they are designed to encourage the audience to engage in conversation about the research. It is not necessary to share every detail of a research project in a poster, the point is to share highlights and then discuss the details with people who are interested.

\subsection{Types of Presentations}

Most books about oral presentations divide presentations into several broad types.

\begin{description}
	\item[Speech to inform] Increase the audience's knowledge, teach about a topic
	or issue, and share the speaker's expertise.
	\item[Speech to demonstrate] Show the audience how to use, operate, or do
	something.
	\item[Speech to persuade] Influence the audience by presenting arguments
	intended to change attitudes, beliefs, or values.
	\item[Speech to entertain] Amuse the audience by engaging them in a
	relatively light-hearted speech that may have a serious point or goal.
	\item[Ceremonial speech] Perform a ritual function, such as give a toast at a
	wedding reception or a eulogy at a funeral.
\end{description}

Often, presentations are stressful since most people do not like speaking in public. However, the following tips may help.

\begin{itemize}
	\item Perfection is not required. Letting go of perfection can be the hardest guideline for speakers to apply to themselves. It is human nature to compare ourselves to others. It seems odd, but most people can forgive another researcher for the occasional slip or ``umm'' during a speech, but then turn right around and chastise themselves for making the same error. Everyone has both strengths and weaknesses and researchers must learn where they can improve is an important first step. The old saying is that Rome was not built in a day and good speakers are not developed overnight. It is true that no one wants to see a researcher fail during a presentation so audience members are generally very forgiving of minor speaking faults.
	\item Take the time to prepare and get organized. Researchers know the topic for the speech (normally a research report) and they are speaking in order to inform or persuade audience members to consider an idea. One of the best ways to build confidence is to know the material being presented ``inside out.''
	\item Public speaking is not unlike participating in a conversation. In regular conversations, researchers do not give a second thought to the process of saying something and then waiting for a reply. A public speech follows a similar pattern, but the reply is normally in the form of a non-verbal body language.
\end{itemize}

There are certain to be various obstacles arise in a presentation.

\begin{itemize}
	\item Language. Researchers work in fields where there are acronyms and insider jargon. As long as a presenter is 100\% certain that everyone in the audience understands some term then it is acceptable to use that term, but if there is any doubt then speakers should defer to common terms. As an example, soldiers may understand a sentence like ``I left the CHU in the COP and was heading to the DFAC when a fast mover dropped a JDAM in the village.''{\footnote{''I left the Containerized Housing Unit in the Combat Outpost and was heading to the Dining Facility when a fighter jet dropped a Joint Direct Attack Munition (missle) in the village.''}} People who lack military experience would not be able to understand that sentence.
	\item Culture. Everyone's culture is different and a speaker must understand that the audience's culture may be different enough that communication becomes challenging.
	\item Role. The speaker and audience may play very different roles in an organization and those roles may create barriers to communication. For example, researchers presenting to a room full of corporate executives would face a much different communications problem than if the room was filled with line workers. Effective speakers must understand the roles of the audience members and then speak in a way that is understandable to those people.
	\item Goal. Each of the audience members will have a goal in mind when attending the presentation. Occasionally, the goal may only be that they were assigned to attend the presentation; but more often than not, researchers will be presenting at a conference where the audience members select to attend the session. As much as possible, the speaker should attempt to understand the audience members' goals and then address those goals.
	\item Ethnocentrism. One obstacle to be avoided is for the researcher, or audience members, to exhibit a feeling that they are somehow superior to everyone else. While that can come from ethnicity, it is also a result of scholarly ``snobbish'' behavior, prejudice, or even stereotypes.
\end{itemize}

\subsection{Visual Aids}

Nearly all research presentations include some sort of visual aid. It is much easier for a speaker to use a graph or chart than to verbally describe some relationship in the data. When preparing visual aids, keep in mind that they should be...

\begin{itemize}
	\item Big. They need to be large enough to be seen from the back row of the auditorium. For graphs and charts this may be easy enough to arrange since those visuals can be made larger on the screen, but for a physical artifact it may be impossible to magnify it so its utility may be questioned.
	\item Clear. The visual needs to clearly convey whatever message is intended.
	\item Simple. There is an old rule of thumb about visual aids: 6 X 6, which means no more than six lines of text and six words per line.
	\item Consistent. All visuals should use a consistent style so audience members do not have to first learn how to read the graphic but can focus, instead, on the information being presented.
\end{itemize}

Color is a powerful communication tool, but speakers must be careful with color. First, keep in mind that some audience members will not be able to distinguish between two or more colors in the visual aid so never use color as the sole information source. Also, avoid using too many colors on one chart, a few well-placed colors are always more powerful than a lot of colors sprinkled seemingly at random around the visual. The following tips may help.

\begin{itemize}
	\item Keep visual aids simple.
	\item Use one key idea per slide.
	\item Avoid clutter, noise, and overwhelming slides.
	\item Use large, bold fonts that the audience can read from at least twenty feet from the screen.
	\item Use contrasting colors to create a dynamic effect.
	\item Use analogous colors to unify your presentation.
	\item Use clip art with permission and sparingly.
	\item Edit and proofread each slide with care and caution.
	\item Use copies of your visuals available as handouts after your presentation.
	\item Check the presentation room beforehand.
	\item Have a backup plan in case technology fails, such as providing printed visuals.
\end{itemize}

\section{Written Presentations}

Written reports that will be read by other scholars generally follow a formal format that is outlined by the publication journal. However, most scholarly reports include an abstract, an introduction, a literature review, a discussion of research methodology, a presentation of findings, and some concluding remarks and discussion about implications of the work. Reports written for scholarly consumption also contain a list of references and many include tables or charts that visually represent some component of the findings. Reading published research in business or economics is an excellent way to develop an understanding of the core components of scholarly research reports and to begin to learn how to write those components.

Reports written for public consumption differ from those written for scholarly consumption. As noted elsewhere in this chapter, knowing the audience is crucial when preparing a written report. Whoever your audience, it is important to keep in mind that scientific evidence is being reported. Writers must take seriously their roles as business researchers and be mindful of their place among peers in the discipline. Findings must be presented as clearly and honestly as possible; appropriate recognition must be afforded to the scholars who have come before, even if the research raises questions about their work; and readers should be engaged in a discussion about the research and potential avenues for further inquiry. Normally, research writers will never meet the readers face-to-face, but it is beneficial to imagine what the readers would ask and provide a detailed response in the written report.

Finally, it is extremely important to not to commit plagiarism in a research report. Presenting someone else's words or ideas as if they are the researcher's own is among the most egregious transgressions a scholar can commit. Indeed, plagiarism has ended many careers and students' opportunities to pursue degrees\cite{maurer2006plagiarism}.

Closely related to plagiarism is libel, which is the written form of defamation, or a false statement that damages a reputation. If a false statement of fact that concerns and harms the person defamed is published, including publication in a digital or online environment, the author of that statement may be sued for libel. If the person defamed is a public figure they must prove malice or the intention to do harm, but if the victim is a private person, libel applies even if the offense cannot be proven to be malicious. Under the United States First Amendment writers have a right to express opinions, but the words used, and how they are used, are relevant to their interpretation as opinion versus fact. Writers must always be careful to qualify what they write and to do no harm.

\subsection{Writing Style}

Writing generally falls into one of three styles, colloquial, casual, and formal.

\subsubsection{Colloquial Style}

Colloquial language is an informal, conversational style of writing. It differs from standard business English in that it often makes use of colorful expressions, slang, and regional phrases. As a result, it can be difficult to understand for an English learner or a person from a different region of the country. Sometimes colloquialism takes the form of a word difference; for example, the difference between a ``Coke,'' a ``tonic,'' a ``pop,'' and a ``soda pop'' primarily depends on where you live. Colloquial phrases can also take the form of a saying. For example, in certain parts of the United States, the phrases ``dumb as a box of rocks'' and ``sharp as a tack'' both refer to a person's intelligence, but this may not be obvious for a person not intimate with English.

Colloquial writing may be permissible, and even preferable, in some business contexts. For example, a marketing letter describing a folksy product such as a wood stove or an old-fashioned popcorn popper might use a colloquial style to create a feeling of relaxing at home with loved ones. Still, it is important to consider how colloquial language  appears to the audience. Will the meaning of the chosen words be clear to a reader who is from a different part of the country? Will a folksy tone sound like the writer is ``talking down'' to the audience? A final point to remember is that colloquial style is not an excuse for using expressions that are sexist, racist, profane, or otherwise offensive.

\subsubsection{Casual Style}

Casual language involves everyday words and expressions in a familiar group context, such as conversations with family or close friends. The emphasis is on the communication interaction itself, and less about the hierarchy, power, control, or social rank of the individuals communicating. Casual communication is the written equivalent of wearing casual attire, like a t-shirt and jeans. When writing for business, a casual style is usually out of place; instead, a respectful, professional tone represents both the researcher and the company well.

\subsubsection{Formal Style}

Formal language is communication that focuses on professional expression with attention to roles, protocol, and appearance. It is characterized by professional vocabulary and syntax. That is, writers using a formal style tend to use a more sophisticated vocabulary, a greater variety of words, and more words with multiple syllables, not for the purpose of throwing big words around, but to enhance the formal mood of the document. They also tend to use more complex syntax, resulting in sentences that are longer and contain more subordinate clauses.

The appropriate style for a particular business document may be very formal, or less so. If a subordinate replies to an email from the supervisor, the exchange may be informal in that it is fluid and relaxed, without much forethought or fanfare, but it will still reflect the formality of the business environment. The subordinate will be careful to use an informative subject line, a semi-formal salutation (``Hi Mr. Smith'' is typical in e-mails), and a brief discussion about the topic at hand. Probably, the subordinate will also check grammar and spelling before clicking ``send.''

A formal document such as a proposal or an annual report will involve a great deal of planning and preparation, and its style may not be fluid or relaxed. Instead, it may use distinct language to emphasize the prestige and professionalism of the company. As an example, imagine a marketing letter that will be printed on company letterhead and mailed to a hundred sales prospects. Naturally, the letter should represent the company in a positive light and may include a sentence like ``The Widget 300 is our premium offering in the line; we have designed it for ease of movement and efficiency of use, with your success foremost in our mind.'' But in an e-mail or a tweet, an informal sentence may be used, ``W300 is a great stapler.''

\subsection{Report Format}

Research reports tend to follow a format that has evolved over many years.

\begin{description}
	\item[Title] This should be a concise description of the research findings.
	\item[Abstract] This is a very brief synopsis of the research findings. The exact size of the abstract is determined by the publisher, but they generally tend to be about 250 words in length.
	\item[Table of Contents] A TOC is not always used, especially for shorter reports. The publisher would determine if a TOC is important in the report.
	\item[Introduction]	This is a short description of the research project, why it was pursued and the anticipated result. The research thesis is often included in the introduction. Also, a few brief details of the methods and results is often found in the introduction.
	\item[Literature Review] This is normally one of the longer parts of the report. It surveys all of the existing related research in an effort to position the current research in the universe of prior research. Some literature reviews are structured chronologically while others are structured thematically. Regardless, there is a clear indication of where the current research project ``fits'' among prior research.
	\item[Methodology] This is a description of the method used during the research project. It would include information on how the sampling frame was selected, what sorts of data were gathered, and how those data were analyzed. This should be written thoroughly enough that another researcher could duplicate the project if desired.
	\item[Results] This details the results of the research project; however, the \textit{interpretation} of the results is normally saved for the ``discussion'' section of the report.
	\item[Discussion] This is a discussion about the relevance of the research project and how it fits with other research identified in the literature review. This would also describe any weaknesses of the research project and offer suggestions about how those could have been overcome.
	\item[Conclusion] This is a brief final summary of the entire research project, including the major findings. This would also suggest future research that would complement the current project.
	\item[Appendices] While appendices are not commonly used, when needed they are included at this point in the report.
	\item[Bibliography] All references used in the report are fully cited here so a reader could find all original research reports mentioned in the literature review or elsewhere in the report if desired.
\end{description}

While there are almost as many formal research report formats as there are publishers or universities that process those reports, the above is a good general-purpose guideline of the types of sections often needed for publication.

\section{Disseminating Findings}

The dissemination of research findings involves careful planning, thought, consideration of target audiences, and the best way to communicate with those audiences. Writing up results from a research project and having others take notice are two entirely different propositions. In fact, the general rule of thumb is that people will not take notice unless they are encouraged to do so. To paraphrase the classic line from the film \textit{Field of Dreams}, just because you build it does not mean they will come.

Disseminating research findings successfully requires determining who the audience is, where that audience is located, and how to reach them. When considering who the audience is, think about who is likely to take interest in the research project. The audience might include those who do not express enthusiastic interest but might nevertheless benefit from an awareness of the research. Of course, the research participants and those who share some characteristics in common with those participants are likely to have some interest in what was discovered in the course of the research. Other scholars who study similar topics are another obvious audience for the work. Perhaps there are policymakers who should take note of the work. Organizations that do work in an area related to the topic of the research project are another possibility. Finally, any and all inquisitive and engaged members of the public represent a possible audience for the work.

Where the audience is located should be fairly obvious once the composition of that audience is determined. The research participants are known since they were part of the study. Interested scholars can be found at professional conferences and via publications such as professional organizations' newsletters and scholarly journals. Policymakers include state and federal representatives who, at least in theory, should be available to hear a constituent speak on matters of policy interest. Organizations that do work in an area related to the research topic can be found with a simple web search. Finally, disseminating findings to the general public could take any number of forms: a letter to the editor of a local newspaper, a blog, or even a Facebook post.

Finally, determining how to reach the target audience will vary depending on which specific audience is of interest. The strategy should be determined by the norms of the audience. For example, scholarly journals provide author submission instructions that clearly define requirements for researchers wishing to disseminate their work via that journal. The same is true for newspaper editorials where the newspaper's website may contain details about how to format and submit letters to the editor. To reach out to political representatives, a call to their offices or a simple web search should provide information about how to proceed.

Researchers who have conducted high-quality research and have findings that are likely to be of interest to any constituents besides themselves would have a duty as a scholar to share those findings. 

\section{Summary}\label{ch15:summary}

\begin{center}
	\begin{tkawybox}{Summary}
		\begin{itemize}
			\setlength{\itemsep}{0pt}
			\setlength{\parskip}{0pt}
			\setlength{\parsep}{0pt}
			
			\item x1.
			\item x2.
			\item x3.
		\end{itemize}
	\end{tkawybox}
\end{center}
