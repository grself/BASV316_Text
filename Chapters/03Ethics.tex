%*****************************************
\chapter{Research Ethics}\label{03:ethics}
%*****************************************

% Ideas From Bryman
% Use Dalton's ``men who manage'' (1959) book (French) as an example of potential harm from not using informed consent (note: I cannot find any direct quotes from the book)
% p131 lists these stances on ethics: universalism, situation (end-justifies-means, no choice), ethical transgression is pervasive, anything goes.
% look at using Zimbardo's prison study as an example of ethics
% can I find info about a French TV show called ``Game of Death'' (2010)
% look at Kramer, Guillory, and Hancock (2014) for a project involving Facebook and informed consent.
% p145. Holliday, R ``Investigating Small Firms: Nice Work?'' (1995) has an interesting story about an ethics issue when doing fieldwork. This might make a good case for Chapter 11. (Could not find this article)


\section{Introduction}

\begin{wrapfigure}{r}{0.4\textwidth}
	\label{03:fig01} 
	\centering
	\includegraphics[width=0.4\textwidth]{gfx/03-book} 
\end{wrapfigure}

Ethics is defined by the Oxford dictionary as ``moral principles that govern a person's behaviour or the conducting of an activity.''\cite{oxford2018ethics}. Such principles are often defined at a disciplinary level though a professional code of conduct, and sometimes enforced by a university committee called an \gls{irb}. Often, codes of conduct are codified in written form, but even if they are not explicitly specified, researchers are still expected to be aware of and abide by general agreements shared by the research community on what constitutes acceptable and non-acceptable behaviors in the professional conduct of their disciplines. For instance, researchers should not manipulate their data collection, analysis, and interpretation procedures in a way that contradicts the principles of science or the scientific method or advances their personal agenda.\blfootnote{Photo by Aaron Burden on Unsplash}

\begin{center}
	\begin{objbox}{Objectives}
		\begin{itemize}
			\setlength{\itemsep}{0pt}
			\setlength{\parskip}{0pt}
			\setlength{\parsep}{0pt}
			
			\item Define ethics
			\item Discuss five primary ethical principles: voluntary participation, informed consent, confidentiality, disclosure, and reporting.
			\item Discuss the unique ethical considerations surrounding research on humans
			\item Define Institutional Review Board and discuss the types of issues that may accompany those boards.			
			\item Discuss various professional codes of ethics
		\end{itemize}
	\end{objbox}
\end{center}

Why is research ethics important? Because, science has often been manipulated in unethical ways by people and organizations to advance their private agenda and engaging in activities that are contrary to the norms of scientific conduct. A classic example is pharmaceutical giant Merck’s drug trials of Vioxx, where the company hid the fatal side-effects of the drug from the scientific community, resulting in $ 3468 $ deaths of Vioxx recipients, mostly from cardiac arrest. In $ 2010 $, the company agreed to a $ \$4.85 $ billion settlement and appointed two independent committees and a chief medical officer to monitor the safety of its drug development process. Merck's conduct was unethical and violation the scientific principles of data collection, analysis, and interpretation. This incident was reported by Ronald Green\cite{green2006direct}.

Ethics is the moral distinction between right and wrong and what is unethical may not necessarily be illegal. A researcher's conduct may not be culpable in the eyes of the law but may still lead to disciplinary hearings, professional notoriety, and even job loss on the grounds of professional misconduct. These ethical norms may vary from one society to another but this book uses the ethical standards as applied to research in Western countries.

\section{Background}

Researchers, in general, attempt to ensure their work is valid and above reproach. To be sure, mistakes are made, but these are typically discovered by peer review before publication and are chalked up to simple human error. There have been, though, a few cases of intentional fraud.

\begin{itemize}
	\item Van der Hayden reported on an outright fraud perpetrated by Woo-Suk Hwang who published a paper in $ 2004 $ about a breakthrough in human stem cell research. It was later admitted that the entire paper was completely fabricated\cite{van2009fraud}.

	\item Sheehan reported on a number of cases of research fraud\cite{sheehan2007fraud}, including, for example, ``Dr. Eric Poehlman of the University of Vermont was sentenced in June 2006 to 1 year in jail for falsifying and fabricating research data related to menopausal changes and metabolism.''

	\item Evans, Smith, and Willen reported on drug research where human trials are ``\ldots poorly regulated, riddled with conflicts of interest---and sometimes deadly.''\cite{evans2005secret}

	\item Meier wrote that the manufacturers of heart devices face a ``huge conflict of interest'' when attempting to balance notifying physicians of potential flaws with the financial harm that admission would create\cite{meier2005maker}.

\end{itemize}

Chubin pointed out that a number of behaviors---ranging from the serious (plagiarism and fabrication) to the not-so-serious (improper acknowledgment of collaborators)---slow scientific progress, undermine trust in the research process, waste public funds, and increase external regulation of science\cite{chubin1985research}.

While the ``big three'' ethics problems---falsification, fabrication, and plagiarism---are a concern, most researchers report that these are not the main issue. Rather, they worry about the mundane, everyday problems that can easily creep into a research project. Two of the main areas are dealing with data and the pressure of production\cite{de2006normal}.

\begin{itemize}
	\item When working with data it is easy to want to ``cut corners.'' There are many examples of researchers who didn't get the exact data they wanted so they may eliminate outliers or otherwise ``shape'' the data to a more desirable form while cleaning it. Even highly respected scientists have been guilty of data manipulation. Sheehan reports that Sigmund Freud fabricated case studies and Isaac Newton altered records of lunar and solar sightings to fit his theories\cite{sheehan2007fraud}.

	\item The second area is the pressure to produce published work. The old saying in research circles is ``publish or perish'' and there is more than a grain of truth to the fact that researchers must continue to produce published reports in order to attract grants and continue their research. This pressure makes it tempting to produce more studies, some of which may be of dubious value.

\end{itemize}

\subsection{Research on Humans}

The U.S. Department of Health and Human Services defines a human subject as ``...a living individual about whom an investigator (whether professional or student) conducting research: (i) Obtains information or biospecimens through intervention or interaction with the individual, and uses, studies, or analyzes the information or biospecimens; or (ii) Obtains, uses, studies, analyzes, or generates identifiable private information or identifiable biospecimens.''\cite{hhs2018human} In some states, human subjects also include deceased individuals and human fetal materials. 

Nonhuman research subjects, on the other hand, are objects or entities that investigators manipulate or analyze in the process of conducting research. In business research, nonhuman subjects typically include sources like newspapers, historical documents, advertisements, television shows, buildings, and even garbage.

Unsurprisingly, research on human subjects is regulated much more heavily than research on nonhuman subjects. However, there are ethical considerations that all researchers must consider regardless of their research subject.

\subsection{History of Research on Humans}

Research on humans has not always been regulated in the way that it is today. The earliest documented cases of research using human subjects are of medical vaccination trials. One such case took place in the late $ 1700 $s, when scientist Edward Jenner exposed an eight-year-old boy to smallpox in order to identify a vaccine for the devastating disease, as reported by Stefan Riedel\cite{riedel2005edward}. Medical research on human subjects continued without much law or policy intervention until the mid-$ 1900 $s when, at the end of World War II, a number of Nazi doctors and scientists were put on trial for conducting human experimentation during the course of which they tortured and murdered many concentration camp inmates, as reported by Ruth Faden\cite{faden1986history}. The trials, conducted in Nuremberg, Germany, resulted in the creation of the Nuremberg Code, a Ten-point set of research principles designed to guide doctors and scientists who conduct research on human subjects. Today, the Nuremberg Code guides medical and other research conducted on human subjects, including social scientific research.

Medical scientists are not the only researchers who have conducted questionable research on humans. In the $ 1960 $s, psychologist Stanley Milgram conducted a series of experiments designed to understand obedience to authority in which he tricked subjects into believing they were administering an electric shock to other subjects\cite{milgram1963behavioral}. In fact, the shocks were not real at all but some participants experienced extreme emotional distress after the experiment. The realization that one is willing to administer painful shocks to another human being just because someone who looks authoritative has told you to do so might indeed be traumatizing even if you later learn that the shocks were not real.

Around the same time that Milgram conducted his experiments, sociology graduate student Laud Humphreys\cite{humphreys1970tearoom} was collecting data for his dissertation research on the tearoom trade, the practice of men engaging in anonymous sexual encounters in public restrooms. Humphreys wished to understand who these men were and why they participated in the trade. To conduct his research, Humphreys offered to serve as a ``watch queen,'' the person who keeps an eye out for police and gets the benefit of being able to watch the sexual encounters, in a local park restroom where the tearoom trade was known to occur. What Humphreys did not do was identify himself as a researcher to his research subjects. Instead, he watched his subjects for several months, getting to know several of them, learning more about the tearoom trade practice and, without the knowledge of his research subjects, jotting down their license plate numbers as they pulled into or out of the parking lot near the restroom. Some time after participating as a watch queen, with the help of several insiders who had access to motor vehicle registration information, Humphreys used those license plate numbers to obtain the names and home addresses of his research subjects. Then, disguised as a public health researcher, Humphreys visited his subjects in their homes and interviewed them about their lives and health. Humphreys' research dispelled a good number of myths and stereotypes about the tearoom trade and its participants. He learned, for example, that over half of his subjects were married to women and many of them did not identify as gay or bisexual. However, once Humphreys’s work became public, the result created a major controversy at his home university, among sociologists in general, and among members of the public, as it raised public concerns about the purpose and conduct of sociological research.

These and other studies\footnote{\url{http://www.journalnma.org/article/S0027-9684(15)30517-4/pdf}} led to increasing public awareness of and concern about research on human subjects. In $ 1974 $, the U.S. Congress enacted the \textit{National Research Act}, which created the \textit{National Commission for the Protection of Human Subjects in Biomedical and Behavioral Research}. The commission produced \textit{The Belmont Report}\cite{national1978belmont}, a document outlining basic ethical principles for research on human subjects. The \textit{National Research Act} also required that all institutions receiving federal support establish an \gls{irb} to protect the rights of human research subjects. Since that time, many organizations that do not receive federal support but where research is conducted have also established review boards to evaluate the ethics of the research that they conduct.

\section{Ethical Principles}

Over the past half century or so several ethical principles have become widely accepted. To be sure, various disciplines, like medicine, have many ethical principles that would not be applicable to other fields. Other ethical principles have been practiced in the past but are no longer considered applicable. The following five principles, though, are generally accepted in all research fields.

\subsection{Voluntary Participation}

Subjects in a research project must be aware that their participation in the study is voluntary, that they have the freedom to withdraw from the study at any time without unfavorable consequences, and they are not harmed as a result of their participation or non-participation in the project. The most flagrant violations of the voluntary participation principle are probably forced medical experiments conducted by Nazi researchers on prisoners of war during World War II, as documented in the post-War Nuremberg Trials (these experiments also originated the term ``crimes against humanity''). Less known violations include the Tuskegee syphilis experiments\cite{reverby2009examining} conducted by the U.S. Public Health Service between $ 1932-1972 $, in which nearly $ 400 $ impoverished African-American men suffering from syphilis were denied treatment even after penicillin was accepted as an effective treatment of syphilis, and subjects were presented with false treatments such as spinal taps as cures for syphilis. Even if subjects face no mortal threat, they should not be subjected to personal agony as a result of their participation. In $ 1971 $, psychologist Philip Zambardo created the \textit{Stanford Prison Experiment}, where Stanford students recruited as subjects were randomly assigned to roles such as prisoners or guards. When it became evident that student prisoners were suffering psychological damage as a result of their mock incarceration and student guards were exhibiting sadism that would later challenge their own self-image the experiment was terminated.

As a less egregious example, instructors often ask students to fill out a questionnaire of some sort and inform the students that their participation is voluntary. This activity must be designed in such a way that students do not fear that their non-participation will hurt their grade in any way. For instance, it is unethical to provide bonus points for participation and no bonus points for non-participation because it places non-participants at a distinct disadvantage. To avoid such circumstances, instructors may provide an alternate task for non-participants so that they can earn the same number of bonus points without participating in the research study or by providing bonus points to everyone irrespective of their participation or non-participation. 

\subsection{Informed Consent}

All participants in a study must receive and sign an Informed Consent form that clearly describes their right to not participate and right to withdraw, before their responses in the study can be recorded. In a medical study, this form must also specify any possible risks to subjects from their participation. For subjects under the age of $ 18 $, this form must be signed by their parent or legal guardian. Researchers must retain these informed consent forms for a period of time (often three years) after the completion of the data collection process in order comply with the norms of scientific conduct in their discipline or workplace.

The consent form, itself, must not waive, or even appear to waive, any of the subject's legal rights. Subjects also cannot release a researcher or institution from any legal liability should something go wrong during the course of their participation in the research. Because sociological research does not typically involve asking subjects to place themselves at risk of physical harm by, for example, taking untested drugs or consenting to new medical procedures, sociological researchers do not often worry about potential liability associated with their research projects. However, their research may involve other types of risks. For example, if a researcher intentionally or accidentally reveals the identity of subjects who admit unusual sexual behavior the subject's social standing, marriage, custody rights, or employment could be jeopardized.

In some cases, subjects are asked to sign a physical consent form indicating that they have read it and fully understand its contents. In other cases, subjects are simply provided a copy of the consent form and researchers are responsible for making sure that subjects have read and understand the form before proceeding with any kind of data collection. 

One last point to consider when preparing to obtain informed consent is that not all potential research subjects are considered equally competent or legally allowed to consent to participate in research. These subjects are sometimes referred to as members of vulnerable populations, people who may be at risk of experiencing undue influence or coercion. The rules for consent are more stringent for vulnerable populations. For example, minors must have the consent of a legal guardian in order to participate in research. In some cases, the minors themselves are also asked to participate in the consent process by signing special, age-appropriate consent forms designed specifically for them. Prisoners and parolees also qualify as vulnerable populations. Concern about the vulnerability of these subjects comes from the very real possibility that prisoners and parolees could perceive that they will receive some highly desired reward, such as early release, if they participate in research. While gaining consent from vulnerable populations can be challenging failing to work with those groups ensures that their stories are never told. While there is no easy solution to this double-edged sword, an awareness of the potential concerns associated with research on vulnerable populations is important for identifying whatever solution is most appropriate for a specific case.

\subsection{Confidentiality}

To protect subjects' interests and future well-being, their identity must be protected in a scientific study. This is done using the dual principles of anonymity and confidentiality. Anonymity implies that the researcher or readers of the final research report or paper cannot identify a given response with a specific respondent. An example of anonymity in scientific research is a mail survey in which no identification numbers are used to track who is responding to the survey and who is not. In studies of deviant or undesirable behaviors, such as drug use or illegal music downloading by students, truthful responses may not be obtained if subjects are not assured of anonymity. Further, anonymity assures that subjects are insulated from law enforcement or other authorities who may have an interest in identifying and tracking such subjects in the future.

In some research designs such as face-to-face interviews, anonymity is not possible. In other designs, such as a longitudinal field survey, anonymity is not desirable because it prevents the researcher from matching responses from the same subject at different points in time for longitudinal analysis. Under such circumstances, subjects should be guaranteed confidentiality, in which the researcher can identify a person's responses, but promises not to divulge that person's identify in any report, paper, or public forum. Confidentiality is a weaker form of protection than anonymity, because social research data do not enjoy the ``privileged communication'' status in United State courts as do communication with priests or lawyers. For instance, two years after the Exxon Valdez supertanker spilled ten million barrels of crude oil near the port of Valdez in Alaska, the communities suffering economic and environmental damage commissioned a San Diego research firm to survey the affected households about personal and embarrassing details about increased psychological problems in their family. Because the cultural norms of many Native Americans made such public revelations particularly painful and difficult, respondents were assured confidentiality of their responses. When this evidence was presented to court, Exxon petitioned the court to subpoena the original survey questionnaires (with identifying information) in order to cross-examine respondents regarding their answers that they had given to interviewers under the protection of confidentiality and was granted that request. Luckily, the Exxon Valdez case was settled before the victims were forced to testify in open court, but the potential for similar violations of confidentiality still remains. Ann Cummings\cite{cummings1992exxon} wrote an excellent review of this incident.

In one extreme case, Rik Scarce, a graduate student at Washington State University, conducted participant observation studies of animal rights activists, and chronicled his findings in a $ 1990 $ book called \textit{Ecowarriors: Understanding the Radical Environmental Movement}\cite{scarce2016eco}. In $ 1993 $, Scarce was called before a grand jury to identify the activists he studied. The researcher refused to answer grand jury questions, in keeping with his ethical obligations as a member of the \textit{American Sociological Association}, and was forced to spend $ 159 $ days at Spokane County Jail. To protect themselves from travails similar to Rik Scarce, researchers should remove any identifying information from documents and data files as soon as they are no longer necessary. In $ 2002 $, the United States Department of Health and Human Services issued a ``Certificate of Confidentiality'' to protect participants in research project from police and other authorities. Not all research projects qualify for this protection, but this can provide an important support for protecting participant confidentiality in many cases.

\subsection{Disclosure}

Usually, researchers have an obligation to provide some information about their study to potential subjects before data collection to help them decide whether or not they wish to participate in the study. For instance, researchers have an obligation to answer questions about who is conducting the study, for what purpose, what outcomes are expected, and who will benefit from the results. However, in some cases, disclosing such information may potentially bias subjects' responses. For instance, if the purpose of a study is to examine to what extent subjects will abandon their own views to conform with ``groupthink'' and they participate in an experiment where they listen to others' opinions on a topic before voicing their own, then disclosing the study's purpose before the experiment will likely sensitize subjects to the treatment. Under such circumstances, even if the study's purpose cannot be revealed before the study, it should be revealed in a debriefing session immediately following the data collection process, with a list of potential risks or harm borne by the participant during the experiment.


\subsection{Reporting}

Researchers also have ethical obligations to the scientific community on how data are analyzed and reported in their study. Unexpected or negative findings should be fully disclosed, even if they cast some doubt on the research design or the findings. Similarly, many interesting relationships are discovered after a study is completed by chance or data mining. It is unethical to present such findings as the product of deliberate design. In other words, hypotheses should not be designed in positivist research after the fact based on the results of data analysis, because the role of data in such research is to test hypotheses, and not build them. It is also unethical to ``carve'' their data into different segments to prove or disprove their hypotheses of interest, or to generate multiple papers claiming different data sets. Misrepresenting questionable claims as valid based on partial, incomplete, or improper data analysis is also dishonest. Science progresses through openness and honesty and researchers can best serve science and the scientific community by fully disclosing the problems with their research so that they can save other researchers from similar problems.

\section{Institutional Review Board}

The \gls{irb} is tasked with ensuring that the rights and welfare of human research subjects will be protected at all institutions, including universities, hospitals, nonprofit research institutions, and other organizations that receive federal support for research. \glspl{irb} typically consist of members from a variety of disciplines, such as sociology, economics, education, social work, and communications. Most \glspl{irb} also include representatives from the community in which they reside. For example, representatives from prisons, hospitals, or treatment centers might sit on the \gls{irb} of nearby universities. The diversity of membership helps to ensure that the complex ethical issues that may arise from human subjects research will be considered fully by a knowledgeable and experienced panel. Investigators conducting research on human subjects are required to submit proposals outlining their research plans to \gls{irb} for review and approval prior to beginning their research. 

The \gls{irb} approval process requires completing a structured application providing complete information about the research project, the researchers (principal investigators), and details on how the subjects' rights will be protected. Additional documentation such as the Informed Consent form, research questionnaire or interview protocol may be needed. The researchers must also demonstrate that they are familiar with the principles of ethical research by providing certification of their participation in an research ethics course. Data collection can commence only after the project is cleared by the \gls{irb} review committee.

Even students who conduct research that involve human subjects must have their proposed work reviewed and approved by the \gls{irb} before beginning any research (though, at some universities exceptions are made for student projects with no danger to the participants and will not be shared outside of the classroom).

It may be surprising to find out that \glspl{irb} are not always popular or appreciated by researchers. In some cases, researchers are concerned that the local \gls{irb} has expertise in biomedical and experimental research but not business or social fields. Unfortunately, business research is often open-ended and that can be problematic for an \gls{irb}. The members of \glspl{irb} often want to know in advance exactly who will be observed, where, when, and for how long, whether and how they will be approached, exactly what questions they will be asked, and what predictions the researcher has for her or his findings. Providing this level of detail for a year-long participant observation within an activist group of $ 200 $-plus members, for example, would be extraordinarily frustrating for the researcher in the best of cases and most likely would prove to be impossible. Of course, \glspl{irb} do not intend to have researchers avoid studying controversial topics or avoid using certain methodologically sound data-collection techniques, but, unfortunately, that is sometimes the result. The solution is not to do away with review boards, which serve a necessary and important function, but instead to help educate \gls{irb} members about the variety of research methods and topics covered by business, sociology, and other social scientists.

\section{Professional Codes of Ethics}

Most professional associations have established and published formal codes of conduct describing what constitute acceptable and unacceptable professional behavior of their member researchers. The following codes are examples for researchers engaging in business research.

\begin{itemize}
	\item Academy of Management (AoM) Code of Ethical Conduct. \url{http://www.aomonline.org/governanceandethics/aomrevisedcodeofethics.pdf}
	\item Chartered Association of Business Schools \url{https://charteredabs.org/wp-content/uploads/2015/02/abs_ethics_guide_-_2012.pdf}
	\item Market Research Society \url{https://www.mrs.org.uk/standards/code_of_conduct/}
\end{itemize}

It may also be useful to consider the codes developed by social science researchers.

\begin{itemize}
	\item Social Research Association (SRA) \url{http://the-sra.org.uk/research-ethics/ethics-guidelines/}
	\item American Sociological Association (ASA) \url{http://www.asanet.org/membership/code-ethics}
	\item American Psychological Association (APA) \url{http://www.apa.org/ethics/code/index.aspx}
\end{itemize}

As an example, following is the summarized \textit{Marketing Research Association's} (MRA) ``Code of Marketing Research Standards''\cite{mra2018standards}. The code is a $ 20 $ page document that includes $ 42 $ principles and is divided into three articles, an enforcement FAQ, and two appendices.

\begin{enumerate}
	\item Article I: Responsibility to Respondents and Prospective Respondents. 
	
	\begin{itemize}
		\item General Conduct. This article focuses on how to treat the respondents in a research project. It includes requirements like protect their right to drop out of a research project and their right to privacy.
		\item Purpose of Use. This article requires researchers to obtain a consent form and protect respondent information from improper use, like solicitations.
		\item Transparancy. This article requires researchers to be honest with respondents and make the research method transparant. It includes things like not collecting information without the respondent's knowledge and keeping an internal ``do not call'' list so respondents can opt out of future contacts.
		\item Technical Compliance. This article focuses on legal and other matters, like adhering to all state laws for projects that cross state borders and being especially careful with vulnerable populations, like children.
	\end{itemize}
	
	\item Article II: Responsibilities to Clients and Vendors. This article requires researchers to maintain a trusted relationship with clients and vendors and refrain from engaging in unacceptable practices with any research partner.
	
	\item Article III: Professional Responsibilities. Researchers are required to report research results accurately and honestly and not falsify or omit data.
	
\end{enumerate}

\section{Ethical Issues in Research}

\subsection{Introduction}

Ethics, when applied to social research, is concerned with the creation of a trusting relationship between those who are researched and the researcher. To ensure that trust is established it is essential that communication is carefully planned and managed, that risks are minimized, and benefits are maximised.\footnote{This section of the book was adapted from information found at the \textit{Kirklees Council} website\cite{kirklees2019ethical}.}

In developing a trusting relationship, researchers adhere to a number of ethical principles which they apply to their work---namely beneficence, autonomy, nonmaleficence, justice, veracity, and privacy.

\subsection{Beneficence (doing good)}

Research should only be carried out if some sort of benefit or good can be derived from it, (\ie contribution to the general body of knowledge or improved service/treatment). Therefore, the question of whether or not a research project is worth undertaking should always be uppermost in the mind of the researcher. If no benefit can be derived then the project is unethical.

\subsection{Autonomy (self-rule)}

Researchers have an obligation to disclose information at a level that participants can understand so that they can either intelligently agree or refuse to participate. In essence, autonomy is concerned with the concept of informed consent whereby people who agree to take part in a study know what they are agreeing to and authorize the researcher to collect information without any form of coercion.

\subsection{Nonmaleficence (do no harm)}

The principle of nonmaleficence places an obligation on researchers to not harm others or expose people to unnecessary risks. Harm can come in many forms, from blows to self-esteem to ``looking bad'' to others, to loss of funding or earnings, to boredom, frustration, or time wasting. In extreme cases, research projects may even lead to physical harm. It is good practice to assume that every research project will potentially involve some form of harm and to consider in advance how best to deal with it.

\subsection{Justice (Fairness)}

This principle implies that everyone should be treated fairly and equally. Researchers should be careful to treat all subjects impartially and without favoritism. Of course, some research projects may be intentionally designed to offer some sort of treatment to one group and not the other so its effectiveness can be measured, but as much as possible, do not discriminate among subjects.

\subsection{Veracity (truth telling)}

This principle concerns telling the truth whereby the researcher is required to provide comprehensive and accurate information in a manner that enhances understanding. For example, if the researcher says that a questionnaire will take ten minutes to complete then the questionnaire should take ten minutes and not 15. Researchers should always be honest with participants and keep all promises made.

\subsection{Privacy}

Privacy concerns the respect for limited access to another person, be it physically, emotionally, or cognitively. For example, although participants grant access to their thoughts and feelings when they agree to participate, they do not agree to unlimited access. Therefore, they always have the right to decline to talk about certain issues or to answer specific questions.

Confidentiality is an extension of privacy that relates specifically to the agreement made between the researcher and participants about what can and cannot be done with information collected over the course of a project. In many cases, confidentiality will be determined by various legal constraints, but even in the absence of the law, information gathered should be protected.

\subsection{Frequently asked questions}

\subsubsection{What is meant by informed consent?}

Being informed means that participants are told everything that might or will occur during a study in a way in which they can understand. Giving consent implies that a) the agreement to participate is voluntary, free from coercion and undue influence and b) that the person providing the consent is competent to make a rational and mature judgment about taking part. If the criteria of being informed and giving consent are met then informed consent is said to be given.

\subsubsection{Does consent have to be in writing?}

It is a good practice to have consent in writing, and many \glspl{irb} will require written consent forms. In practice, however, this is not always possible especially, when undertaking focus groups or field observations. The convention here is to go through the consent procedure with the group and record video any objections. For field research, informed consent should be the goal as much as that is practical. To prevent breaches of confidentially, consent forms with personal identifiable information attached should be stored in a locked container away other from information about the project.

\subsubsection{What information should be included on a consent form?}

There are no hard and fast rules; however, as a rough guide, the following sorts of things should be included:

\begin{itemize}
	\item a heading stating the title, the organization carrying out the research and the name of the researcher.
	\item a statement of agreement to participate.
	\item a statement that indicates the length of time an activity is likely to take.
	\item a statement that indicates what will happen to the information collected.
	\item a statement about confidentiality and anonymity.
	\item confirmation that there is no obligation to take part and that participants have the right to withdraw at any time or not answer questions.
	\item signatures and date.
\end{itemize}

The following optional statements may be included.

\begin{itemize}
	\item a statement that the use of recording equipment has been explained.
	\item a statement that a leaflet has been provided and that the information has been read and understood.
	\item a statement that permission has been granted to contact participants in the future if that is necessary.
	\item a statement that indicates whether permission has been granted for participants' names to be added to a database etc.
\end{itemize}

\section{Research with children and young people}

\subsection{Introduction}

If a research project involves children and young people there are a number of issues that must be considered in addition to the usual methodological, ethical and practical concerns of any research project. More time, thought and planning is therefore required when compared to adult respondents. This section provides a brief overview of some of the specific issues that need to be considered.\footnote{This section of the book was adapted from information found at the \textit{Kirklees Council} website\cite{kirklees2019children}.}

\subsection{Defining children and young people}

The Office for Human Research Protections of the US Department of Health \& Human Services defines children as ``persons who have not attained the legal age for consent to treatments or procedures involved in the research, under applicable law of the jurisdiction in which the research will be conducted. Generally the law considers any person under $ 18 $ years old to be a child.''\cite{hhs2018children}

\subsection{Is there a minimum age for conducting research?}

While the US Department of Healt \& Human Services does not specify a minimum age, conducting research with very young children should be avoided unless absolutely necessary and certainly should not be undertaken by a non-specialist.

\subsection{How are children and young people recruited for a research project?}

Recruitment of children and young people to take part in research almost always needs to be done via a ``gatekeeper.'' The gatekeeper will usually be a responsible adult, \ie the person responsible for protecting the child/young person's safety and welfare at the time of the research. Gatekeepers will vary in different contexts but examples might include a parent, a teacher, a caregiver, or a youth worker. It is important to consult and involve gatekeepers during the planning stages of any research project since they will usually be the ones who provide the initial consent to approach children/young people to take part.

\subsection{Gaining consent}

In general, children under $ 18 $ must not be consulted without the consent of a parent, guardian, or responsible adult. Where research is being undertaken within a school environment it is suggested that consent is sought from parents/guardians as well as the teacher or other responsible adult at the school.

Consent is a two stage process since it must always be obtained from the child/young person themselves as well as the responsible adult.

\begin{itemize}
	\item \textit{Stage $ 1 $}. The responsible adult must provide consent to approach potential participants.

	\item \textit{Stage $ 2 $} The child/young person must give their own consent to take part in the research and have the opportunity to decline if they wish.
\end{itemize}

\subsection{Informed consent}

It is important to introduce the purpose and aims of the research clearly to ensure that both responsible adults and children/young people are able to give their informed consent to take part. This means that they must be given enough information to understand what is being asked of them. This introductory information should be in writing wherever possible and contact details for the person undertaking the research should always be provided.

It is not essential for consent to be provided in writing unless the subject matter is potentially sensitive, but it is often advisable in order to create an audit trail. The name, relationship and role (\eg parent) of the responsible adult giving consent should always be recorded in writing.

\subsection{Different scenarios for giving consent}

\begin{itemize}
	\item Postal questionnaires: these should be sent to the responsible adult in the first instance and not the child. Space should be provided for the responsible adult to sign that they have given their consent for the child to complete the questionnaire. 

	\item Telephone interviews: the consent of the responsible adult may be obtained verbally, but a written record should be retained and that record sent to the responsible adult on request.

	\item Qualitative research: written consent forms should be issued to parents/guardians at the recruitment stage asking for permission to ask children to take part.

	\item Online research: a notice explaining that consent is required must be posted, with an explanation of the procedure for obtaining consent. Consent should be verified by letter or phone if it is provided via email. Respondents should be asked to give their age before providing any other personal information and if the age given is under $ 18 $, they must be excluded from providing further information until the appropriate consent has been obtained.
\end{itemize}

\subsection{What else should be considered?}

\begin{itemize}
	\item \textit{Subject matter}. Extra care must be taken when consulting over sensitive or potentially contentious topics, for example race, religion or alcohol/drug use.

	\item \textit{Questionnaire design}. The content should be appropriate to the age of respondents and relevant to their experience. The language used should be simple but not patronizing.

	\item \textit{Qualitative methods}. Group-based activities can be used to encourage participation and promote discussion and the presence of peers may put people at ease. One-to-one interviews are not recommended for young children but can work well with teenagers. ``Friendship pair'' interviews can be another useful technique.

	\item \textit{Venue}. Research should only be conducted in safe and appropriate environments where children/young people feel safe and comfortable.

	\item \textit{Personal safety}. Precautions must be taken to ensure that research does not harm or adversely affect participants. To protect the children, interviewers who will have contact with them should be checked against information held by law authorities.

	\item \textit{Incentives}. Any incentives used should be suitable for the age of the child/young person and appropriate to the task required.

	\item \textit{Feedback}. As with all research, the results must be fed back to participants. Asking for feedback on the findings and their experience of being consulted might also help improve knowledge of how to engage children/young people in the future.

\end{itemize}

\subsection{Commissioning a specialist}

Since research with children and young people requires specialist approaches, it is often advisable to commission someone with the appropriate expertise in this area to carry out the work.

\subsection{Involving Children/Young People}

Experts on involving children/young people in research projects offer these recommendations.

\begin{itemize}
	\item Work with a whole range of approaches

	\item Avoid creating an ``elite'' who are assumed to represent children/young people on every issue

	\item Meet children/young people on their own territory, at times they choose and in ways that make sense to them

	\item Give children/young people the chance to influence not just the answers, but the questions

	\item Work on the basis that children/young people are the experts on how to involve them and take their evaluation seriously

	\item Provide opportunities for enjoyment and the chance to build relationships

	\item Guarantee a feedback loop

\end{itemize}

\section{Best Practices}

The following list of ``best practices'' in ethics can be derived from the information in this chapter.

\begin{itemize}

	\item Be honest. Nearly every code of ethical conduct boils down to only a few simple principles and one of the most common is to conduct research with honesty and integrity. Researchers who are honest will rarely make the wrong ethical decision.
	\item Care. Researchers must be careful in every aspect of the research project. Many ethical problems arise when a researcher ``takes shortcuts'' when gathering and analyzing data.
	\item Be respectful. Researchers must respect the intellectual property of others. If colleagues assist with a project then they should be acknowledged. Closely aligned with respect is to avoid plagiarism.
	\item Maintain confidentiality. If participants expect their participation to be confidential then researchers must take great pains to protect that anonymity.  
	\item Know ethical principles. Researchers owe it to themselves, their colleagues, and participants to be knowledgeable about ethical principles. More importantly, they need to know where their research project fits within an ethical framework and their responsibilities to both the research community and the participants.
	\item Disclosure. Be open with participants about all facets of the research project. They should know the goals of the project, how the data will be protected and analyzed, and how the results will be shared. This open approach will also lead to informed consent to participate. Finally, every research project should include some sort of debriefing plan so participants can achieve a sense of closure when the project is finished.
\end{itemize}

\noindent\begin{minipage}{\textwidth}
	\begin{figure}[H]
		\centering
		\includegraphics[width=.85\linewidth]{gfx/Sampling_Of_Research}
		\caption*{}
		\label{03:sampling_of_research}
	\end{figure}
	\vspace{-10.0ex} %Note: neg 10 heights of letter x
	\section{A Sampling of Research}
\end{minipage}

\subsection{Zimbardo's Prison}

In August 1971, Dr. Zimbardo\footnote{Interestingly, Philip Zimbardo and Stanley Milgram, mentioned elsewhere in this chapter, were classmates at James Monroe High School in the Bronx. Zimbardo recalled the Milgram was ``\ldots considered the smartest kid and I voted the most popular.''\cite{zimbardo2000reflections}} started a social science experiment that has been condemned for ethical violations\cite{zimbardo2000reflections}. The experiment involved selecting $ 24 $ students from $ 70 $ who had volunteered to participate in a study of prison life. Those students were randomly assigned to one of two groups: prisoners and guards. The guards helped build a mock prison in the basement of the Stanford University psychology department then nine prisoners were assigned to three cells in that prison. All students signed informed consent forms that indicated some of their basic rights would be violated if they were selected to be prisoners and that only minimally adequate diet and health care would be provided. 

The first day was rather uneventful, but after that day the guards steadily increased their coercive and aggressive tactics and resorted to humiliation and dehumanization of the prisoners. Within $ 36 $ hours of the initial ``arrest'' the first prisoner had to be released because of extreme stress reactions like crying and cursing. Over the next three days, three other prisoners had to be released due to acting ``crazy.'' The guards began to execute a number of ``controlling'' practices like waking the prisoners during the night for ``counts,'' basically depriving them of REM sleep. They became brutal and locked misbehaving prisoners in ``solitary confinement'' (a closet), made them perform meaningless physical activities (like jumping jacks), and even sprayed them with high-pressure fire extinguishers.

The experiment was ended after only six days rather than the planned two-week study because, in the words of Zimbardo, ``\ldots too many normal young men were behaving pathologically as powerless prisoners or as sadistic, all-powerful guards.'' The tipping point for the experiment came after a recently graduated Ph.D. came to the prison to assist with interviews. She was not part of the experiment from the start and became emotionally upset and angry over the madness that she witnessed. She convinced Zimbardo to end the experiment for the ``well-being of the young men entrusted to our care as research participants.'' 

The ethics of this experiment has been debated for decades. On the one hand it was conducted using the guidelines promulgated by the Human Subjects Review Board. In fact, it was that board that required fire extinguishers be added to the prison since there were limited emergency escape routes. Additionally, the participants were told in advance to expect their rights to be suspended. Finally, the prisoners were ``visited'' regularly by their parents, a priest, friends, a public defender, and many graduate students and staff of the psychology department, and none of those people raised any alarms.

On the other hand, it seems problematic that one group of humans was permitted to inflict pain and humiliation on another group for an extended period of time. The prisoners experienced social and psychological pain, but even the guards had to live with the pain of knowing that they were inflicting suffering on a peer who had done nothing wrong.

\subsection{Target}

Target stores received considerable negative publicity for using data mining and market basket analysis to identify women who are pregnant\footnote{This incident was widely reported in the popular press, including Forbes \cite{hill2012target} and the New York Times \cite{duhigg2012companies}.}. 

Market Basket analysis is attempting to determine the types of products that are typically purchased together, they are in the same ``market basket.'' For example, beer and potato chips are often purchased together so if a store were running a special sale on beer it my also have a promotion for potato chips. Of course, formal market basket analysis is much more in-depth than that and can find odd relationships between numerous products. When market basket analysis gets personal the results can be ethically interesting.

For example, Target, Inc., like many large chains, have customer loyalty programs where customers can use a card or phone number to get special discounts on certain items. Of course, the entire shopping trip is then recorded in the company's database and market basket analysis can determine what this one specific customer is likely to purchase.

Target's problem began when a father in Minneapolis complained that his teen-aged daughter had received pregnancy-related coupons. He felt that the coupons were inappropriate and promoted teen pregnancy. He later found out that his daughter was, in fact, pregnant and apologized to the store manager. Target had been able to use market basket research to determine that the girl was purchasing the types of items that pregnant women purchase so they sent her targeted ads for things pregnant women need.

To build its predictive models, Target focused on women who had signed up for the baby registry. They then compared those women’s purchases with all customers. Twenty-five variables were found that could identify pregnant women and even when their babies were due to an amazing degree of accuracy. The variables included things like buying large quantities of unscented lotions, supplements like calcium, magnesium, and zinc, and washcloths. The analytics were good enough that Target found that pregnant women tend to buy more hand sanitizers and washcloths as they get close to their delivery date. Target used these predictions to identify which women should receive specific coupons. 

After that particular controversy settled down, Target used the same analytics to predict when people were getting married and they would send out invitations to join the bridal registry before the marrying couple had a chance to tell their parents.

In response to the negative press, Target no longer sends out ads for only one specific item, but they have become much more devious. If they know, for example, that some woman is pregnant then the circular going to that particular house will have ads for garden implements and coffee, but there will be a number of prominent ads for items that a pregnant woman would need. The house next door would get a different circular with prominent ads for, maybe, party items.

The market basket analysis being done by Target, and other stores, is perfectly legal; however, it does raise ethical questions concerning customer privacy and informed consent. 

\subsection{Facebook}

In June $ 2014 $, Kramer, Guillory, and Hancock (employees of Facebook) published a paper describing their experiment with Facebook data\cite{kramer2014experimental}. The purpose of the experiment was to determine if emotional contagion occurs in social media. For the experiment, they manipulated the content in News Feeds for a select group of Facebook customers. In some cases they reduced the number of positive posts and in others the number of negative posts. They found that when positive posts were reduced people produced fewer positive posts; and when negative posts were reduced the opposite happened. They suggested that emotions expressed by others on Facebook influence our own emotions, ``constituting experimental evidence for massive-scale contagion via social networks.''

This study was widely reported in the popular press and was the impetus for a number of investigations into privacy and the way Facebook controls data. Specifically, there was wide-spread criticism about the lack of informed consent and opportunity for users to opt out of the experiment. However, the study's authors noted that the experiment was ``consistent with Facebook's Data Use Policy, to which all users agree prior to creating an account on Facebook, constituting informed consent for this research.'' There was also discussion about oversight for the research project. The fact is that this research was conducted by Facebook, Inc. for its internal use and falls outside the oversight of a university research department. Moreover, as a private company, Facebook is under no obligation to conform to the provisions of the U.S. Department of Health and Human Services Policy for the Protection of Human Research Subjects. In short, while the company could choose to follow ethics best practices concerning informed consent and participant opt out, they are not required to do so.\footnote{It is understood that Facebook did not break any laws nor physically harm any customers with this research project.}

\section{Key Takeaways}

Students who are embarking on a research project within a university setting should follow  the code of ethics from the \gls{irb} of their institution. Researchers who are working independently should join an appropriate professional organization (depending on the type of research they are conducting) adopt the code of ethics from that organization. Following are the major topics covered in this chapter.

\begin{center}
	\begin{tkawybox}{Research Ethics}
		\begin{itemize}
			\setlength{\itemsep}{0pt}
			\setlength{\parskip}{0pt}
			\setlength{\parsep}{0pt}
			
			\item Define ethics as the ``...moral principles that govern a person's behaviour or the conducting of an activity.''
			\item Discuss five primary ethical principles: voluntary participation, informed consent, confidentiality, disclosure, and reporting.
			\item Discuss the unique ethical considerations surrounding research on humans
			\item Define Institutional Review Board and discuss the types of issues that may accompany those boards.
			\item Discuss various professional codes of ethics
		\end{itemize}
	\end{tkawybox}
\end{center}


