%*****************************************
\chapter{Research Ethics}\label{03:ethics}
%*****************************************
%TODO Status: Text Draft - Checked against the LO version

\section{Introduction}

Ethics is defined by Webster's dictionary as conformance to the standards of conduct of a given profession or group. Such standards are often defined at a disciplinary level though a professional code of conduct, and sometimes enforced by university committees called an \textit{Institutional Review Board}. Even if not explicitly specified, scientists are still expected to be aware of and abide by general agreements shared by the research community on what constitutes acceptable and non-acceptable behaviors in the professional conduct of their disciplines. For instance, researchers should not manipulate their data collection, analysis, and interpretation procedures in a way that contradicts the principles of science or the scientific method or advances their personal agenda.

Why is research ethics important? Because, science has often been manipulated in unethical ways by people and organizations to advance their private agenda and engaging in activities that are contrary to the norms of scientific conduct. A classic example is pharmaceutical giant Merck’s drug trials of Vioxx, where the company hid the fatal side-effects of the drug from the scientific community, resulting in $ 3468 $ deaths of Vioxx recipients, mostly from cardiac arrest. In $ 2010 $, the company agreed to a $ \$4.85 $ billion settlement and appointed two independent committees and a chief medical officer to monitor the safety of its drug development process. Merck's conduct was unethical and violation the scientific principles of data collection, analysis, and interpretation\footnote{See Green, Ronald M. "Direct-to-consumer advertising and pharmaceutical ethics: The case of Vioxx." Hofstra L. Rev. 35 (2006): 749, for an excellent and understandable version of the ethical issues in the Merck case.}

Ethics is the moral distinction between right and wrong and what is unethical may not necessarily be illegal. A scientist's conduct may not be culpable in the eyes of the law but may still lead to disciplinary hearings, professional notoriety, and even job loss on the grounds of professional misconduct. These ethical norms may vary from one society to another and this book uses the ethical standards as applied to scientific research in Western countries.

\section{Ethical Principles}

Some of the expected tenets of ethical behavior that are widely accepted within the scientific community include:

\subsection{Voluntary Participation}

Subjects in a research project must be aware that their participation in the study is voluntary, that they have the freedom to withdraw from the study at any time without unfavorable consequences, and they are not harmed as a result of their participation or non-participation in the project. The most flagrant violations of the voluntary participation principle are probably forced medical experiments conducted by Nazi researchers on prisoners of war during World War II, as documented in the post-War Nuremberg Trials (these experiments also originated the term ``crimes against humanity''). Less known violations include the Tuskegee syphilis experiments conducted by the U.S. Public Health Service between $ 1932-1972 $, in which nearly $ 400 $ impoverished African-American men suffering from syphilis were denied treatment even after penicillin was accepted as an effective treatment of syphilis, and subjects were presented with false treatments such as spinal taps as cures for syphilis. Even if subjects face no mortal threat, they should not be subjected to personal agony as a result of their participation. In $ 1971 $, psychologist Philip Zambardo created the \textit{Stanford Prison Experiment}, where Stanford students recruited as subjects were randomly assigned to roles such as prisoners or guards. When it became evident that student prisoners were suffering psychological damage as a result of their mock incarceration and student guards were exhibiting sadism that would later challenge their own self-image the experiment was terminated.

As a less egregious example, instructors often ask students to fill out a questionnaire of some sort and inform the students that their participation is voluntary. This activity must be designed in such a way that students do not fear that their non-participation will hurt their grade in any way. For instance, it is unethical to provide bonus points for participation and no bonus points for non-participation because it places non-participants at a distinct disadvantage. To avoid such circumstances, instructors may provide an alternate task for non-participants so that they can earn the same number of bonus points without participating in the research study or by providing bonus points to everyone irrespective of their participation or non-participation. 

\subsection{Informed Consent}

All participants in a study must receive and sign an Informed Consent form that clearly describes their right to not participate and right to withdraw, before their responses in the study can be recorded. In a medical study, this form must also specify any possible risks to subjects from their participation. For subjects under the age of $ 18 $, this form must be signed by their parent or legal guardian. Researchers must retain these informed consent forms for a period of time (often three years) after the completion of the data collection process in order comply with the norms of scientific conduct in their discipline or workplace.

The consent form, itself, must not waive, or even appear to waive, any of the subject's legal rights. Subjects also cannot release a researcher or institution from any legal liability should something go wrong during the course of their participation in the research. Because sociological research does not typically involve asking subjects to place themselves at risk of physical harm by, for example, taking untested drugs or consenting to new medical procedures, sociological researchers do not often worry about potential liability associated with their research projects. However, their research may involve other types of risks. For example, if a researcher intentionally or accidentally reveals the identity of subjects who admit unusual sexual behavior the subject's social standing, marriage, custody rights, or employment could be jeopardized.

In some cases, subjects are asked to sign a physical consent form indicating that they have read it and fully understand its contents. In other cases, subjects are simply provided a copy of the consent form and researchers are responsible for making sure that subjects have read and understand the form before proceeding with any kind of data collection. 

One last point to consider when preparing to obtain informed consent is that not all potential research subjects are considered equally competent or legally allowed to consent to participate in research. These subjects are sometimes referred to as members of vulnerable populations, people who may be at risk of experiencing undue influence or coercion. The rules for consent are more stringent for vulnerable populations. For example, minors must have the consent of a legal guardian in order to participate in research. In some cases, the minors themselves are also asked to participate in the consent process by signing special, age-appropriate consent forms designed specifically for them. Prisoners and parolees also qualify as vulnerable populations. Concern about the vulnerability of these subjects comes from the very real possibility that prisoners and parolees could perceive that they will receive some highly desired reward, such as early release, if they participate in research. While gaining consent from vulnerable populations can be challenging failing to work with those groups ensures that their stories are never told. While there is no easy solution to this double-edged sword, an awareness of the potential concerns associated with research on vulnerable populations is important for identifying whatever solution is most appropriate for a specific case.

\subsection{Confidentiality}

To protect subjects' interests and future well-being, their identity must be protected in a scientific study. This is done using the dual principles of anonymity and confidentiality. Anonymity implies that the researcher or readers of the final research report or paper cannot identify a given response with a specific respondent. An example of anonymity in scientific research is a mail survey in which no identification numbers are used to track who is responding to the survey and who is not. In studies of deviant or undesirable behaviors, such as drug use or illegal music downloading by students, truthful responses may not be obtained if subjects are not assured of anonymity. Further, anonymity assures that subjects are insulated from law enforcement or other authorities who may have an interest in identifying and tracking such subjects in the future.

In some research designs such as face-to-face interviews, anonymity is not possible. In other designs, such as a longitudinal field survey, anonymity is not desirable because it prevents the researcher from matching responses from the same subject at different points in time for longitudinal analysis. Under such circumstances, subjects should be guaranteed confidentiality, in which the researcher can identify a person's responses, but promises not to divulge that person's identify in any report, paper, or public forum. Confidentiality is a weaker form of protection than anonymity, because social research data do not enjoy the ``privileged communication'' status in United State courts as do communication with priests or lawyers. For instance, two years after the Exxon Valdez supertanker spilled ten million barrels of crude oil near the port of Valdez in Alaska, the communities suffering economic and environmental damage commissioned a San Diego research firm to survey the affected households about personal and embarrassing details about increased psychological problems in their family. Because the cultural norms of many Native Americans made such public revelations particularly painful and difficult, respondents were assured confidentiality of their responses. When this evidence was presented to court, Exxon petitioned the court to subpoena the original survey questionnaires (with identifying information) in order to cross-examine respondents regarding their answers that they had given to interviewers under the protection of confidentiality and was granted that request. Luckily, the Exxon Valdez case was settled before the victims were forced to testify in open court, but the potential for similar violations of confidentiality still remains\footnote{See Cummings, Ann D. "The Exxon Valdez Oil Spill and the Confidentiality of Natural Resource Damage Assessment Data." Ecology Law Quarterly 19.2 (1992): 363-412, for an excellent review of the intersection between research and law in the Exxon Valdez case.}.

In one extreme case, Rik Scarce, a graduate student at Washington State University, conducted participant observation studies of animal rights activists, and chronicled his findings in a $ 1990 $ book called \textit{Ecowarriors: Understanding the Radical Environmental Movement}. In $ 1993 $, Scarce was called before a grand jury to identify the activists he studied. The researcher refused to answer grand jury questions, in keeping with his ethical obligations as a member of the \textit{American Sociological Association}, and was forced to spend $ 159 $ days at Spokane County Jail. To protect themselves from travails similar to Rik Scarce, researchers should remove any identifying information from documents and data files as soon as they are no longer necessary. In $ 2002 $, the United States Department of Health and Human Services issued a ``Certificate of Confidentiality'' to protect participants in research project from police and other authorities. Not all research projects qualify for this protection, but this can provide an important support for protecting participant confidentiality in many cases.

\subsection{Disclosure}

Usually, researchers have an obligation to provide some information about their study to potential subjects before data collection to help them decide whether or not they wish to participate in the study. For instance, researchers have an obligation to answer questions about who is conducting the study, for what purpose, what outcomes are expected, and who will benefit from the results. However, in some cases, disclosing such information may potentially bias subjects' responses. For instance, if the purpose of a study is to examine to what extent subjects will abandon their own views to conform with ``groupthink'' and they participate in an experiment where they listen to others' opinions on a topic before voicing their own, then disclosing the study's purpose before the experiment will likely sensitize subjects to the treatment. Under such circumstances, even if the study's purpose cannot be revealed before the study, it should be revealed in a debriefing session immediately following the data collection process, with a list of potential risks or harm borne by the participant during the experiment.


\subsection{Reporting}

Researchers also have ethical obligations to the scientific community on how data is analyzed and reported in their study. Unexpected or negative findings should be fully disclosed, even if they cast some doubt on the research design or the findings. Similarly, many interesting relationships are discovered after a study is completed by chance or data mining. It is unethical to present such findings as the product of deliberate design. In other words, hypotheses should not be designed in positivist research after the fact based on the results of data analysis, because the role of data in such research is to test hypotheses, and not build them. It is also unethical to ``carve'' their data into different segments to prove or disprove their hypotheses of interest, or to generate multiple papers claiming different data sets. Misrepresenting questionable claims as valid based on partial, incomplete, or improper data analysis is also dishonest. Science progresses through openness and honesty and researchers can best serve science and the scientific community by fully disclosing the problems with their research so that they can save other researchers from similar problems.

\section{Research on Humans}

The U.S. Department of Health and Human Services defines a human subject as ``...a living individual about whom an investigator (whether professional or student) conducting research obtains (1) Data through intervention or interaction with the individual, or (2) Identifiable private information.\footnote{\url{https://www.hhs.gov/ohrp/regulations-and-policy/regulations/45-cfr-46/index.html\#46.102}.}.'' In some states, human subjects also include deceased individuals and human fetal materials. 

Nonhuman research subjects, on the other hand, are objects or entities that investigators manipulate or analyze in the process of conducting research. In business research, nonhuman subjects typically include sources like newspapers, historical documents, advertisements, television shows, buildings, and even garbage.

Unsurprisingly, research on human subjects is regulated much more heavily than research on nonhuman subjects. However, there are ethical considerations that all researchers must consider regardless of their research subject.

\subsection{A Historical Look at Research on Humans}

Research on humans has not always been regulated in the way that it is today. The earliest documented cases of research using human subjects are of medical vaccination trials. One such case took place in the late $ 1700 $s, when scientist Edward Jenner exposed an eight-year-old boy to smallpox in order to identify a vaccine for the devastating disease\footnote{Riedel, Stefan. "Edward Jenner and the history of smallpox and vaccination." Proceedings (Baylor University. Medical Center) 18.1 (2005): 21.}. Medical research on human subjects continued without much law or policy intervention until the mid-$ 1900 $s when, at the end of World War II, a number of Nazi doctors and scientists were put on trial for conducting human experimentation during the course of which they tortured and murdered many concentration camp inmates\footnote{Faden, Ruth R., and Tom L. Beauchamp. A history and theory of informed consent. Oxford University Press, 1986.}. The trials, conducted in Nuremberg, Germany, resulted in the creation of the Nuremberg Code, a Ten-point set of research principles designed to guide doctors and scientists who conduct research on human subjects. Today, the Nuremberg Code guides medical and other research conducted on human subjects, including social scientific research.

Medical scientists are not the only researchers who have conducted questionable research on humans. In the $ 1960 $s, psychologist Stanley Milgram\footnote{\url{http://www.ernestoamaral.com/docs/dcp033-102/Milgram(1963).pdf}} conducted a series of experiments designed to understand obedience to authority in which he tricked subjects into believing they were administering an electric shock to other subjects. In fact, the shocks were not real at all but some participants experienced extreme emotional distress after the experiment. The realization that one is willing to administer painful shocks to another human being just because someone who looks authoritative has told you to do so might indeed be traumatizing even if you later learn that the shocks were not real.

Around the same time that Milgram conducted his experiments, sociology graduate student Laud Humphreys\footnote{\url{http://makelearninghappen.com/wp-content/uploads/Humphries-1970.pdf}} was collecting data for his dissertation research on the tearoom trade, the practice of men engaging in anonymous sexual encounters in public restrooms. Humphreys wished to understand who these men were and why they participated in the trade. To conduct his research, Humphreys offered to serve as a ``watch queen,'' the person who keeps an eye out for police and gets the benefit of being able to watch the sexual encounters, in a local park restroom where the tearoom trade was known to occur. What Humphreys did not do was identify himself as a researcher to his research subjects. Instead, he watched his subjects for several months, getting to know several of them, learning more about the tearoom trade practice and, without the knowledge of his research subjects, jotting down their license plate numbers as they pulled into or out of the parking lot near the restroom. Some time after participating as a watch queen, with the help of several insiders who had access to motor vehicle registration information, Humphreys used those license plate numbers to obtain the names and home addresses of his research subjects. Then, disguised as a public health researcher, Humphreys visited his subjects in their homes and interviewed them about their lives and health. Humphreys' research dispelled a good number of myths and stereotypes about the tearoom trade and its participants. He learned, for example, that over half of his subjects were married to women and many of them did not identify as gay or bisexual. However, once Humphreys’s work became public, the result created a major controversy at his home university, among sociologists in general, and among members of the public, as it raised public concerns about the purpose and conduct of sociological research.

These and other studies\footnote{\url{http://www.journalnma.org/article/S0027-9684(15)30517-4/pdf}} led to increasing public awareness of and concern about research on human subjects. In $ 1974 $, the U.S. Congress enacted the \textit{National Research Act}, which created the \textit{National Commission for the Protection of Human Subjects in Biomedical and Behavioral Research}. The commission produced \textit{The Belmont Report}\footnote{US Department of Health and Human Services. "The Belmont Report. 1979." URL: http://www. hhs. gov/ohrp/humansubjects/guidance/belmont. html [accessed 2014-10-20].}, a document outlining basic ethical principles for research on human subjects. The \textit{National Research Act} also required that all institutions receiving federal support establish \acp{IRB} to protect the rights of human research subjects. Since that time, many organizations that do not receive federal support but where research is conducted have also established review boards to evaluate the ethics of the research that they conduct.

\subsection{Institutional Review Boards}

The \acf{IRB} is tasked with ensuring that the rights and welfare of human research subjects will be protected at all institutions, including universities, hospitals, nonprofit research institutions, and other organizations that receive federal support for research. \acp{IRB} typically consist of members from a variety of disciplines, such as sociology, economics, education, social work, and communications. Most \acp{IRB} also include representatives from the community in which they reside. For example, representatives from nearby prisons, hospitals, or treatment centers might sit on the \acp{IRB} of nearby universities. The diversity of membership helps to ensure that the complex ethical issues that may arise from human subjects research will be considered fully by a knowledgeable and experienced panel. Investigators conducting research on human subjects are required to submit proposals outlining their research plans to \acp{IRB} for review and approval prior to beginning their research. 

The \ac{IRB} approval process requires completing a structured application providing complete information about the research project, the researchers (principal investigators), and details on how the subjects' rights will be protected. Additional documentation such as the Informed Consent form, research questionnaire or interview protocol may be needed. The researchers must also demonstrate that they are familiar with the principles of ethical research by providing certification of their participation in an research ethics course. Data collection can commence only after the project is cleared by the \ac{IRB} review committee.

Even students who conduct research on human subjects must have their proposed work reviewed and approved by the \ac{IRB} before beginning any research (though, at some universities exceptions are made for student projects that will not be shared outside of the classroom).

It may be surprising to find out that \acp{IRB} are not always popular or appreciated by researchers. In some cases, researchers are concerned that the local \ac{IRB} has expertise in biomedical and experimental research but not business or social fields. Unfortunately, business research is often open-ended and that can be problematic for an \ac{IRB}. The members of \acp{IRB} often want to know in advance exactly who will be observed, where, when, and for how long, whether and how they will be approached, exactly what questions they will be asked, and what predictions the researcher has for her or his findings. Providing this level of detail for a year-long participant observation within an activist group of $ 200 $-plus members, for example, would be extraordinarily frustrating for the researcher in the best of cases and most likely would prove to be impossible. Of course, \acp{IRB} do not intend to have researchers avoid studying controversial topics or avoid using certain methodologically sound data-collection techniques, but, unfortunately, that is sometimes the result. The solution is not to do away with review boards, which serve a necessary and important function, but instead to help educate \ac{IRB} members about the variety of research methods and topics covered by business, sociology, and other social scientists.

\section{Professional Code of Ethics}

Most professional associations have established and published formal codes of conduct describing what constitute acceptable and unacceptable professional behavior of their member researchers. The following codes are examples for researchers engaging in business research.

\begin{itemize}
	\item Academy of Management (AoM) Code of Ethical Conduct. \url{http://www.aomonline.org/governanceandethics/aomrevisedcodeofethics.pdf}
	\item Chartered Association of Business Schools \url{https://charteredabs.org/wp-content/uploads/2015/02/abs_ethics_guide_-_2012.pdf}
	\item Market Research Society \url{https://www.mrs.org.uk/standards/code_of_conduct/}
\end{itemize}

It may also be useful to consider the codes developed by social science researchers.

\begin{itemize}
	\item Social Research Association (SRA) \url{http://the-sra.org.uk/research-ethics/ethics-guidelines/}
	\item American Sociological Association (ASA) \url{http://www.asanet.org/membership/code-ethics}
	\item American Psychological Association (APA) \url{http://www.apa.org/ethics/code/index.aspx}
\end{itemize}

As an example, following is the summarized \textit{Marketing Research Association's} (MRA) ``Code of Marketing Research Standards''\footnote{The complete code of conduct is available online at \url{http://www.insightsassociation.org/issues-policies/mra-code-marketing-research-standards}}. The code is a $ 20 $ page document that includes $ 42 $ principles and is divided into three articles, an enforcement FAQ, and two appendices.

\begin{enumerate}
	\item Article I: Responsibility to Respondents and Prospective Respondents. 
	
	\begin{itemize}
		\item General Conduct. This article focuses on how to treat the respondents in a research project. It includes requirements like protect their right to drop out of a research project and their right to privacy.
		\item Purpose of Use. This article requires researchers to obtain a consent form and protect respondent information from improper use, like solicitations.
		\item Transparancy. This article requires researchers to be honest with respondents and make the research method transparant. It includes things like not collecting information without the respondent's knowledge and keeping an internal ``do not call'' list so respondents can opt out of future contacts.
		\item Technical Compliance. This article focuses on legal and other matters, like adhering to all state laws for projects that cross state borders and being especially careful with vulnerable populations, like children.
	\end{itemize}
	
	\item Article II: Responsibilities to Clients and Vendors. This article requires researchers to maintain a trusted relationship with clients and vendors and refrain from engaging in unacceptable practices with any research partner.
	
	\item Article III: Professional Responsibilities. Researchers are required to report research results accurately and honestly and not falsify or omit data.
	
\end{enumerate}

\section{An Ethical Controversy}

Target stores received considerable negative publicity for using data mining and market basket analysis to identify women who are pregnant\footnote{This incident was widely reported in the popular press, including Forbes \url{https://www.forbes.com/sites/kashmirhill/2012/02/16/how-target-figured-out-a-teen-girl-was-pregnant-before-her-father-did/\#25e2ed896668} and the New York Times \url{http://www.nytimes.com/2012/02/19/magazine/shopping-habits.html?_r=1&hp=&pagewanted=all}.}. 

Market Basket analysis is attempting to determine the types of products that are typically purchased together, they are in the same ``market basket.'' For example, beer and potato chips are often purchased together so if a store were running a special sale on beer it my also have a promotion for potato chips. Of course, formal market basket analysis is much more in-depth than that and can find odd relationships between numerous products. When market basket analysis gets personal the results can be ethically interesting.

For example, Target, Inc., like many large chains, have customer loyalty programs where customers can use a card or phone number to get special discounts on certain items. Of course, the entire shopping trip is then recorded in the company's database and market basket analysis can determine what this one specific customer is likely to purchase.

Target's problem began when a father in Minneapolis complained that his teen-aged daughter had received pregnancy-related coupons. He felt that the coupons were inappropriate and promoted teen pregnancy. He later found out that his daughter was, in fact, pregnant and apologized to the store manager. Target had been able to use market basket research to determine that the girl was purchasing the types of items that pregnant women purchase so they sent her targeted ads for things pregnant women need.

To build its predictive models, Target focused on women who had signed up for the baby registry. They then compared those women’s purchases with all customers. Twenty-five variables were found that could identify pregnant women and even when their babies were due to an amazing degree of accuracy. The variables included things like buying large quantities of unscented lotions, supplements like calcium, magnesium, and zinc, and washcloths. The analytics were good enough that Target found that pregnant women tend to buy more hand sanitizers and washcloths as they get close to their delivery date. Target used these predictions to identify which women should receive specific coupons. 

After that particular controversy settled down, Target used the same analytics to predict when people were getting married and they would send out invitations to join the bridal registry before the marrying couple had a chance to tell their parents.

In response to the negative press, Target no longer sends out ads for only one specific item, but they have become much more devious. If they know, for example, that some woman is pregnant then the circular going to that particular house will have ads for garden implements and coffee, but there will be a number of prominent ads for items that a pregnant woman would need. The house next door would get a different circular with prominent ads for, maybe, party items.

The market basket analysis being done by Target, and other stores, is perfectly legal; however, it does raise ethical questions concerning customer privacy and informed consent. 

\section{Summary}

Students who are embarking on a research project within a university setting should follow  the code of ethics from the \ac{IRB} of their institution. Researchers who are working independently should join an appropriate professional organization (depending on the type of research they are conducting) adopt the code of ethics from that organization.

