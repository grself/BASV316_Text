%********************************************************************
% Appendix
%*******************************************************
% If problems with the headers: get headings in appendix etc. right
%\markboth{\spacedlowsmallcaps{Appendix}}{\spacedlowsmallcaps{Appendix}}
\chapter{Appendix}

\section*{Glossary}

\begin{enumerate}
	
	\item \textbf{Boundary Conditions} (Chapter 2)
	
	\item \textbf{Constructivism} A philosophical stance that reality is a construct of the human mind and is, therefore, subjective. Normally, qualitative research methods are used by researchers who are constructivists. (Chapter 1)
	
	\item \textbf{Constructs} (Chapter 2)
	
	\item \textbf{Deductive Research} A research methodology that works from a general theory to specific observations. This is sometimes called the ``theory-testing'' form of research.
	
	\item \textbf{Epistomology} A branch of philosophy that is concerned with the sources of knowledge. (Chapter 1)
	
	\item \textbf{Falsifiability} To have credence, a hypothesis must be disapprovable; that is, there must be a way to prove it wrong.
	
	\item \textbf{Grounded Theory} (Chapter 2)
	
	\item \textbf{Idiographic} An explanation for an observed phenomenon that explains only a single case and is not applicable to a wider population. (Chapter 2)
	
	\item \textbf{Inductive Research} A research methodolody that works from specific observations to a general theory. This is sometimes called the ``theory-building'' form of research.
	
	\item \textbf{Logic} (Chapter 2)
	
	\item \textbf{Macro-Level Research} (Chapter 2)
	
	\item \textbf{Meso-Level Research} (Chapter 2)
	
	\item \textbf{Micro-Level Research} (Chapter 2)
	
	\item \textbf{Nomothetic} An explanation for an observed phenomenon that is applicable across a wide population rather than a single example. (Chapter 2)
	
	\item \textbf{Objectivism} A philosophical stance that there exists an objective reality can be studied and understood. (Chapter 1)
	
	\item \textbf{Ontology} The branch of philosophy that is concerned with the nature of reality. (Chapter 1)
	
	\item \textbf{Paradigm} A pattern or model of how things work in the world. (Chapter 1)
	
	\item \textbf{Parsimony} A fundamental aspect of research that states if two or more competing explanations are considered then the simplest must be accepted. Thus, researcher would state that the pyramids were built by humans using known technology rather than aliens in spaceships. (Chapter 1)
	
	\item \textbf{Positivism} (Chapter 2)
	
	\item \textbf{Postmodernism} (Chapter 2)
	
	\item \textbf{Precision} Research projects must precisely focus on one aspect of a problem or they will become so broad that their value will be diminished. (Chapter 1)
	
	\item \textbf{Propositions} (Chapter 2)
	
	\item \textbf{Replicability} A research project must be able to be replicated by other researchers or at other times in order to be considered sound. (Chapter 1)
	
\end{enumerate}


\cleardoublepage
\section*{Bibliography}

\subsection*{\ref{ch01:introduction}:  \nameref{ch01:introduction}}

\begin{description}

	\item Bobbitt-Zeher, Donna, and Douglas B. Downey. "Number of siblings and friendship nominations among adolescents." Journal of Family Issues 34.9 (2013): 1175-1193.
	
	\item Ellwood, David, and Thomas J. Kane. "Who is getting a college education? Family background and the growing gaps in enrollment." Securing the future: Investing in children from birth to college (2000): 283-324.

	\item Twain, Mark. Following the equator. Trajectory Inc, 2014.
	
\end{description}

\subsection*{\ref{ch02:foundations}: \nameref{ch02:foundations}}

\begin{description}

	\item Bacharach, Samuel B. "Organizational theories: Some criteria for evaluation." Academy of management review 14.4 (1989): 496-515
	
	\item Bansal, Pratima, and Kendall Roth. "Why companies go green: A model of ecological responsiveness." Academy of management journal 43.4 (2000): 717-736
	
	\item Delaney, John T., and Mark A. Huselid. "The impact of human resource management practices on perceptions of organizational performance." Academy of Management journal 39.4 (1996): 949-969
	
	\item Eisenhardt, Kathleen M., and Melissa E. Graebner. "Theory building from cases: Opportunities and challenges." Academy of management journal 50.1 (2007): 25-32
	
	\item Hackman, J. Richard, and Greg R. Oldham. "Motivation through the design of work: Test of a theory." Organizational behavior and human performance 16.2 (1976): 250-279
	
	\item Markus, M. Lynne. "Toward a “critical mass” theory of interactive media: Universal access, interdependence and diffusion." Communication research 14.5 (1987): 491-511
	
	\item Parboteeah, K. Praveen, Yongsun Paik, and John B. Cullen. "Religious groups and work values: A focus on Buddhism, Christianity, Hinduism, and Islam." International Journal of Cross Cultural Management 9.1 (2009): 51-67
	
	\item Sharma, Sanjay. "Managerial interpretations and organizational context as predictors of corporate choice of environmental strategy." Academy of Management journal 43.4 (2000): 681-697
	
	\item Sherman, Lawrence W., and Richard A. Berk. "The specific deterrent effects of arrest for domestic assault." American sociological review (1984): 261-272
	
	\item Sia, Surendra Kumar, and Gopa Bhardwaj. "Employees ‘perception of diversity climate: Role of psychological contract." Journal of Indian Academy of Applied Psychology 35 (2009): 305-312
	
	\item Steinfield, Charles W., and Janet Fulk. "The theory imperative." Organizations and communication technology (1990): 13-25

	\item Strauss, Anselm, and Juliet Corbin. "Grounded theory methodology." Handbook of qualitative research 17 (1994): 273-85

	\item Whetten, David A. "What constitutes a theoretical contribution?." Academy of management review 14.4 (1989): 490-495

\end{description}

\subsection*{\ref{ch03:research_ethics}: \nameref{ch03:research_ethics}}

\begin{description}

	\item Cummings, Ann D. "The Exxon Valdez Oil Spill and the Confidentiality of Natural Resource Damage Assessment Data." Ecology Law Quarterly 19.2 (1992): 363-412
	
	\item Faden, Ruth R., and Tom L. Beauchamp. A history and theory of informed consent. Oxford University Press, 1986

	\item Green, Ronald M. "Direct-to-consumer advertising and pharmaceutical ethics: The case of Vioxx." Hofstra L. Rev. 35 (2006): 749
	
	\item Riedel, Stefan. "Edward Jenner and the history of smallpox and vaccination." Proceedings (Baylor University. Medical Center) 18.1 (2005): 21

\end{description}

\subsection*{\ref{ch04:research_design}: \nameref{ch04:research_design}}

asdf










