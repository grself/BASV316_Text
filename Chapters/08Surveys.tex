%*****************************************
\chapter{Survey Research}\label{08:surveys}
%*****************************************

\begin{wrapfigure}{R}{0.4\textwidth}
	\label{08:fig01} 
	\centering
	\includegraphics[width=0.4\textwidth]{gfx/08-01} 
\end{wrapfigure}

How do retailers know what sorts of products their customers are likely to purchase? They use surveys to ask questions. Unfortunately, creating an unbiased survey that asks the right questions is a much more complex task than it may seem. This chapter introduces the art and science of survey design, from types of surveys, writing questions, and analyzing the results.
\blfootnote{Photo by Joshua Rawson-Harris on Unsplash}

\section{Introduction}

Survey research is a method involving the use of standardized questionnaires or interviews to collect data about people and their preferences, thoughts, and behaviors in a systematic manner. Although census surveys were conducted as early as Ancient Egypt, survey as a formal research method was pioneered in the 1930-40s by sociologist Paul Lazarsfeld to examine the effects of the radio on political opinion formation of the United States. This method has since become a very popular method for quantitative research in business and social sciences. Because most students have completed many surveys, they often underestimate the skill and effort needed to create a valid survey. The process is time-consuming and tedious and requires many revisions.

The survey method is best suited for studies that have individual people as the unit of analysis. Although other units of analysis, such as groups, organizations or dyads (pairs of organizations, such as buyers and sellers), are also studied using surveys, such studies often use a specific person from each unit as a ``key informant'' or a ``proxy'' for that unit, and such surveys may be subject to respondent bias if the informant chosen does not have adequate knowledge or has a biased opinion about the phenomenon of interest. For instance, Chief Executive Officers may not adequately know employee's perceptions or teamwork in their own companies, and may therefore be the wrong informant for studies of team dynamics or employee self-esteem.

Survey research has several inherent strengths compared to other research methods. 

\begin{enumerate}
	\item Surveys are an excellent vehicle for measuring a wide variety of unobservable data, such as people's preferences (e.g., political orientation), traits (e.g., self-esteem), attitudes (e.g., toward immigrants), beliefs (e.g., about a new law), behaviors (e.g., smoking or drinking behavior), or factual information (e.g., income). 
	\item Survey research is also ideally suited for remotely collecting data about a population that is too large to observe directly. A large area, such as an entire country, can be covered using mail-in, electronic mail, or telephone surveys using meticulous sampling to ensure that the population is adequately represented in a small sample. 
	\item Due to their unobtrusive nature and the ability to respond at one's convenience, questionnaire surveys are preferred by some respondents.
	\item Surveys are more easily generalized than other research techniques since data can be collected from very large samples at a relatively low cost.
	\item Because surveys are standardized in that the same questions, phrased in exactly the same way, are posed to all participants they are more reliable than other methods of gathering data.
	\item Interviews may be the only way of reaching certain population groups such as the homeless or illegal immigrants for which there is no sampling frame available. 
	\item Large sample surveys may allow detection of small effects even while analyzing multiple variables, and depending on the survey design, may also allow comparative analysis of population subgroups (i.e., within-group and between-group analysis). 
	\item Survey research is economical in terms of researcher time, effort and cost than most other methods such as experimental research and case research.
\end{enumerate}

At the same time, survey research also has some disadvantages. 

\begin{enumerate}
	\item It is subject to a large number of biases such as non-response bias, sampling bias, social desirability bias, and recall bias.
	\item While surveys are flexible in the sense that any number of questions on any number of topics may be asked, the researcher is also stuck with that instrument even if it is later shown to contain confusing items. 
	\item Survey questions must be written such that a broad range of people will understand each of them. Because of this, survey results may suffer from validity concerns not found in methods that are more flexible. 
\end{enumerate}
 
\section{Types of Surveys}

There is much variety when it comes to surveys. This variety comes both in terms of time, when or how frequently a survey is administered, and in terms of administration, how a survey is delivered to respondents. This section develops both types of concepts.

\subsection{Time}

In terms of time, there are two main types of surveys: cross-sectional and longitudinal.

\subsubsection{Cross-Sectional}

Cross-sectional surveys are those that are administered at just one point in time. These surveys offer researchers a snapshot in time and provides an idea about how things are at the particular point in time. These surveys are call ``cross-sectional'' since they will take a snapshot across multiple analytical units. For example, a survey may be administered to staff members in the human resources department of five different companies or customers of several different movie theaters on the same evening. 

An example of a cross-sectional survey is a study of e-cigarette use among adolescents conducted by Dutra and Glantz\cite{dutra2014electronic}. They used a cross-sectional survey of more than 40,000 students from more than 200 middle and high schools across the United States. They determined that the use of e-cigarettes was ``...associated with higher odds of ever or current cigarette smoking...''

Another example of a cross-sectional survey, J\o{}rgensen, et. al.\cite{jorgensen2016does}, investigated if workplace health promotions depend on the work environment. They surveyed 10,605 Danish workers and determined that lower participation in health promotions is dependent on when they are offered (during or afterwork), the social support at work for the programs, and whether their work has high physical demands.

One problem with cross-sectional surveys is that the events, opinions, behaviors, and other phenomena that such surveys are designed to assess are generally not stagnant. Thus, generalizing from a cross-sectional survey about the way things are can be tricky. Perhaps something can be concluded about the way things \textit{were} in the moment that the survey was administered, but it is difficult to know whether things remained that way afterwards. For example, imagine how Americans might have responded to a survey about terrorism on September 10, 2001, compared to September 12, 2001. The point is not that cross-sectional surveys are useless, but researchers must remember that these surveys are a snapshot in time.

\subsubsection{Longitudinal}

Longitudinal surveys are those that include observations made over some extended period of time. There are three types of longitudinal surveys: trend, panel, and cohort.

\paragraph{Trend Survey}

The first type of longitudinal survey is a trend survey. This type of study takes place over a long period of time, often years, and involves multiple surveys of many different people. As one example of a trend survey, Dobrow, Ganzach, and Liu\cite{dobrow2015time} studied job satisfaction in light of employee's age and tenure. They surveyed $ 21,670 $ people in $ 34 $ ``waves'' of data collection spanning $ 40 $ years. They found that as tenure in an organization increased people tend to be less satisfied with their jobs, however, as people age and move from job to job their satisfaction increased.

\paragraph{Panel Surveys}

Unlike a trend survey, a panel survey uses the same people each time it is administered. As you might imagine, panel studies can be difficult and costly. Imagine trying to administer a survey to the same $ 100 $ people every year for five years in a row. Keeping track of where people live, when they move, and when they die takes resources that researchers often do not have. Panel surveys, however, can be quite powerful. 

The \gls{yds}\cite{uminn2018youth}, administered by the University of Minnesota, is an excellent example of a panel study. Since 1988, \gls{yds} researchers have administered an annual survey to the same $ 1,000 $ people. Study participants were in ninth grade when it began and they are now in their thirties. Several hundred papers, articles, and books have been written using data from the YDS. One of the major lessons learned from this panel study is that work has a largely positive impact on young people. Contrary to popular belief about the impact of work on adolescents' performance in school and transition to adulthood, work increases confidence, enhances academic success, and prepares students for success in their future careers. This panel study provided important information about the affect of work on young people.

As an example of a panel survey for business, Huhtala, Kaptein, and Feldt conducted a two-year study concerning how the ethical culture of organizations influence the well-being of managers\cite{huhtala2016perceived}. They found that managers in low or decreasing ethical cultures experienced changes in their well-being over the two years of the study.

\paragraph{Cohort Surveys}

In a cohort survey, a researcher identifies some category of people that are of interest and then regularly surveys the people who fall into that category. The same people do not necessarily participate from year to year, but all participants must meet whatever categorical criteria fulfill the researcher's primary interest. Common cohorts that may be of interest to researchers include people of particular generations or those who were born around the same time period, graduating classes, people who began work in a given industry at the same time, or perhaps people who have some specific life experience in common. An example of this sort of research can be seen in Christine Percheski’s work on cohort differences in women's employment\cite{percheski2008opting}. Percheski compared women's employment rates across seven different generational cohorts, from Progressives born between 1906 and 1915 to Generation Xers born between 1966 and 1975. She found, among other patterns, that professional women's labor force participation had increased across all cohorts. She also found that professional women with young children from Generation X had higher labor force participation rates than similar women from previous generations, concluding that mothers do not appear to be opting out of the workforce as some journalists have speculated.

In another cohort study, Wright and Hinson surveyed public relations practitioners over a ten-year period from $ 2005 $ until $ 2015 $\cite{wright2015examining}. While some of the same practitioners would have participated over that entire span, it is reasonable to assume that at least a few of the respondents changed during that period, but they were all public relations experts in a company. They found that both Facebook and Twitter were the dominant means of public relations communication while LinkedIn and YouTube were not as popular. The survey respondents, though, agreed that social media were changing the way public relations is practiced. It would be interesting to see how the perceptions of public relations experts have changed since $ 2016 $ given the heavy reliance on Twitter in the United States by the Trump administration.

All three types of longitudinal surveys share the strength that they permit a researcher to make observations over time. This means that if whatever behavior or other phenomenon the researcher is interested in changes, either because of some world event or because people age, the researcher will be able to capture those changes. Table \ref{tab08.01} summarizes each of the three types of longitudinal survey.

\begin{table}[H]
	\centering
	\definecolor{ltgray}{gray}{0.95} % this is a light gray
	\rowcolors{1}{}{ltgray} % zebra striping background
	\begin{tabularx}{0.95\linewidth}{p{0.15\linewidth}p{0.75\linewidth}}
		\toprule
		\textbf{Type} & \textbf{Description} \\
		\midrule
		Trend & Researcher examines changes in trends over time; the same people do not necessarily participate in the survey more than once. \\
		Panel & Researcher surveys the exact same sample several times over a period of time. \\
		Cohort & Researcher identifies some category of people that are of interest and then regularly surveys people who fall into that category.\\
		\bottomrule
	\end{tabularx}
	\caption{Compare the Three Types of Longitudinal Survey}
	\label{tab08.01}
\end{table}

\paragraph{Retrospective Surveys}

%TODO Start Here

Retrospective surveys are similar to other longitudinal studies in that they concern changes over time, but like a cross-sectional study, they are administered only once. In a retrospective survey, participants are asked to report events from the past. By having respondents report past behaviors, beliefs, or experiences, researchers are able to gather longitudinal-like data without actually incurring the time or expense of a longitudinal survey. Of course, this benefit must be weighed against the possibility that people's recollections of their pasts may be faulty. Imagine, for example, that people are asked in a survey to respond to questions about where, how, and with whom they spent last Valentine's Day. Since Valentine's Day cannot be more than 12 months ago, chances are good that they may be able to respond accurately. But if the question is to compare last Valentine's Day with the six previous Valentine's Days the result would be much different.

\subsection{Administration}

Surveys vary not just in terms of when they are administered but also in terms of how they are administered. One common way to administer surveys is in the form of self-administered questionnaires. This means that research participants are given a set of questions, in writing, to which they are asked to respond. Self-administered questionnaires can be delivered in hard copy format, typically via mail, or increasingly more commonly, on-line. Both modes of delivery are considered here.

Hard copy self-administered questionnaires may be delivered to participants in person or via snail mail. Students are commonly given surveys in person on campus or in large classes. Researchers may also deliver surveys in person by going door-to-door and either asking people to fill them out right away or making arrangements for the researcher to return to pick up completed surveys. Though the advent of on-line survey tools has made door-to-door delivery of surveys nearly extinct, an occasional survey researcher may still use this method, especially around election time.

If a researcher is not able to visit each member of the sample to personally deliver a survey, sending it through the mail may be another consideration. While this mode of delivery may not be ideal (imagine how much less likely someone would be to return a survey where the researcher was not standing on the doorstep waiting), sometimes it is the only available or the most practical option.

Often survey researchers who deliver their surveys via snail mail may provide some advance notice to respondents about the survey to get people thinking about and preparing to complete it. They may also follow up with their sample a few weeks after their survey has been sent out. This can be done not only to remind those who have not yet completed the survey to please do so but also to thank those who have already returned the survey. Most survey researchers agree that this sort of follow-up is essential for improving mailed surveys’ return rates\footnote{Babbie, E., \& Wagenaar, T. (2010). Unobtrusive research. The practice of social research, 320.}.

Online delivery of surveys are another approach to the administration challenge. This delivery mechanism is becoming increasingly common, no doubt because it is easy to use, relatively cheap, and may be quicker than knocking on doors or waiting for mailed surveys to be returned. To deliver a survey online, researchers may subscribe to a service that offers online delivery or use some delivery mechanism that is available for free, like \textit{SurveyMonkey}\footnote{\url{http://www.surveymonkey.com}}. One advantage to using a service like \textit{SurveyMonkey}, aside from the advantages of online delivery, is that results can be provided in formats that are readable by data analysis programs such as \textit{R} and \textit{Excel}. This saves researchers the step of having to manually enter data into an analysis program, as is necessary for hard copy surveys.

Many of the suggestions provided for improving the response rate on a hard copy questionnaire apply to online questionnaires as well. One difference, of course, is that the sort of incentive that can be provided in an online format differ from those that can be given in person or sent through the mail. Many online surveys only come with the incentive of knowing that the respondent is helping other people. It is possible, though, to provide some sort of coupon to a local store or Amazon.com. Commonly, online surveys provide some sort of food, like a ``free drink,'' from the restaurant that is administering the survey. Finally, it is possible to have respondents provide some sort of contact information, like an email address, and then have a drawing for a free \textit{Fire} tablet or some other prize. Using these sorts of rewards raises questions about the validity of the results. If people are only participating in a survey to have a chance at a prize then are they going to simply pattern-respond (choose all ``A'' answers, for example) or will they take the time to thoughtfully respond?

Sometimes surveys are administered by researchers posing questions directly to respondents rather than them read the questions on their own. These types of surveys are a form of interviews, which is discussed elsewhere in this book. It is enough at this point to mention that interview methodology differs significantly from survey research in that data are collected via a personal interaction. 

Whatever mechanism is selected, there are both strengths and weaknesses which must be considered. While online surveys may be faster and cheaper than mailed surveys, it may be that not everyone in the sample has easy access to a computer and the internet. On the other hand, mailed surveys are more likely to reach the entire sample but also more likely to be ignored. The choice of which delivery mechanism is best depends on a number of factors including the researcher's resources, the respondent's resources, and the time available to distribute surveys and wait for responses.

\section{Designing Effective Questionnaires}

Invented by Sir Francis Galton, a questionnaire is a research instrument consisting of a set of items intended to capture responses from respondents in a standardized format. Items may be either structured or unstructured. Structured items ask respondents to select an answer from a set of choices. The responses are then aggregated into a composite scale or index for statistical analysis. On the other hand, unstructured questions ask respondents to provide a response in their own words using a free-flow type of entry. Questions should be designed such that respondents are able to read, understand, and respond to them in a meaningful way so surveys would not be appropriate for certain demographic groups such as children or the illiterate. 

Most questionnaire surveys tend to be self-administered mail surveys, where the same questionnaire is mailed to a large number of people and respondents complete and return the survey at their own convenience. Mail surveys are advantageous in that they are unobtrusive and inexpensive to administer. However, response rates from mail surveys tend to be quite low since most people ignore survey requests. There may also be long delays, perhaps several months, before receiving the responses. That means that researchers must continuously monitor responses and send reminders to non-respondents. Questionnaire surveys are also not well-suited for issues that require clarification or require detailed responses. Finally, a longitudinal research design can send a survey to the same group of respondents several times over a long period but response rates tend to fall precipitously from one period to the next.

A second type of survey is a group-administered questionnaire. A sample of respondents is brought together at a common place and time and each respondent is asked to complete the survey questionnaire while in that room. Respondents enter their responses independently without interacting with each other. This format is convenient for the researcher and high response rate is assured. Also, if respondents do not understand any specific question, they can ask for clarification. These types of surveys are most useful in an organization where it is relatively easy to assemble a group of employees in a conference room or lunch room, especially if the survey is approved by corporate executives.

A more recent type of questionnaire survey is an online or web survey. These surveys are administered over the Internet using interactive forms. Respondents may receive an electronic mail or text message request for participation in the survey with a link to a site where the survey may be completed. Alternatively, the survey may be embedded in an e-mail and can be completed and returned immediately. These surveys are very inexpensive to administer, results are instantly recorded in an online database, and the survey can be easily modified if needed. However, if the survey website is not password-protected or designed to prevent multiple submissions, the responses can be easily compromised. Furthermore, sampling bias may be a significant issue since the survey cannot reach people that do not have computer or Internet access, such as many of the poor, senior, and minority groups; moreover, the respondent sample will be skewed toward a younger demographic who are online much of the time and have the time and ability to complete such surveys. Computing the response rate may be problematic, if the survey link is posted in Facebook, Twitter, or other social media sites instead of being e-mailed directly to targeted respondents. 

\subsection{Effective Questions}

The first thing needed to write effective survey questions is identifying what, exactly, is being sought. Though it seems obvious, missing important questions when designing a survey is far too common. Suppose researchers want to understand how students make a successful transition from high school to college and what factors contribute to that success. To understand those factors, the researchers will need to include questions about all of the possible factors that could contribute to success. They woudl consult the literature on the topic, but should also take the time for brainstorming and talking with other researchers (and even high school students) about what may be important in the transition to college. It may not be possible to include every single factor on a survey since that may make the survey pages long, but some thought would generate a list of the most important factors.

While it is important to include questions on all topics that are important to the research question, an ``everything-but-the-kitchen-sink'' approach is counterproductive since it puts an unnecessary burden on the survey respondents. Remember that respondents have agreed to volunteer their time and attention so they deserve respect in only asking on-topic questions.

Once the question topics are identified the questions need to be drafted. Questions should be as clear and to the point as possible. This is not the time for researchers to show off their creative writing skills; a survey is a technical instrument and should be written in a way that is as direct and succinct as possible. As much as possible, every question on the survey should be relevant to every person who is asked to respond. This means two things: first, that respondents have knowledge about the topic and second, they have experience with the events, behaviors, or feelings being probed. For example, a sample of 18-year-old respondents should not be asked how they would have advised President Clinton concerning his impeachment. For one thing, few 18-year-olds are likely to have any clue about how to advise a president; moreover, today's 18-year-olds were not even alive during Clinton's impeachment, so they would have had no experience with the event. In the example of successful college transition, respondents must understand the phrase ``transition to college'' and have actually experienced that transition themselves.

If a survey includes items that only a portion of respondents will have had experience it should include a ``filter question.'' A filter question is designed to identify some subset of survey respondents who are asked additional questions that are not relevant to the entire sample. Using the successful college transition survey mentioned above, if alcohol abuse is determined to be relevant to college success it would not be appropriate to ask ``How often did you drink alcohol during your first semester in college?'' That presupposes alcohol use and many students may abstain altogether. It would be better to include a statement like ``If you drank alcohol during your first semester of college please answer questions 13 and 14, otherwise skip to question 15.''

Responses obtained in survey research are very sensitive to the types of questions asked. Poorly framed or ambiguous questions will likely result in meaningless responses with very little value. Dillman\cite{dillman2011mail} recommends several rules for creating good survey questions.

\begin{itemize}
	\item Is the question clear and understandable: Survey questions should be stated in a very simple language, preferably in active voice, and without complicated words or jargon that may not be understood by a typical respondent. All questions in the questionnaire 	should be worded in a similar manner to make it easy for respondents to read and 	understand them. The only exception is if your survey is targeted at a specialized group of respondents, such as doctors, lawyers and researchers, who use such jargon in their everyday environment.
	\item Is the question worded in a negative manner: Negatively worded questions, such as ``should your local government not raise taxes,'' tend to confuse respondents and lead to inaccurate responses. Such questions should be avoided. More importantly, in all cases double-negatives must be avoided.
	\item Is the question ambiguous: Survey questions should not include words or expressions that may be interpreted differently by different respondents (e.g., words like ``any'' or ``justice''). For instance, if survey includes a question like, ``what is your annual income,'' it is unclear whether it is referring to only wages or also dividend, rental, and other income. Different interpretation by different respondents will lead to incomparable responses that cannot be interpreted correctly.
	\item Does the question have biased or value-laden words: Bias refers to any property of a question that encourages subjects to answer in a certain way. Kenneth Rasinky (1989) examined several studies on people's attitude toward government spending, and observed that respondents tend to indicate stronger support for ``assistance to the poor'' and less for ``welfare,'' even though both terms had the same meaning. In this study, 	more support was also observed for ``halting rising crime rate'' (and less for ``law enforcement''), ``solving problems of big cities'' (and less for ``assistance to big cities''), 	and ``dealing with drug addiction'' (and less for ``drug rehabilitation''). A biased language or tone tends to skew observed responses. It is often difficult to anticipate in advance the biasing wording, but to the greatest extent possible, survey questions should be carefully scrutinized to avoid biased language.
	%TODO I could not find the Rasinky study that Anol referenced above
	\item Is the question double-barreled: Double-barreled questions are those that can have multiple answers. For example, a question like ``Are you satisfied with the hardware and software provided for your work?'' may confuse respondents who may be satisfied with the hardware but not the software. It is always best to separate double-barreled questions into separate questions.
	\item Is the question too general: Sometimes, questions that are too general may not accurately convey respondents' perceptions. If a survey question asked ``How big is your firm,'' which could be interpreted differently by respondents, ask more specific questions like ``how many people does the firm employ,'' or ``what is the firm's annual revenue.''
	\item Is the question too detailed: Avoid unnecessarily detailed questions that serve no specific research purpose. For instance, does the research project require the ages for each child in a household or is just the number of children enough? Conversely, it is usually better to gather too many details than not enough.
	\item Is the question presumptuous: If a survey asks ``what are the benefits of a tax cut,'' there is a presumption that the respondent sees the tax cut as beneficial. Many people, though, may not view tax cuts as being beneficial since that generally leads to decreased funding for public services. Questions with built-in presumptions should be avoided on a survey.
	\item Is the question imaginary: A popular question in many television game shows is ``if you win a million dollars on this show, how will you spend it?'' Most respondents have never been faced with such an amount of money and have never thought about it so their answers tend to be random and trite, such as take a tour around the world. Imaginary questions have imaginary answers, and those cannot be used for valid inferences.
	\item Do respondents have the information needed to correctly answer the question: Often times, the assumption is that subjects have the necessary information to answer a question when, in reality, they do not. Even if a response is obtained the responses tend to be inaccurate. For instance, the CEO of a company should not be asked about the day-to-day operational details of their company since they do not work at that level.
	\item Is there a socially desirable response: respondents usually try to answer questions in a way that will match social norms. For example, if a group of students were asked if they cheat on exams they would likely not admit to that behavior since cheating is not socially acceptable. 
\end{itemize}

\subsection{Response Options}

While posing clear and understandable questions is important, so, too, is providing respondents with unambiguous response options. This assumes that the questions are closed-ended, that is, respondents are only permitted to select from a group of options. This puts a burden on the researcher to provide respondents an effective set of response options. Researchers should keep the following in mind when writing responses.

\begin{itemize}
	\item Response options should be mutually exclusive. If a survey asks the respondents to indicate an age group and the selections are ``less than 20,'' ``20-30,'' ``30-40,'' ``40-50,'' ``above 50'' then which should a 30-year-old person select since that age is in two groups? This is another one of those points about question construction that seems fairly obvious but is easily overlooked. 
	\item Response options should be exhaustive. Every possible response should be covered in the set of response options that you provide. For example, if a survey asks the respondents to indicate sex then ``male'' and ``female'' would not be enough since there are other potential responses. At the very least, surveys should include options like ``do not care to respond'' or ``none are correct'' for questions that may include controversial information.
\end{itemize}

Of course, surveys need not be limited to closed-ended questions. Researchers can include open-ended questions, which do not include response options, as a way to gather additional details. An open-ended question asks respondents to reply to the question in their own way, using their own words. These questions are generally used to find out more about a survey participant's experiences or feelings about whatever they are being asked to report in the survey. If, for example, a survey includes closed-ended questions asking respondents to report on their involvement in extracurricular activities during college, an open-ended question could ask respondents why they participated in those activities or what they gained from their participation. Allowing respondents to reply in their own words can make the experience of completing the survey more satisfying and often reveals new information that had not occurred to the researcher.

Two opposite respondent behaviors that should be considered by researchers are fence-sitting and floating. Fence-sitting is when a respondent tends to select a ``no opinion'' option rather than take a stance while floating is when a respondent tends to select an opinion when, in fact, they may have none. These behaviors are especially evident when Likert questions are asked. Consider the two possible response selections for a poll question asking about a fictitious ``Proposition 100.''

Do you agree with this statement: If Proposition 100 is passed my taxes will increase?

\begin{enumerate}
	\item Strongly disagree -- disagree -- neither agree nor disagree -- agree -- strongly agree
	\item Strongly disagree -- disagree -- agree -- strongly agree
\end{enumerate}

The first set of responses permit a respondent to ``fence-sit'' and select a neutral opinion while the second set force the respondent to indicate some level of agreement. Either of these response sets could be viable depending on the goal of the research project. For questions that probe socially undesirable behavior (like cheating on exams) it may be appropriate to give respondents the option to remain neutral while in other cases the researcher may want to force respondents to take a stance.

A matrix, which lists a set of questions that use the same response categories, creates a compact presentation that is easy to understand and encourages participation. Following is an example matrix for an imaginary set of election propositions.



\vspace{.15in}

\begin{tabulary}{\linewidth}{LCCCC}
	\hline
	\multicolumn{5}{l}{\textbf{Do you support these propositions?}} \\
	\hline
	Prop & Strongly Support & Support & Do Not Support & Strongly Do Not Support  \\ 
	\hline
	100 & $\bigcirc$ & $\bigcirc$ & $\bigcirc$ & $\bigcirc$ \\ 
	115 & $\bigcirc$ & $\bigcirc$ & $\bigcirc$ & $\bigcirc$ \\ 
	220 & $\bigcirc$ & $\bigcirc$ & $\bigcirc$ & $\bigcirc$ \\ 
	\hline
\end{tabulary} 

\vspace{.15in}

Responses to closed-ended questions are captured using one of the following response formats.

\begin{itemize}
\item Dichotomous response, where respondents are asked to select one of two possible choices, such as true/false, yes/no, or agree/disagree. An example of such a question is: Do you think that the death penalty is justified under some circumstances: yes / no.
\item Nominal response, where respondents are presented with more than two unordered options, such as: What is your industry of employment: manufacturing / consumer services / retail / education / health care / tourism \& hospitality / other.
\item Ordinal response, where respondents have more than two ordered options, such as: what is your highest level of education: high school / college degree / graduate studies.
\item Interval-level response, where respondents are presented with a 5-point or 7-point Likert scale, semantic differential scale, or Guttman scale. 
\item Continuous response, where respondents enter a continuous (ratio-scaled) value with a meaningful zero point, such as their age or tenure in a firm. These responses generally tend to be fill-in-the blanks.
\end{itemize}
%TODO Start Here
\subsection{Designing Questionnaires}

In addition to constructing quality questions and posing clear response options, you’ll also need to think about how to present your written questions and response options to survey respondents. Questions are presented on a questionnaire, the document (either hard copy or online) that contains all your survey questions that respondents read and mark their responses on. Designing questionnaires takes some thought, and in this section we’ll discuss the sorts of things you should think about as you prepare to present your well-constructed survey questions on a questionnaire.

One of the first things to do once you’ve come up with a set of survey questions you feel confident about is to group those questions thematically. In our example of the transition to college, perhaps we’d have a few questions asking about study habits, others focused on friendships, and still others on exercise and eating habits. Those may be the themes around which we organize our questions. Or perhaps it would make more sense to present any questions we had about precollege life and habits and then present a series of questions about life after beginning college. The point here is to be deliberate about how you present your questions to respondents.

Once you have grouped similar questions together, you’ll need to think about the order in which to present those question groups. Most survey researchers agree that it is best to begin a survey with questions that will want to make respondents continue (Babbie, 2010; Dillman, 2000; Neuman, 2003). [3] In other words, don’t bore respondents, but don’t scare them away either. There’s some disagreement over where on a survey to place demographic questions such as those about a person’s age, gender, and race. On the one hand, placing them at the beginning of the questionnaire may lead respondents to think the survey is boring, unimportant, and not something they want to bother completing. On the other hand, if your survey deals with some very sensitive or difficult topic, such as child sexual abuse or other criminal activity, you don’t want to scare respondents away or shock them by beginning with your most intrusive questions.

In truth, the order in which you present questions on a survey is best determined by the unique characteristics of your research—only you, the researcher, hopefully in consultation with people who are willing to provide you with feedback, can determine how best to order your questions. To do so, think about the unique characteristics of your topic, your questions, and most importantly, your sample. Keeping in mind the characteristics and needs of the people you will ask to complete your survey should help guide you as you determine the most appropriate order in which to present your questions.

You’ll also need to consider the time it will take respondents to complete your questionnaire. Surveys vary in length, from just a page or two to a dozen or more pages, which means they also vary in the time it takes to complete them. How long to make your survey depends on several factors. First, what is it that you wish to know? Wanting to understand how grades vary by gender and year in school certainly requires fewer questions than wanting to know how people’s experiences in college are shaped by demographic characteristics, college attended, housing situation, family background, college major, friendship networks, and extracurricular activities. Keep in mind that even if your research question requires a good number of questions be included in your questionnaire, do your best to keep the questionnaire as brief as possible. Any hint that you’ve thrown in a bunch of useless questions just for the sake of throwing them in will turn off respondents and may make them not want to complete your survey.

Second, and perhaps more important, how long are respondents likely to be willing to spend completing your questionnaire? If you are studying college students, asking them to use their precious fun time away from studying to complete your survey may mean they won’t want to spend more than a few minutes on it. But if you have the endorsement of a professor who is willing to allow you to administer your survey in class, students may be willing to give you a little more time (though perhaps the professor will not). The time that survey researchers ask respondents to spend on questionnaires varies greatly. Some advise that surveys should not take longer than about 15 minutes to complete (cited in Babbie 2010), [4] others suggest that up to 20 minutes is acceptable (Hopper, 2010). [5] As with question order, there is no clear-cut, always-correct answer about questionnaire length. The unique characteristics of your study and your sample should be considered in order to determine how long to make your questionnaire.

A good way to estimate the time it will take respondents to complete your questionnaire is through pretesting. Pretesting allows you to get feedback on your questionnaire so you can improve it before you actually administer it. Pretesting can be quite expensive and time consuming if you wish to test your questionnaire on a large sample of people who very much resemble the sample to whom you will eventually administer the finalized version of your questionnaire. But you can learn a lot and make great improvements to your questionnaire simply by pretesting with a small number of people to whom you have easy access (perhaps you have a few friends who owe you a favor). By pretesting your questionnaire you can find out how understandable your questions are, get feedback on question wording and order, find out whether any of your questions are exceptionally boring or offensive, and learn whether there are places where you should have included filter questions, to name just a few of the benefits of pretesting. You can also time pretesters as they take your survey. Ask them to complete the survey as though they were actually members of your sample. This will give you a good idea about what sort of time estimate to provide respondents when it comes time to actually administer your survey, and about whether you have some wiggle room to add additional items or need to cut a few items.

Perhaps this goes without saying, but your questionnaire should also be attractive. A messy presentation style can confuse respondents or, at the very least, annoy them. Be brief, to the point, and as clear as possible. Avoid cramming too much into a single page, make your font size readable (at least 12 point), leave a reasonable amount of space between items, and make sure all instructions are exceptionally clear. Think about books, documents, articles, or web pages that you have read yourself—which were relatively easy to read and easy on the eyes and why? Try to mimic those features in the presentation of your survey questions.

\subsection{Anol: Question Sequencing}

Question sequencing. In general, questions should flow logically from one to the next. To achieve the best response rates, questions should flow from the least sensitive to the most sensitive, from the factual and behavioral to the attitudinal, and from the more general to the more specific. Some general rules for question sequencing:

\begin{itemize}
	\item Start with easy non-threatening questions that can be easily recalled. Good options are 	demographics (age, gender, education level) for individual-level surveys and firmographics (employee count, annual revenues, industry) for firm-level surveys.
	\item Never start with an open ended question.
	\item If following an historical sequence of events, follow a chronological order from earliest to latest.
	\item Ask about one topic at a time. When switching topics, use a transition, such as “The next section examines your opinions about …”
	\item Use filter or contingency questions as needed, such as: “If you answered “yes” to question 5, please proceed to Section 2. If you answered “no” go to Section 3.”
\end{itemize}

\subsection{Anol: Other Golden Rules}

Other golden rules. Do unto your respondents what you would have them do unto you. Be attentive and appreciative of respondents’ time, attention, trust, and confidentiality of personal information. Always practice the following strategies for all survey research:

\begin{itemize}
	\item People’s time is valuable. Be respectful of their time. Keep your survey as short as possible and limit it to what is absolutely necessary. Respondents do not like spending more than 10-15 minutes on any survey, no matter how important it is. Longer surveys tend to dramatically lower response rates.
	\item Always assure respondents about the confidentiality of their responses, and how you will use their data (e.g., for academic research) and how the results will be reported (usually, in the aggregate).
	\item For organizational surveys, assure respondents that you will send them a copy of the final results, and make sure that you follow up with your promise.
	\item Thank your respondents for their participation in your study.
	\item Finally, always pretest your questionnaire, at least using a convenience sample, before administering it to respondents in a field setting. Such pretesting may uncover ambiguity, lack of clarity, or biases in question wording, which should be eliminated before administering to the intended sample.
\end{itemize}

\paragraph{Key Takeaways}

\begin{itemize}
	\setlength{\itemsep}{0pt}
	\setlength{\parskip}{0pt}
	\setlength{\parsep}{0pt}
	
	\item Brainstorming and consulting the literature are two important early steps to take when preparing to write effective survey questions.
	\item Make sure that your survey questions will be relevant to all respondents and that you use filter questions when necessary.
	\item Getting feedback on your survey questions is a crucial step in the process of designing a survey.
	\item When it comes to creating response options, the solution to the problem of fence-sitting might cause floating, whereas the solution to the problem of floating might cause fence sitting.
	\item Pretesting is an important step for improving one’s survey before actually administering it.
	
\end{itemize}

\section{Analysis of Survey Data}

\begin{center}
	\begin{objbox}{Objectives}
		\begin{itemize}
			\setlength{\itemsep}{0pt}
			\setlength{\parskip}{0pt}
			\setlength{\parsep}{0pt}
			
			\item Define response rate, and discuss some of the current thinking about response rates.
			\item Describe what a codebook is and what purpose it serves.
			\item Define univariate, bivariate, and multivariate analysis.
			\item Describe each of the measures of central tendency.
			\item Describe what a contingency table displays.
			
		\end{itemize}
	\end{objbox}
\end{center}

This text is primarily focused on designing research, collecting data, and becoming a knowledgeable and responsible consumer of research. We won’t spend as much time on data analysis, or what to do with our data once we’ve designed a study and collected it, but I will spend some time in each of our data-collection chapters describing some important basics of data analysis that are unique to each method. Entire textbooks could be (and have been) written entirely on data analysis. In fact, if you’ve ever taken a statistics class, you already know much about how to analyze quantitative survey data. Here we’ll go over a few basics that can get you started as you begin to think about turning all those completed questionnaires into findings that you can share.

\subsection{From Completed Questionnaires to Data}

It can be very exciting to receive those first few completed surveys back from respondents. Hopefully you’ll even get more than a few back, and once you have a handful of completed questionnaires, your feelings may go from initial euphoria to dread. Data are fun and can also be overwhelming. The goal with data analysis is to be able to condense large amounts of information into usable and understandable chunks. Here we’ll describe just how that process works for survey researchers.

As mentioned, the hope is that you will receive a good portion of the questionnaires you distributed back in a completed and readable format. The number of completed questionnaires you receive divided by the number of questionnaires you distributed is your response rate. Let’s say your sample included 100 people and you sent questionnaires to each of those people. It would be wonderful if all 100 returned completed questionnaires, but the chances of that happening are about zero. If you’re lucky, perhaps 75 or so will return completed questionnaires. In this case, your response rate would be 75\% (75 divided by 100). That’s pretty darn good. Though response rates vary, and researchers don’t always agree about what makes a good response rate, having three-quarters of your surveys returned would be considered good, even excellent, by most survey researchers. There has been lots of research done on how to improve a survey’s response rate. We covered some of these previously, but suggestions include personalizing questionnaires by, for example, addressing them to specific respondents rather than to some generic recipient such as “madam” or “sir”; enhancing the questionnaire’s credibility by providing details about the study, contact information for the researcher, and perhaps partnering with agencies likely to be respected by respondents such as universities, hospitals, or other relevant organizations; sending out prequestionnaire notices and postquestionnaire reminders; and including some token of appreciation with mailed questionnaires even if small, such as a \$1 bill.

The major concern with response rates is that a low rate of response may introduce nonresponse bias into a study’s findings. What if only those who have strong opinions about your study topic return their questionnaires? If that is the case, we may well find that our findings don’t at all represent how things really are or, at the very least, we are limited in the claims we can make about patterns found in our data. While high return rates are certainly ideal, a recent body of research shows that concern over response rates may be overblown (Langer, 2003). [1] Several studies have shown that low response rates did not make much difference in findings or in sample representativeness (Curtin, Presser, \& Singer, 2000; Keeter, Kennedy, Dimock, Best, \& Craighill, 2006; Merkle \& Edelman, 2002). [2] For now, the jury may still be out on what makes an ideal response rate and on whether, or to what extent, researchers should be concerned about response rates. Nevertheless, certainly no harm can come from aiming for as high a response rate as possible.

Whatever your survey’s response rate, the major concern of survey researchers once they have their nice, big stack of completed questionnaires is condensing their data into manageable, and analyzable, bits. One major advantage of quantitative methods such as survey research, as you may recall from Chapter 1 "Introduction", is that they enable researchers to describe large amounts of data because they can be represented by and condensed into numbers. In order to condense your completed surveys into analyzable numbers, you’ll first need to create a codebook. A codebook is a document that outlines how a survey researcher has translated her or his data from words into numbers. An excerpt from the codebook I developed from my survey of older workers can be seen in Table 8.2 "Codebook Excerpt From Survey of Older Workers". The coded responses you see can be seen in their original survey format in Chapter 6 "Defining and Measuring Concepts", Figure 6.12 "Example of an Index Measuring Financial Security". As you’ll see in the table, in addition to converting response options into numerical values, a short variable name is given to each question. This shortened name comes in handy when entering data into a computer program for analysis.

If you’ve administered your questionnaire the old fashioned way, via snail mail, the next task after creating your codebook is data entry. If you’ve utilized an online tool such as SurveyMonkey to administer your survey, here’s some good news—most online survey tools come with the capability of importing survey results directly into a data analysis program. Trust me—this is indeed most excellent news. (If you don’t believe me, I highly recommend administering hard copies of your questionnaire next time around. You’ll surely then appreciate the wonders of online survey administration.)

For those who will be conducting manual data entry, there probably isn’t much I can say about this task that will make you want to perform it other than pointing out the reward of having a database of your very own analyzable data. We won’t get into too many of the details of data entry, but I will mention a few programs that survey researchers may use to analyze data once it has been entered. The first is SPSS, or the Statistical Package for the Social Sciences (http://www.spss.com). SPSS is a statistical analysis computer program designed to analyze just the sort of data quantitative survey researchers collect. It can perform everything from very basic descriptive statistical analysis to more complex inferential statistical analysis. SPSS is touted by many for being highly accessible and relatively easy to navigate (with practice). Other programs that are known for their accessibility include MicroCase (http://www.microcase.com/index.html), which includes many of the same features as SPSS, and Excel (http://office.microsoft.com/en-us/excel-help/about-statistical-analysis-tools-HP005203873.aspx), which is far less sophisticated in its statistical capabilities but is relatively easy to use and suits some researchers’ purposes just fine. Check out the web pages for each, which I’ve provided links to in the chapter’s endnotes, for more information about what each package can do.

\subsection{Identifying Patterns}

Data analysis is about identifying, describing, and explaining patterns.Univariate analysis is the most basic form of analysis that quantitative researchers conduct. In this form, researchers describe patterns across just one variable. Univariate analysis includes frequency distributions and measures of central tendency. A frequency distribution is a way of summarizing the distribution of responses on a single survey question. Let’s look at the frequency distribution for just one variable from my older worker survey. We’ll analyze the item mentioned first in the codebook excerpt given earlier, on respondents’ self-reported financial security.

As you can see in the frequency distribution on self-reported financial security, more respondents reported feeling “moderately secure” than any other response category. We also learn from this single frequency distribution that fewer than 10\% of respondents reported being in one of the two most secure categories. Another form of univariate analysis that survey researchers can conduct on single variables is measures of central tendency. Measures of central tendency tell us what the most common, or average, response is on a question. Measures of central tendency can be taken for any level variable of those we learned about in Chapter 6 "Defining and Measuring Concepts", from nominal to ratio. There are three kinds of measures of central tendency: modes, medians, and means. Mode refers to the most common response given to a question. Modes are most appropriate for nominal-level variables. A median is the middle point in a distribution of responses. Median is the appropriate measure of central tendency for ordinal-level variables. Finally, the measure of central tendency used for interval- and ratio-level variables is the mean. To obtain a mean, one must add the value of all responses on a given variable and then divide that number of the total number of responses.

In the previous example of older workers’ self-reported levels of financial security, the appropriate measure of central tendency would be the median, as this is an ordinal-level variable. If we were to list all responses to the financial security question in order and then choose the middle point in that list, we’d have our median. In Figure 8.12 "Distribution of Responses and Median Value on Workers’ Financial Security", the value of each response to the financial security question is noted, and the middle point within that range of responses is highlighted. To find the middle point, we simply divide the number of valid cases by two. The number of valid cases, 180, divided by 2 is 90, so we’re looking for the 90th value on our distribution to discover the median. As you’ll see inFigure 8.12 "Distribution of Responses and Median Value on Workers’ Financial Security", that value is 3, thus the median on our financial security question is 3, or “moderately secure.”

Figure 8.12 Distribution of Responses and Median Value on Workers’ Financial Security

As you can see, we can learn a lot about our respondents simply by conducting univariate analysis of measures on our survey. We can learn even more, of course, when we begin to examine relationships among variables. Either we can analyze the relationships between two variables, called bivariate analysis, or we can examine relationships among more than two variables. This latter type of analysis is known as multivariate analysis.

Bivariate analysis allows us to assess covariation among two variables. This means we can find out whether changes in one variable occur together with changes in another. If two variables do not covary, they are said to have independence. This means simply that there is no relationship between the two variables in question. To learn whether a relationship exists between two variables, a researcher may cross-tabulate the two variables and present their relationship in a contingency table. A contingency table shows how variation on one variable may be contingent on variation on the other. Let’s take a look at a contingency table. In Table 8.4 "Financial Security Among Men and Women Workers Age 62 and Up", I have cross-tabulated two questions from my older worker survey: respondents’ reported gender and their self-rated financial security.

You’ll see in Table 8.4 "Financial Security Among Men and Women Workers Age 62 and Up" that I collapsed a couple of the financial security response categories (recall that there were five categories presented in Table 8.3 "Frequency Distribution of Older Workers’ Financial Security"; here there are just three). Researchers sometimes collapse response categories on items such as this in order to make it easier to read results in a table. You’ll also see that I placed the variable “gender” in the table’s columns and “financial security” in its rows. Typically, values that are contingent on other values are placed in rows (a.k.a. dependent variables), while independent variables are placed in columns. This makes comparing across categories of our independent variable pretty simple. Reading across the top row of our table, we can see that around 44\% of men in the sample reported that they are not financially secure while almost 52\% of women reported the same. In other words, more women than men reported that they are not financially secure. You’ll also see in the table that I reported the total number of respondents for each category of the independent variable in the table’s bottom row. This is also standard practice in a bivariate table, as is including a table heading describing what is presented in the table.

Researchers interested in simultaneously analyzing relationships among more than two variables conduct multivariate analysis. If I hypothesized that financial security declines for women as they age but increases for men as they age, I might consider adding age to the preceding analysis. To do so would require multivariate, rather than bivariate, analysis. We won’t go into detail here about how to conduct multivariate analysis of quantitative survey items here, but we will return to multivariate analysis in Chapter 14 "Reading and Understanding Social Research", where we’ll discuss strategies for reading and understanding tables that present multivariate statistics. If you are interested in learning more about the analysis of quantitative survey data, I recommend checking out your campus’s offerings in statistics classes. The quantitative data analysis skills you will gain in a statistics class could serve you quite well should you find yourself seeking employment one day.


\subsection{Anol: Biases in Survey Research}

Despite all of its strengths and advantages, survey research is often tainted with systematic biases that may invalidate some of the inferences derived from such surveys. Five such biases are the non-response bias, sampling bias, social desirability bias, recall bias, and common method bias.

Non-response bias. Survey research is generally notorious for its low response rates. A response rate of 15-20\% is typical in a mail survey, even after two or three reminders. If the majority of the targeted respondents fail to respond to a survey, then a legitimate concern is whether non-respondents are not responding due to a systematic reason, which may raise questions about the validity of the study’s results. For instance, dissatisfied customers tend to be more vocal about their experience than satisfied customers, and are therefore more likely to respond to questionnaire surveys or interview requests than satisfied customers. Hence, any respondent sample is likely to have a higher proportion of dissatisfied customers than the underlying population from which it is drawn. In this instance, not only will the results lack generalizability, but the observed outcomes may also be an artifact of the biased sample. Several strategies may be employed to improve response rates:

\begin{itemize}
	\item Advance notification: A short letter sent in advance to the targeted respondents soliciting their participation in an upcoming survey can prepare them in advance and improve their propensity to respond. The letter should state the purpose and importance of the study, mode of data collection (e.g., via a phone call, a survey form in the mail, etc.), and appreciation for their cooperation. A variation of this technique may request the respondent to return a postage-paid postcard indicating whether or not they are willing to participate in the study.
	\item Relevance of content: If a survey examines issues of relevance or importance to respondents, then they are more likely to respond than to surveys that don’t matter to them.
	\item Respondent-friendly questionnaire: Shorter survey questionnaires tend to elicit higher response rates than longer questionnaires. Furthermore, questions that are clear, nonoffensive, and easy to respond tend to attract higher response rates.
	\item Endorsement: For organizational surveys, it helps to gain endorsement from a senior executive attesting to the importance of the study to the organization. Such endorsement can be in the form of a cover letter or a letter of introduction, which can improve the researcher’s credibility in the eyes of the respondents.
	\item Follow-up requests: Multiple follow-up requests may coax some non-respondents to respond, even if their responses are late.
	\item Interviewer training: Response rates for interviews can be improved with skilled interviewers trained on how to request interviews, use computerized dialing techniques to identify potential respondents, and schedule callbacks for respondents who could not be reached.
	\item Incentives: Response rates, at least with certain populations, may increase with the use of incentives in the form of cash or gift cards, giveaways such as pens or stress balls, entry into a lottery, draw or contest, discount coupons, promise of contribution to charity, and so forth.
	\item Non-monetary incentives: Businesses, in particular, are more prone to respond to nonmonetary incentives than financial incentives. An example of such a non-monetary incentive is a benchmarking report comparing the business’s individual response against the aggregate of all responses to a survey.
	\item Confidentiality and privacy: Finally, assurances that respondents’ private data or responses will not fall into the hands of any third party, may help improve response rates.
\end{itemize}

Sampling bias. Telephone surveys conducted by calling a random sample of publicly available telephone numbers will systematically exclude people with unlisted telephone numbers, mobile phone numbers, and people who are unable to answer the phone (for instance, they are at work) when the survey is being conducted, and will include a disproportionate number of respondents who have land-line telephone service with listed phone numbers and people who stay home during much of the day, such as the unemployed, the disabled, and the elderly. Likewise, online surveys tend to include a disproportionate number of students and younger people who are constantly on the Internet, and systematically exclude people with limited or no access to computers or the Internet, such as the poor and the elderly. Similarly, questionnaire surveys tend to exclude children and the illiterate, who are unable to read, understand, or meaningfully respond to the questionnaire. A different kind of sampling bias relate to sampling the wrong population, such as asking teachers (or parents) about academic learning of their students (or children), or asking CEOs about operational details in their company. Such biases make the respondent sample unrepresentative of the intended population and hurt generalizability claims about inferences drawn from the biased sample.

Social desirability bias. Many respondents tend to avoid negative opinions or embarrassing comments about themselves, their employers, family, or friends. With negative questions such as do you think that your project team is dysfunctional, is there a lot of office politics in your workplace, or have you ever illegally downloaded music files from the Internet, the researcher may not get truthful responses. This tendency among respondents to “spin the truth” in order to portray themselves in a socially desirable manner is called the “socialdesirability bias”, which hurts the validity of response obtained from survey research. There is practically no way of overcoming the social desirability bias in a questionnaire survey, but in an interview setting, an astute interviewer may be able to spot inconsistent answers and ask probing questions or use personal observations to supplement respondents’ comments.

Recall bias. Responses to survey questions often depend on subjects’ motivation, memory, and ability to respond. Particularly when dealing with events that happened in the distant past, respondents may not adequately remember their own motivations or behaviors or perhaps their memory of such events may have evolved with time and no longer retrievable. For instance, if a respondent to asked to describe his/her utilization of computer technology one year ago or even memorable childhood events like birthdays, their response may not be accurate due to difficulties with recall. One possible way of overcoming the recall bias is by anchoring respondent’s memory in specific events as they happened, rather than asking them to recall their perceptions and motivations from memory.

Common method bias. Common method bias refers to the amount of spurious covariance shared between independent and dependent variables that are measured at the same point in time, such as in a cross-sectional survey, using the same instrument, such as a questionnaire. In such cases, the phenomenon under investigation may not be adequately separated from measurement artifacts. Standard statistical tests are available to test for common method bias, such as Harmon’s single-factor test (Podsakoff et al. 2003), Lindell and Whitney’s (2001) market variable technique, and so forth. This bias can be potentially avoided if the independent and dependent variables are measured at different points in time, using a longitudinal survey design, of if these variables are measured using different methods, such as computerized recording of dependent variable versus questionnaire-based self-rating of independent variables.

\section{Summary}\label{08:summary}

\begin{itemize}
	\setlength{\itemsep}{0pt}
	\setlength{\parskip}{0pt}
	\setlength{\parsep}{0pt}
	
	\item Sometimes researchers may make claims about populations other than those from whom their samples were drawn; other times they may make claims about a population based on a sample that is not representative. As consumers of research, we should be attentive to both possibilities.
	\item A researcher’s findings need not be generalizable to be valuable; samples that allow for comparisons of theoretically important concepts or variables may yield findings that contribute to our social theories and our understandings of social processes.
	
\end{itemize}

\printbibliography
