%*****************************************
\chapter{Interpretive Research}\label{ch13:interpretive_research}
%*****************************************
% Chapter 12 from the Anol book.
% This seems to be a collection of several methods, like ethnography. Perhaps I need to move some of that material into this chapter.
%TODO Status: Pre-draft

\begin{center}
	\begin{objbox}{Objectives}
		\begin{itemize}
			\setlength{\itemsep}{0pt}
			\setlength{\parskip}{0pt}
			\setlength{\parsep}{0pt}
			
			\item x1.
			\item x2.
			\item x3.
		\end{itemize}
	\end{objbox}
\end{center}


\section{Introduction}

This chapter explores interpretive research. Recall that positivist or deductive methods, such as laboratory experiments and survey research, are those that are specifically intended for theory (or hypotheses) testing, while interpretive or inductive methods, such as action research and ethnography, are intended for theory building. Unlike a positivist method, where researchers start with a theory and tests theoretical postulates using empirical data, in interpretive methods, researchers start with data and try to derive a theory about the phenomenon of interest from the observed data.

The term ``interpretive research'' is often used loosely and synonymously with ``qualitative research'', although the two concepts are quite different. Interpretive research is a research paradigm that is based on the assumption that social reality is not singular or objective, but is rather shaped by human experiences and social contexts (ontology), and is therefore best studied within its socio-historic context by reconciling the subjective interpretations of its various participants (epistemology). Because interpretive researchers view social reality as being embedded within and impossible to abstract from their social settings, they ``interpret'' the reality though a ``sense-making'' process rather than a hypothesis testing process. This is in contrast to the positivist or functionalist paradigm that assumes that the reality is relatively independent of the context, can be abstracted from their contexts, and studied in a decomposable functional manner using objective techniques such as standardized measures. Whether a researcher should pursue interpretive or positivist research depends on paradigmatic considerations about the nature of the phenomenon under consideration and the best way to study it.

However, qualitative versus quantitative research refers to empirical or data-oriented considerations about the type of data to collect and how to analyze them. Qualitative research relies mostly on non-numeric data, such as interviews and observations, in contrast to quantitative research which employs numeric data such as scores and metrics. Hence, qualitative research is not amenable to statistical procedures such as regression analysis, but is coded using techniques like content analysis. Sometimes, coded qualitative data is tabulated quantitatively as frequencies of codes, but this data is not statistically analyzed. Many puritan interpretive researchers reject this coding approach as a futile effort to seek consensus or objectivity in a social phenomenon which is essentially subjective.

Although interpretive research tends to rely heavily on qualitative data, quantitative data may add more precision and clearer understanding of the phenomenon of interest than qualitative data. For example, Eisenhardt (1989), in her interpretive study of decision making in high-velocity firms collected numeric data on how long it took each firm to make certain strategic decisions (which ranged from 1.5 months to 18 months), how many decision alternatives were considered for each decision, and surveyed her respondents to capture their perceptions of organizational conflict. Such numeric data helped her clearly distinguish the high-speed decision making firms from the low-speed decision makers, without relying on respondents' subjective perceptions, which then allowed her to examine the number of decision alternatives considered by and the extent of conflict in high-speed versus low-speed firms. Interpretive research should attempt to collect both qualitative and quantitative data pertaining to their phenomenon of interest, and so should positivist research as well. Joint use of qualitative and quantitative data, often called ``mixedmode designs'', may lead to unique insights and are highly prized in the scientific community.

Interpretive research has its roots in anthropology, sociology, psychology, linguistics, and semiotics, and has been available since the early 19th century, long before positivist techniques were developed. Many positivist researchers view interpretive research as erroneous and biased, given the subjective nature of the qualitative data collection and interpretation process employed in such research. However, the failure of many positivist techniques to generate interesting insights or new knowledge have resulted in a resurgence of interest in interpretive research since the 1970's, albeit with exacting methods and stringent criteria to ensure the reliability and validity of interpretive inferences.

\section{Distinctions From Positivist Research}

In addition to fundamental paradigmatic differences in ontological and epistemological assumptions discussed above, interpretive and positivist research differ in several other ways. First, interpretive research employs a theoretical sampling strategy, where study sites, respondents, or cases are selected based on theoretical considerations such as whether they fit the phenomenon being studied (e.g., sustainable practices can only be studied in organizations that have implemented sustainable practices), whether they possess certain characteristics that make them uniquely suited for the study (e.g., a study of the drivers of firm innovations should include some firms that are high innovators and some that are low innovators, in order to draw contrast between these firms), and so forth. In contrast, positivist research employs random sampling (or a variation of this technique), where cases are chosen randomly from a population, for purposes of generalizability. Hence, convenience samples and small samples are considered acceptable in interpretive research as long as they fit the nature and purpose of the study, but not in positivist research.

Second, the role of researchers receives critical attention in interpretive research. In some methods such as ethnography, action research, and participant observation, researchers are considered part of the social phenomenon, and her specific role and involvement in the research process must be made clear during data analysis. In other methods, such as case research, researchers must take a ``neutral'' or unbiased stance during the data collection and analysis processes, and ensure that their personal biases or preconceptions does not taint the nature of subjective inferences derived from interpretive research. In positivist research, however, researchers are considered to be external to and independent of the research context and is not presumed to bias the data collection and analytic procedures.

Third, interpretive analysis is holistic and contextual, rather than being reductionist and isolationist. Interpretive interpretations tend to focus on language, signs, and meanings from the perspective of the participants involved in the social phenomenon, in contrast to statistical techniques that are employed heavily in positivist research. Rigor in interpretive research is viewed in terms of systematic and transparent approaches for data collection and analysis rather than statistical benchmarks for construct validity or significance testing.

Lastly, data collection and analysis can proceed simultaneously and iteratively in interpretive research. For instance, researchers may conduct an interview and code it before proceeding to the next interview. Simultaneous analysis helps researchers correct potential flaws in the interview protocol or adjust it to capture the phenomenon of interest better. Researchers may even change their original research question if they realize that their original research questions are unlikely to generate new or useful insights. This is a valuable but often understated benefit of interpretive research, and is not available in positivist research, where the research project cannot be modified or changed once the data collection has started without redoing the entire project from the start.

\section{Benefits and Challenges of Interpretive Research}

Interpretive research has several unique advantages. First, they are well-suited for exploring hidden reasons behind complex, interrelated, or multifaceted social processes, such as inter-firm relationships or inter-office politics, where quantitative evidence may be biased, inaccurate, or otherwise difficult to obtain. Second, they are often helpful for theory construction in areas with no or insufficient a priori theory. Third, they are also appropriate for studying context-specific, unique, or idiosyncratic events or processes. Fourth, interpretive research can also help uncover interesting and relevant research questions and issues for follow-up research.

At the same time, interpretive research also has its own set of challenges.

\begin{itemize}
	\item This type of research tends to be more time and resource intensive than positivist research in data collection and analytic efforts. Too little data can lead to false or premature assumptions, while too much data may not be effectively processed by the researcher.
	\item Interpretive research requires well-trained researchers who are capable of seeing and interpreting complex social phenomenon from the perspectives of the embedded participants and reconciling the diverse perspectives of these participants, without injecting their personal biases or preconceptions into their inferences.
	\item All participants or data sources may not be equally credible, unbiased, or knowledgeable about the phenomenon of interest, or may have undisclosed political agendas, which may lead to misleading or false impressions. Inadequate trust between participants and researcher may hinder full and honest self-representation by participants, and such trust building takes time. It is the job of the interpretive researcher to ``see through the smoke'' (hidden or biased agendas) and understand the true nature of the problem.
	\item Given the heavily contextualized nature of inferences drawn from interpretive research, such inferences do not lend themselves well to replicability or generalizability. 
	\item Interpretive research may sometimes fail to answer the research questions of interest or predict future behaviors.
\end{itemize}

\section{Characteristics of Interpretive Research}

All interpretive research must adhere to a common set of principles, as described below.

\begin{description}
	\item[Naturalistic inquiry] Social phenomena must be studied within their natural setting. Because interpretive research assumes that social phenomena are situated within and cannot be isolated from their social context, interpretations of such phenomena must be grounded within their socio-historical context. This implies that contextual variables should be observed and considered in seeking explanations of a phenomenon of interest, even though context sensitivity may limit the generalizability of inferences.

	\item[Researcher as instrument] Researchers are often embedded within the social context that they are studying, and are considered part of the data collection instrument in that they must use their observational skills, their trust with the participants, and their ability to extract the correct information. Further, their personal insights, knowledge, and experiences of the social context is critical to accurately interpreting the phenomenon of interest. At the same time, researchers must be fully aware of their personal biases and preconceptions, and not let such biases interfere with their ability to present a fair and accurate portrayal of the phenomenon.

	\item[Interpretive analysis] Observations must be interpreted through the eyes of the participants embedded in the social context. Interpretation must occur at two levels. The first level involves viewing or experiencing the phenomenon from the subjective perspectives of the social participants. The second level is to understand the meaning of the participants' experiences in order to provide a ``thick description'' or a rich narrative story of the phenomenon of interest that can communicate why participants acted the way they did.

	\item[Use of expressive language] Documenting the verbal and non-verbal language of participants and the analysis of such language are integral components of interpretive analysis. The study must ensure that the story is viewed through the eyes of a person, and not a machine, and must depict the emotions and experiences of that person, so that readers can understand and relate to that person. Use of imageries, metaphors, sarcasm, and other figures of speech is very common in interpretive analysis.

	\item[Temporal nature] Interpretive research is often not concerned with searching for specific answers, but with understanding or ``making sense of'' a dynamic social process as it unfolds over time. Hence, such research requires an immersive involvement of the researcher at the study site for an extended period of time in order to capture the entire evolution of the phenomenon of interest.

	\item[Hermeneutic circle] Interpretive interpretation is an iterative process of moving back and forth from pieces of observations (text) to the entirety of the social phenomenon (context) to reconcile their apparent discord and to construct a theory that is consistent with the diverse subjective viewpoints and experiences of the embedded participants. Such iterations between the understanding/meaning of a phenomenon and observations must continue until ``theoretical saturation'' is reached, whereby any additional iteration does not yield any more insight into the phenomenon of interest.

\end{description}

\section{Interpretive Data Collection}

Data is collected in interpretive research using a variety of techniques. The most frequently used technique is interviews (face-to-face, telephone, or focus groups). Interview types and strategies are discussed in detail in a previous chapter on survey research. A second technique is observation. Observational techniques include direct observation, where researchers are a neutral and passive external observer and are not involved in the phenomenon of interest (as in case research), and participant observation, where researchers are an active participant in the phenomenon and their inputs or mere presence influence the phenomenon being studied (as in action research). A third technique is documentation, where external and internal documents, such as memos, electronic mails, annual reports, financial statements, newspaper articles, websites, may be used to cast further insight into the phenomenon of interest or to corroborate other forms of evidence.

\section{Interpretive Research Designs}

\subsection{Case research.} As discussed in the previous chapter, case research is an intensive longitudinal study of a phenomenon at one or more research sites for the purpose of deriving detailed, contextualized inferences and understanding the dynamic process underlying a phenomenon of interest. Case research is a unique research design in that it can be used in an interpretive manner to build theories or in a positivist manner to test theories. The previous chapter on case research discusses both techniques in depth and provides illustrative exemplars. Furthermore, the case researcher is a neutral observer (direct observation) in the social setting rather than an active participant (participant observation). As with any other interpretive approach, drawing meaningful inferences from case research depends heavily on the observational skills and integrative abilities of the researcher.

Case research, also called case study, is a method of intensively studying a phenomenon over time within its natural setting in one or a few sites. Multiple methods of data collection, such as interviews, observations, prerecorded documents, and secondary data, may be employed and inferences about the phenomenon of interest tend to be rich, detailed, and contextualized. Case research can be employed in a positivist manner for the purpose of theory testing or in an interpretive manner for theory building. This method is more popular in business research than in other social science disciplines.

Case research has several unique strengths over competing research methods such as experiments and survey research. First, case research can be used for either theory building or theory testing, while positivist methods can be used for theory testing only. In interpretive case research, the constructs of interest need not be known in advance, but may emerge from the data as the research progresses. Second, the research questions can be modified during the research process if the original questions are found to be less relevant or salient. This is not possible in any positivist method after the data is collected. Third, case research can helpderive richer, more contextualized, and more authentic interpretation of the phenomenon of interest than most other research methods by virtue of its ability to capture a rich array of contextual data. Fourth, the phenomenon of interest can be studied from the perspectives of multiple participants and using multiple levels of analysis (e.g., individual and organizational).

At the same time, case research also has some inherent weaknesses. Because it involves no experimental control, internal validity of inferences remain weak. Of course, this is a common problem for all research methods except experiments. However, as described later, the problem of controls may be addressed in case research using ``natural controls''. Second, the quality of inferences derived from case research depends heavily on the integrative powers of the researcher. An experienced researcher may see concepts and patterns in case data that a novice researcher may miss. Hence, the findings are sometimes criticized as being subjective. Finally, because the inferences are heavily contextualized, it may be difficult to generalize inferences from case research to other contexts or other organizations.

It is important to recognize that case research is different from case descriptions such as Harvard case studies discussed in business classes. While case descriptions typically describe an organizational problem in rich detail with the goal of stimulating classroom discussion and critical thinking among students, or analyzing how well an organization handled a specific problem, case research is a formal research technique that involves a scientific method to derive explanations of organizational phenomena.

Case research is a difficult research method that requires advanced research skills on the part of the researcher, and is therefore, often prone to error. Benbasat et al. (1987)8 describe five problems frequently encountered in case research studies.

\begin{itemize}
	\item Many case research studies start without specific research questions, and therefore end up without having any specific answers or insightful inferences.
	\item Case sites are often chosen based on access and convenience, rather than based on the fit with the research questions, and are therefore cannot adequately address the research questions of interest.
	\item Researchers often do not validate or triangulate data collected using multiple means, which may lead to biased interpretation based on responses from biased interviewees.
	\item Many studies provide very little details on how data was collected (e.g., what interview questions were used, which documents were examined, what are the organizational positions of each interviewee, etc.) or analyzed, which may raise doubts about the reliability of the inferences.
	\item Despite its strength as a longitudinal research method, many case research studies do not follow through a phenomenon in a longitudinal manner, and hence present only a cross-sectional and limited view of organizational processes and phenomena that are temporal in nature.
\end{itemize}

\subsubsection{Key Decisions in Case Research}

Several key decisions must be made by a researcher when considering a case research method. 

\begin{itemize}
	\item Is this the right method for the research questions being studied? The case research method is particularly appropriate for exploratory studies for discovering relevant constructs in areas where theory building at the formative stages, for studies where the experiences of participants and context of actions are critical, and for studies aimed at understanding complex, temporal processes (why and how of a phenomenon) rather than factors or causes (what). This method is well-suited for studying complex organizational processes that involve multiple participants and interacting sequences of events, such as organizational change and large-scale technology implementation projects.

	\item What is the appropriate unit of analysis for a case research study? Since case research can simultaneously examine multiple units of analyses, researchers must decide whether they wishe to study a phenomenon at the individual, group, and organizational level or at multiple levels. For instance, a study of group decision making or group work may combine individual-level constructs such as individual participation in group activities with group-level constructs, such as group cohesion and group leadership, to derive richer understanding than that can be achieved from a single level of analysis.

	\item Should researchers employ a single-case or multiple-case design? The single case design is more appropriate at the outset of theory generation, if the situation is unique or extreme, if it is revelatory (i.e., the situation was previously inaccessible for scientific investigation), or if it represents a critical or contrary case for testing a well-formulated theory. The multiple case design is more appropriate for theory testing, for establishing generalizability of inferences, and for developing richer and more nuanced interpretations of a phenomenon. Yin (1984)9 recommends the use of multiple case sites with replication logic, viewing each case site as similar to one experimental study, and following rules of scientific rigor similar to that used in positivist research.

	\item What sites should be chosen for case research? Given the contextualized nature of inferences derived from case research, site selection is a particularly critical issue because selecting the wrong site may lead to the wrong inferences. If the goal of the research is to test theories or examine generalizability of inferences, then dissimilar case sites should be selected to increase variance in observations. For instance, if the goal of the research is to understand the process of technology implementation in firms, a mix of large, mid-sized, and small firms should be selected to examine whether the technology implementation process differs with firm size. Site selection should not be opportunistic or based on convenience, but rather based on the fit with research questions though a process called ``theoretical sampling.''

	\item What techniques of data collection should be used in case research? Although interview (either open-ended/unstructured or focused/structured) is by far the most popular data collection technique for case research, interview data can be supplemented or corroborated with other techniques such as direct observation (e.g., attending executive meetings, briefings, and planning sessions), documentation (e.g., internal reports, presentations, and memoranda, as well as external accounts such as newspaper reports), archival records (e.g., organization charts, financial records, etc.), and physical artifacts (e.g., devices, outputs, tools). Furthermore, the researcher should triangulate or validate observed data by comparing responses between interviewees.
\end{itemize}

\subsubsection{Conducting Case Research}

Most case research studies tend to be interpretive in nature. Interpretive case research is an inductive technique where evidence collected from one or more case sites is systematically analyzed and synthesized to allow concepts and patterns to emerge for the purpose of building new theories or expanding existing ones. Eisenhardt (1989)10 propose a ``roadmap'' for building theories from case research, a slightly modified version of which is described below. For positivist case research, some of the following stages may need to be rearranged or modified; however sampling, data collection, and data analytic techniques should generally remain the same.

\paragraph{Define research questions.} Like any other scientific research, case research must also start with defining research questions that are theoretically and practically interesting, and identifying some intuitive expectations about possible answers to those research questions or preliminary constructs to guide initial case design. In positivist case research, the preliminary constructs are based on theory, while no such theory or hypotheses should be considered ex ante in interpretive research. These research questions and constructs may be changed in interpretive case research later on, if needed, but not in positivist case research.

\paragraph{Select case sites.} Researchers should use a process of ``theoretical sampling'' (not random sampling) to identify case sites. In this approach, case sites are chosen based on theoretical, rather than statistical, considerations, for instance, to replicate previous cases, to extend preliminary theories, or to fill theoretical categories or polar types. Care should be taken to ensure that the selected sites fit the nature of research questions, minimize extraneous variance or noise due to firm size, industry effects, and so forth, and maximize variance in the dependent variables of interest. For instance, if the goal of the research is to examine how some firms innovate better than others, researchers should select firms of similar size within the same industry to reduce industry or size effects, and select some more innovative and some less innovative firms to increase variation in firm innovation. Instead of cold-calling or writing to a potential site, it is better to contact someone at executive level inside each firm who has the authority to approve the project or someone who can identify a person of authority. During initial conversations, researchers should describe the nature and purpose of the project, any potential benefits to the case site, how the collected data will be used, the people involved in data collection (other researchers, research assistants, etc.), desired interviewees, and the amount of time, effort, and expense required of the sponsoring organization. Researchers must also assure confidentiality, privacy, and anonymity of both the firm and the individual respondents.

\paragraph{Create instruments and protocols.} Since the primary mode of data collection in case research is interviews, an interview protocol should be designed to guide the interview process. This is essentially a list of questions to be asked. Questions may be open-ended (unstructured) or closed-ended (structured) or a combination of both. The interview protocol must be strictly followed, and the interviewer must not change the order of questions or skip any question during the interview process, although some deviations are allowed to probe further into respondent's comments that are ambiguous or interesting. The interviewer must maintain a neutral tone, not lead respondents in any specific direction, say by agreeing or disagreeing with any response. More detailed interviewing techniques are discussed in the chapter on surveys. In addition, additional sources of data, such as internal documents and memorandums, annual reports, financial statements, newspaper articles, and direct observations should be sought to supplement and validate interview data.

\paragraph{Select respondents.} Select interview respondents at different organizational levels, departments, and positions to obtain divergent perspectives on the phenomenon of interest. A random sampling of interviewees is most preferable; however a snowball sample is acceptable, as long as a diversity of perspectives is represented in the sample. Interviewees must be selected based on their personal involvement with the phenomenon under investigation and their ability and willingness to answer the researcher's questions accurately and adequately, and not based on convenience or access.

\paragraph{Start data collection.} It is usually a good idea to electronically record interviews for future reference. However, such recording must only be done with the interviewee's consent. Even when interviews are being recorded, the interviewer should take notes to capture important comments or critical observations, behavioral responses (e.g., respondent's body language), and the researcher's personal impressions about the respondent and his/her comments. After each interview is completed, the entire interview should be transcribed verbatim into a text document for analysis.

\paragraph{Conduct within-case data analysis.} Data analysis may follow or overlap with data collection. Overlapping data collection and analysis has the advantage of adjusting the data collection process based on themes emerging from data analysis, or to further probe into these themes. Data analysis is done in two stages. In the first stage (within-case analysis), the researcher should examine emergent concepts separately at each case site and patterns between these concepts to generate an initial theory of the problem of interest. The researcher can interview data subjectively to ``make sense'' of the research problem in conjunction with using her personal observations or experience at the case site. Alternatively, a coding strategy such as Glasser and Strauss' (1967) grounded theory approach, using techniques such as open coding, axial coding, and selective coding, may be used to derive a chain of evidence and inferences. Homegrown techniques, such as graphical representation of data (e.g., network diagram) or sequence analysis (for longitudinal data) may also be used. Note that there is no predefined way of analyzing the various types of case data, and the data analytic techniques can be modified to fit the nature of the research project.

\paragraph{Conduct cross-case analysis.} Multi-site case research requires cross-case analysis as the second stage of data analysis. In such analysis, researchers should look for similar concepts and patterns between different case sites, ignoring contextual differences that may lead to idiosyncratic conclusions. Such patterns may be used for validating the initial theory, or for refining it (by adding or dropping concepts and relationships) to develop a more inclusive and generalizable theory. This analysis may take several forms. For instance, researchers may select categories (e.g., firm size, industry, etc.) and look for within-group similarities and between-group differences (e.g., high versus low performers, innovators versus laggards). Alternatively, they can compare firms in a pair-wise manner listing similarities and differences across pairs of firms.

\paragraph{Build and test hypotheses.} Based on emergent concepts and themes that are generalizable across case sites, tentative hypotheses are constructed. These hypotheses should be compared iteratively with observed evidence to see if they fit the observed data, and if not, the constructs or relationships should be refined. Also researchers should compare the emergent constructs and hypotheses with those reported in the prior literature to make a case for their internal validity and generalizability. Conflicting findings must not be rejected, but rather reconciled using creative thinking to generate greater insight into the emergent theory. When further iterations between theory and data yield no new insights or changes in the existing theory, ``theoretical saturation'' is reached and the theory building process is complete.

\paragraph{Write case research report.} In writing the report, researchers should describe very clearly the detailed process used for sampling, data collection, data analysis, and hypotheses development, so that readers can independently assess the reasonableness, strength, and consistency of the reported inferences. A high level of clarity in research methods is needed to ensure that the findings are not biased by the researcher's preconceptions.

\section{Ethnography}
% Anol Ch 12

The ethnographic research method, derived largely from the field of anthropology, emphasizes studying a phenomenon within the context of its culture. Researchers must be deeply immersed in the social culture over an extended period of time (usually eight months to two years) and should engage, observe, and record the daily life of the studied culture and its social participants within their natural setting. The primary mode of data collection is participant observation, and data analysis involves a ``sense-making'' approach. In addition, researchers must take extensive field notes and narrate their experiences in descriptive detail so that readers may experience the same culture. In this method, researchers have two roles: rely on her unique knowledge and engagement to generate insights (theory), and convince the scientific community of the trans-situational nature of the studied phenomenon.

The classic example of ethnographic research is Jane Goodall's study of primate behaviors, where she lived with chimpanzees in their natural habitat at Gombe National Park in Tanzania, observed their behaviors, interacted with them, and shared their lives. During that process, she chronicled how chimpanzees seek food and shelter, how they socialize with each other, their communication patterns, their mating behaviors, and so forth. A more contemporary example of ethnographic research is Myra Bluebond-Langer's (1996)14 study of decision making in families with children suffering from life-threatening illnesses, and the physical, psychological, environmental, ethical, legal, and cultural issues that influence such decision-making. The researcher followed the experiences of approximately 80 children with incurable illnesses and their families for a period of over two years. Data collection involved participant observation and formal/informal conversations with children, their parents and relatives, and health care providers to document their lived experience.

\subsection{Action research.} Action research is a qualitative but positivist research design aimed at theory testing rather than theory building. This is an interactive design that assumes that complex social phenomena are best understood by introducing changes, interventions, or ``actions'' into those phenomena and observing the outcomes of such actions on the phenomena of interest. In this method, the researcher is usually a consultant or an organizational member embedded into a social context (such as an organization), who initiates an action in response to a social problem, and examines how her action influences the phenomenon while also learning and generating insights about the relationship between the action and the phenomenon. Examples of actions may include organizational change programs, such as the introduction of new organizational processes, procedures, people, or technology or replacement of old ones, initiated with the goal of improving an organization's performance or profitability in its business environment. The researcher's choice of actions must be based on theory, which should explain why and how such actions may bring forth the desired social change. The theory is validated by the extent to which the chosen action is successful in remedying the targeted problem. Simultaneous problem solving and insight generation is the central feature that distinguishes action research from other research methods (which may not involve problem solving) and from consulting (which may not involve insight generation). Hence, action research is an excellent method for bridging research and practice.

There are several variations of the action research method. The most popular of these method is the participatory action research, designed by Susman and Evered (1978)13. This method follows an action research cycle consisting of five phases: (1) diagnosing, (2) action planning, (3) action taking, (4) evaluating, and (5) learning (see Figure 10.1). Diagnosing involves identifying and defining a problem in its social context. Action planning involves identifying and evaluating alternative solutions to the problem, and deciding on a future course of action (based on theoretical rationale). Action taking is the implementation of the planned course of action. The evaluation stage examines the extent to which the initiated action is successful in resolving the original problem, i.e., whether theorized effects are indeed realized in practice. In the learning phase, the experiences and feedback from action evaluation are used to generate insights about the problem and suggest future modifications or improvements to the action. Based on action evaluation and learning, the action may be modified or adjusted to address the problem better, and the action research cycle is repeated with the modified action sequence. It is suggested that the entire action research cycle be traversed at least twice so that learning from the first cycle can be implemented in the second cycle. The primary mode of data collection is participant observation, although other techniques such as interviews and documentary evidence may be used to corroborate the researcher's observations.

%Note: nice Action Research graphic here

\subsection{Phenomenology.} Phenomenology is a research method that emphasizes the study of conscious experiences as a way of understanding the reality around us. It is based on the ideas of German philosopher Edmund Husserl in the early 20th century who believed that human experience is the source of all knowledge. Phenomenology is concerned with the systematic reflection and analysis of phenomena associated with conscious experiences, such as human judgment, perceptions, and actions, with the goal of (1) appreciating and describing social reality from the diverse subjective perspectives of the participants involved, and (2) understanding the symbolic meanings (``deep structure'') underlying these subjective experiences. Phenomenological inquiry requires that researchers eliminate any prior assumptions and personal biases, empathize with the participant's situation, and tune into existential dimensions of that situation, so that they can fully understand the deep structures that drives the conscious thinking, feeling, and behavior of the studied participants.

Some researchers view phenomenology as a philosophy rather than as a research method. In response to this criticism, Giorgi and Giorgi (2003)15 developed an existential phenomenological research method to guide studies in this area. This method, illustrated in Figure 10.2, can be grouped into data collection and data analysis phases. In the data collection phase, participants embedded in a social phenomenon are interviewed to capture their subjective experiences and perspectives regarding the phenomenon under investigation. Examples of questions that may be asked include ``can you describe a typical day'' or ``can you describe that particular incident in more detail?'' These interviews are recorded and transcribed for further analysis. During data analysis, researchers read the transcripts to: (1) get a sense of the whole, and (2) establish ``units of significance'' that can faithfully represent participants' subjective experiences. Examples of such units of significance are concepts such as ``felt space'' and ``felt time,'' which are then used to document participants' psychological experiences. For instance, did participants feel safe, free, trapped, or joyous when experiencing a phenomenon (``felt-space'')? Did they feel that their experience was pressured, slow, or discontinuous (``felt-time'')? Phenomenological analysis should take into account the participants' temporal landscape (i.e., their sense of past, present, and future), and researchers must transpose themselves in an imaginary sense in the participant's situation (i.e., temporarily live the participant's life). The participants' lived experience is described in form of a narrative or using emergent themes. The analysis then delves into these themes to identify multiple layers of meaning while retaining the fragility and ambiguity of subjects' lived experiences.

\section{Rigor in Interpretive Research}

While positivist research employs a ``reductionist'' approach by simplifying social reality into parsimonious theories and laws, interpretive research attempts to interpret social reality through the subjective viewpoints of the embedded participants within the context where the reality is situated. These interpretations are heavily contextualized, and are naturally less generalizable to other contexts. However, because interpretive analysis is subjective and sensitive to the experiences and insight of the embedded researcher, it is often considered less rigorous by many positivist (functionalist) researchers. Because interpretive research is based on different set of ontological and epistemological assumptions about social phenomenon than positivist research, the positivist notions of rigor, such as reliability, internal validity, and generalizability, do not apply in a similar manner. However, Lincoln and Guba (1985)16 provide an alternative set of criteria that can be used to judge the rigor of interpretive research.

\begin{description}
	\item[Dependability] Interpretive research can be viewed as dependable or authentic if two researchers assessing the same phenomenon using the same set of evidence independently arrive at the same conclusions or the same researcher observing the same or a similar phenomenon at different times arrives at similar conclusions. This concept is similar to that of reliability in positivist research, with agreement between two independent researchers being similar to the notion of inter-rater reliability, and agreement between two observations of the same phenomenon by the same researcher akin to test-retest reliability. To ensure dependability, interpretive researchers must provide adequate details about their phenomenon of interest and the social context in which it is embedded so as to allow readers to independently authenticate their interpretive inferences.

	\item[Credibility] Interpretive research can be considered credible if readers find its inferences to be believable. This concept is akin to that of internal validity in functionalistic research. The credibility of interpretive research can be improved by providing evidence of the researcher's extended engagement in the field, by demonstrating data triangulation across subjects or data collection techniques, and by maintaining meticulous data management and analytic procedures, such as verbatim transcription of interviews, accurate records of contacts and interviews, and clear notes on theoretical and methodological decisions, that can allow an independent audit of data collection and analysis if needed.

	\item[Confirmability] Confirmability refers to the extent to which the findings reported in interpretive research can be independently confirmed by others (typically, participants). This is similar to the notion of objectivity in functionalistic research. Since interpretive research rejects the notion of an objective reality, confirmability is demonstrated in terms of ``inter- subjectivity'', i.e., if the study's participants agree with the inferences derived by the researcher. For instance, if a study's participants generally agree with the inferences drawn by a researcher about a phenomenon of interest (based on a review of the research paper or report), then the findings can be viewed as confirmable.

	\item[Transferability] Transferability in interpretive research refers to the extent to which the findings can be generalized to other settings. This idea is similar to that of external validity in functionalistic research. The researcher must provide rich, detailed descriptions of the research context (``thick description'') and thoroughly describe the structures, assumptions, and processes revealed from the data so that readers can independently assess whether and to what extent are the reported findings transferable to other settings.
	
\end{description}

\section{Summary}\label{ch13:summary}

\begin{center}
	\begin{tkawybox}{Summary}
		\begin{itemize}
			\setlength{\itemsep}{0pt}
			\setlength{\parskip}{0pt}
			\setlength{\parsep}{0pt}
			
			\item x1.
			\item x2.
			\item x3.
		\end{itemize}
	\end{tkawybox}
\end{center}
