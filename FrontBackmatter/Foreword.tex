%************************************************
\chapter*{Foreword}\label{ch:foreword}
%************************************************
%TODO text draft done

I have taught BASV 316, \textit{Introductory Methods of Analysis}, online for the University of Arizona in Sierra Vista since 2010 and enjoy working with students on research methodology. I wanted a textbook that presented research in a practical way so students could use the lessons learned in their own research projects. I found an excellent book but over the years the cost of that book increased to the point that I felt like it was an unfair burden on students. 

I began by looking for an acceptable “open source” book since authors make those available to students free of charge and I could modify the book to meet my own objectives. I could not find any that were focused on business research though I tried for several years—and keep looking to this day. I did, though, find a few open source books about research in the social and psychological sciences that were reasonably close to what I needed. So, I modified those books to emphasize business research and then provided my work to students free of charge. 

The following books were the major sources for this book. They are all open source and published under a Creative Commons license that permitted me to copy and modify them.

\begin{itemize}
	\item Bhattacherjee, Anol, \textit{Social Science Research: Principles, Methods, and Practices}, found at \url{http://scholarcommons.usf.edu/oa_textbooks/3/}
	\item Blackstone, Amy, \textit{Principles of Sociological Inquiry: Qualitative and Quantitative Methods}, found at \url{https://open.umn.edu/opentextbooks/BookDetail.aspx?bookId=139}
	\item Price, Paul, \textit{Research Methods in Psychology}, found at \url{http://open.lib.umn.edu/psychologyresearchmethods/}. Note: this book was later updated by Price and posted in HTML format at \url{https://github.com/CrumpLab/ResearchMethods} 
\end{itemize}

Three goals shaped the choices made about the topics covered by the text and how those topics are presented. 

\begin{itemize}
	\item The topics must have relevance for business students. 
	\item Both qualitative and quantitative research methods are given roughly equal attention since both types of research are used in business. 
	\item The text is engaging and readable.
\end{itemize}

While the book is useful in its current form, I will continually update it based on emerging trends in research. 

This book is published under a Creative Commons \href{https://creativecommons.org/licenses/by-nc-sa/4.0/legalcode}{ Attribution-NonCommercial-ShareAlike} license, just like the books that provided its foundation. The source is available at my GitHub account: \url{http://bit.ly/2xIjzXL}. It is my hope that students can use this book to learn about business research and other instructors can modify and use it for their own classes.

\bigskip
\begin{flushright}
  \textemdash \; George Self
\end{flushright}
