%*******************************************************
% Acknowledgments
%*******************************************************
\chapter*{Acknowledgments}
This book was created from my original work and from several Open Educational Resources. I gratefully acknowledge the contribution from these sources:

%\begin{enumerate}
%%  \item Author Unknown, \textit{Introductory Statistics}, found at the Saylor.org site: \url{https://www.saylor.org/site/textbooks/Introductory Statistics.pdf}
%    
%  \item Author Unknown, \textit{Principles of Sociological Inquiry}, found at the Saylor.org site: \url{https://www.saylor.org/site/textbooks/Principles of Sociological Inquiry.pdf}
%    
%%  \item Author Unknown, \textit{Research Methods in Psychology}, found at the Saylor.org site: \url{https://www.saylor.org/site/textbooks/Research Methods in Psychology.pdf}
%  
%  \item Bhattacherjee, Anol, ``Social Science Research: Principles, Methods, and Practices'' (2012). \textit{Textbooks Collection}. Book 3.   \url{http://scholarcommons.usf.edu/oa\_textbooks/3}
%  
%\end{enumerate}

\begin{itemize}
	\item Amy Blackstone's book about Sociological Research found at saylor.org\footnote{Blackstone, A. "Principles of sociological inquiry: Qualitative and quantitative methods, v. 1.0 (pp. 236)." (2012).}.
	\item Anol Bhattacherjee's book about Social Science Research found at the University of Southern Florida's Scholar Commons\footnote{Bhattacherjee, Anol. "Social science research: Principles, methods, and practices." (2012).}.
\end{itemize}

All of these books and materials were published under a Creative Commons license which is often said to incorporate five R's:  

\begin{itemize}
	\item Reuse: I am using the two books to form my own work.
	\item Redistribute: My book is being freely distributed to my students and I've made it available to anyone else who wants to use it.
	\item Revise: I revised this book to include examples that are focused on research normally needed by business and marketing students rather than social science students.
	\item Remix: I mixed both books into one, removed the redundant parts, and added my own components.
	\item Retain: Students are able to retain a copy of this book until they choose to delete it.
\end{itemize}


\bigskip
\begin{flushright}
  \textemdash  George Self
\end{flushright}



