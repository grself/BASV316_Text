% This file contains all of the acronymns and glossary entries for the book.

% Acronyms
% The ``code'' is what I use in the document to link to a definition here
% The ``name'' is what the user sees in the document
% The ``definition'' is what the user sees in the glossary
%\newacronym{code}{name}{Definition}

%\newacronym{nasa}{NASA}{National Aeronatics and Space Administration}

% Glossary
% The ``code'' is what I use in the document to link to a definition here
% The ``name'' is the phrase listed in the glossary
% The ``description'' defines the phrase
% Can also include a ``text'' entry that is how the phrase will appear in the text
% Ex: name{Latin Alphabet} uses caps in the Glossary
%     text{Latin alphabet} uses LC in the text
% Can also include a ``plural'' entry to spell non-standard plural words
% Ex: name{bravo} 
%     plural{bravi}

%\newglossaryentry{code}{name={Phrase in Glossary},
%		description={This describes the phrase.}}
% ``see'' creates a cross-link in the glossary, usually not needed
% Note that a glossary entry will not be printed if it is not in the
% main text somewhere.

%\newglossaryentry{myphone}{
%	name={George's Phone},
%	description={(520) 335-2958},
%	see={myaddr}
%}

% For long entries, the percent sign after the opening brace
% removes the CR/LF so you don't get an extra vertical space

% To use:
%   \gls{label} (regular form)
%   \glspl{label} (plural form)
%   \Gls{label} (Capitalized regular form)
%   \Glspl{label} (Capitalized plural form)



\longnewglossaryentry{anova}
{name={ANOVA}}
{%
	A test used to analyze the difference in the variation of measured observations. ANOVAs are considered to be ``one-way'' or ``two-way'' depending on the number of variables being analyzed.
}

\longnewglossaryentry{ancova}
{name={ANCOVA}}
{%
	A test used to analyze the effects of categorical variables on a main dependent variable in a research project.
}

\longnewglossaryentry{appliedresearch}
{name={applied research},
 see={basicresearch}}
{%
	Research that is intended to be applied to a situation rather than further the knowledge of some topic. For example, if a researcher completes a project designed to increase the sales of bottled water in a small town it would be considered applied research.
}

\longnewglossaryentry{basicresearch}
{name={basic research},
	see={appliedresearch}}
{%
	Research that is intended to be further the knowledge of some topic rather than be applied to a specific situation. For example, if a researcher completes a project designed to refine some aspect of the Law of Supply and Demand it would be considered basic research.  
}

\longnewglossaryentry{bias}
{name={bias},
 plural={biases}}
{%
	An undesired over- or under-estimate of the value of a population's parameter. Bias has many potential sources, including sampling error, measurement error, and missing data. On a survey question, bias tends to elicit a particular response which would skew the data collected.
}

\longnewglossaryentry{binaryscale}
{name={binary}}
{%
	A binary scale is used to measure nominal data that have only two values, like true/false or yes/no.
}

\longnewglossaryentry{bivariate}
{name={bivariate},
	see={univariate}}
{%
	A type of analysis involving two variables. Examples of bivariate analysis include finding a correlation and regression.
}

\longnewglossaryentry{boundarycondition}
{name={boundary condition}}
{%
	The assumptions about the ``who, when, and where'' in a theory. Boundary conditions govern how a theory can be applied or not applied.
}

\longnewglossaryentry{chisquare}
{name={chi-square}}
{%
	The chi-square test is used to determine if there is a significant difference between the actual and expected result of a non-parametric (usually nominal) variable in a research project.
}

\longnewglossaryentry{concurrentvalidity}
{name={concurrent validity},
	see={predictivevalidity}}
{%
	The degree that a measure relates to an outcome that is presumed to occur simultaneously. For example, the results of a new employee attitude test would be the same as an older test if those tests have high concurrent validity.
}

\longnewglossaryentry{constructivism}
{name={constructivism}}
{%
	A philosophical stance that reality is a construct of the human mind and is, therefore, subjective. Normally, qualitative research methods are used by researchers who are constructivists.
}

\longnewglossaryentry{construct}
{name={construct}}
{%
	The characteristic of a person or organization to be assessed. Normally, a construct is not directly measurable so various indirect indicators must be used. Examples of constructs include employee attitude, happiness, or self-esteem. Usually, researchers operationalize a concept by defining it in measurable terms, and that becomes a construct.
}

\longnewglossaryentry{constructvalidity}
{name={construct validity},
	see={validity}}
{%
	The degree to which a test measures what it claims to measure. For example, if a research project purports to investigate some aspect of local farmers' markets, does the project actually research that aspect? Construct validity is sometimes thought to be the overarching type of validity since research projects that do not address the construct of interest can have no other validity.
}

\longnewglossaryentry{contentvalidity}
{name={content validity},
	see={validity}}
{%
	A determination of whether a measure correctly assesses the construct's content. For example, if a research project is attempting to determine the drivers for total sales in a store but only measured the price of the merchandise being sold then ignoring factors like advertising, competition, and even the general economy of the region would call into question the content validity of the study.
}

\longnewglossaryentry{continuousdata}
{name={continuous data},
	see={quantitativedata}}
{%
	Continuous data are a type of quantitative data that can represent any measured value, including fractions and decimals. In mathematics terms, continuous data are members of the real number system. 
}

\newacronym{cpi}{CPI}{Consumer Price Index}

\longnewglossaryentry{convergentvalidity}
{name={convergent validity},
	see={discriminantvalidity}}
{%
	The closeness that two measures relate to, ``converge on,'' a single construct. For example, if a research project measures the amount of sales of carbonated drinks, fruit juices, and bottled water in a store it would be expected that those would converge on a construct of ``drink sales.''
}

\longnewglossaryentry{correlation}
{name={correlation}}
{%
	A correlation is a relationship between two variables. Correlations are normally defined statistically as a value between $ -1.00 $ and $ +1.00 $. Correlations should not be confused with causation but only indicate that two variables seem to vary together. 
}

\longnewglossaryentry{covariate}
{name={covariate}}
{%
	A variable that is not manipulated but is observed during a research project. Covariates can sometimes affect the outcome of a project but may not be the variable of direct interest. For example, ``education level'' is often a covariate in research projects since that would affect things like income and standard of living but may not be the focus of the study.
}

\longnewglossaryentry{criterionvalidity}
{name={criterion validity},
	see={validity}}
{%
	The degree to which a measure is related to an outcome.  
}

\longnewglossaryentry{crosssectional}
{name={cross-sectional},
	see={longitudinal}}
{%
	A type of research that is conducted in a single point in time that crosses multiple analytical units. This is most often in reference to a survey but could be applied to other research methods. For example, a survey of several different small business owners in a single city would be cross-sectional. 
}

\longnewglossaryentry{database}
{name={database}}
{%
	A database is a collection of data that is organized in such a way that the data can be easily managed. While the internal structure of a database can be complex, it is typically represented as tables with data in rows and columns, like a spread sheet.
}

\longnewglossaryentry{deductiveresearch}
{name={deductive research}}
{%
	A research methodology that works from a general theory to specific observations. This is sometimes called the ``theory-testing'' form of research.
}

\longnewglossaryentry{dependentvariable}
{name={dependent variable},
	see={independentvariable}}
{%
	Dependent variables are the outcomes for an observation. For example, if a medical researcher conducts an experiment where a drug is administered and then the patient's blood pressure is measured, the blood pressure reading is the dependent variable; that is, the blood pressure depends on the drug being administered.
}

\longnewglossaryentry{descriptiveresearch}
{name={descriptive research},
 see={exploratoryresearch}}
{%
	Research that is designed to describe observed phenomena. The goal is to improve 	understanding rather than explore new ideas.
}

\longnewglossaryentry{discretedata}
{name={discrete data},
	see={quantitativedata}}
{%
	Discrete data are a type of quantitative data that can be counted with integers. In mathematics terms, discrete data are integers, though negative values are rather rare. 
}

\longnewglossaryentry{discriminantvalidity}
{name={discriminant validity},
	see={convergentvalidity}}
{%
	The degree that a measure does not measure, ``discriminates between,'' one of two competing constructs. For example, a measure of the sale of toiletries in a department store would not be related to the construct of ``drink sales.'' 
}

\newglossaryentry{epistemology}{
	name={epistomology}, 
	description={A branch of philosophy that is concerned with the sources of knowledge.}
}

\longnewglossaryentry{excesskurtosis}
{name={excess kurtosis}}
{%
	Excess kurtosis is a measure of the ``tailedness'' of a normal distribution. Greater excess kurtosis values indicate longer ``tails'' (and a ``sharper'' appearance) on the graph of the distribution.
}

\longnewglossaryentry{explanatoryresearch}
{name={explanatory research},
	see={exploratoryresearch}}
{%
	Research that is designed to explain an observed phenomena or process. 
}

\longnewglossaryentry{explanatorypower}
{name={explanatory power}}
{%
	A theory or hypothesis has explanatory power if it accurately predicts phenomena. This can be statistically measured in quantitative research projects by calculating variance in regression analysis.
}

\newglossaryentry{exploratoryresearch}{
	name={exploratory research}, 
	description={Research that explores data in an effort to find new ideas.}
}

\longnewglossaryentry{externalvalidity}
{name={external validity},
	see={validity}}
{%
	The degree to which a research project's results can be applied outside the context of the study. For example, if the results of a research project that studied manufacturing firms in the mid-west could be applied to firms in the south then that study would have high external validity.
}

\longnewglossaryentry{facevalidity}
{name={face validity},
	see={validity}}
{%
	A determination of whether an indicator is a reasonable measure of an underlying construct ``on its face.'' For example, is the amount of money spent on live theater tickets a measure social class?
}

\longnewglossaryentry{factorialdesign}
{name={factorial design}}
{%
	An experimental setup that includes one or more factors and their influence on the subject of interest. Each factor has two or more levels and each level is analyzed separately. For example, sex is often a factor in experiments where the treatment's affect on males, females, and others can be independently evaluated.
}

\longnewglossaryentry{falsifiability}
{name={falsifiability}}
{%
	To have credence, a theory or hypothesis must be disprovable; that is, there must be a way to prove it wrong using evidence.
}

\longnewglossaryentry{functionalism}
{name={functionalism}}
{%
	A belief in the practical application of a theory. Functionalism is more concerned with how a theory can be used in the real world than conducting	research for the sake of increasing understanding. 
}

\longnewglossaryentry{groundedtheory}
{name={grounded theory},
	plural={grounded theories}}
{%
	A theory based on observation rather than experimentation. Thus, the strength of the theory is dependent on the skill of the researcher and may not be repeatable by a different researcher or at a different time.
}

\longnewglossaryentry{guttmanscale}
{name={guttman}}
{%
	The Guttman scale uses a series of questions in increasing intensity to determine how strongly respondents believe some proposition. 
}

\longnewglossaryentry{hawthorne}
{name={hawthorne}
{%
	This term refers to the tendency of people to react differently when they know that they are being observed. For example, factory workers tend to work harder with fewer breaks when a researcher is watching. This is named for the Hawthorne Works electric company where the effect was first observed during experiments conducted the 1920s.
}

\longnewglossaryentry{hermeneutics}
{name={hermeneutics}}
{%
	Hermeneutics is the study of the theory and principle of interpretation. This word originated with the study of biblical and philosophical texts but is widely applied in fields like law and history.
}

\longnewglossaryentry{hypothesis}
{name={hypothesis},
	plural={hypotheses}}
{%
	A proposed explanation for an observed phenomenon. Often, a hypothesis that may be based on incomplete information is the starting point for further investigation. As an example, if a merchant notices that eye-level shelves tend to need restocking more frequently a hypothesis may be proposed that shoppers purchase goods from eye-level shelves first. 
}

\longnewglossaryentry{idiographic}
{name={idiographic}}
{%
	An explanation for an observed phenomenon that explains only a single case and is not applicable to a wider population.
}

\longnewglossaryentry{independentvariable}
{name={independent variable},
	see={dependentvariable}}
{%
	Independent variables are those that create an observed effect. For example, if a farmer conducts an experiment where different types of fertilizer are applied to two fields in order to see which is more effective then the type of fertilizer is the independent variable; that is, the type of fertilizer is the variable that is creating the observed effect.
}

\longnewglossaryentry{inductiveresearch}
{name={inductive research}}
{%
	A research methodology that works from specific observations to a general theory. This is sometimes called the ``theory-building'' form of research.
}

\newacronym{irb}{IRB}{Institutional Review Board}

\longnewglossaryentry{internalvalidity}
{name={internal validity},
	see={validity}}
{%
	The degree to which a research project avoids confounding multiple variables within the study. A project with high internal validity facilitates selecting one explanation over an alternate since the number of confounding variables are controlled.
}

\longnewglossaryentry{interpretivism}
{name={interpretivism}}
{%
	A research method that relies on observation and techniques like interviews to understand phenomena. 
}

\longnewglossaryentry{intervaldata}
{name={interval data},
	see={quantitativedata}}
{%
	Interval data are a type of quantitative data that are measured along a scale where each point is an equal distance from the next. It is possible to compare the distance between two points on an interval scale; for example, the difference between $ 90 $ and $ 100 $ degrees is the same as the difference between $ 40 $ and $ 50 $ degrees. However, since an interval scale does not have a zero point, stating $ 100 $ degrees is twice as hot as $ 50 $ is not possible. 
}

\longnewglossaryentry{likertscale}
{name={likert}}
{%
	The Likert scale is one of the most commonly-used instruments for measuring attitudes and opinions. It consists of a statement followed by, typically, five selections: Strongly Agree, Agree, Neutral, Disagree, Strongly Disagree.
}

\longnewglossaryentry{logic}
{name={logic}}
{%
	A systematic set of principles that can be used to help validate a theory. Logic is a mode of reasoning that can link observations to explanations.
}

\longnewglossaryentry{logicalconsistency}
{name={logical consistency}}
{%
	A theory or hypothesis is logically consistent when all of the constructs, propositions, boundary conditions, and assumptions are congruous.
}

\longnewglossaryentry{longitudinal}
{name={longitudinal},
	see={crosssectional}}
{%
	A type of research that is conducted over a long period of time. This is most often in reference to a survey but could be applied to other research methods. For example, repeated surveys over a period of five years of small business owners in a single city would be longitudinal. 
}

\longnewglossaryentry{meta-analysis}
{name={meta-analysis}}
{%
	Meta-analysis is a method that combines the results of multiple previously-published studies. As an example, since there have been numerous studies exploring the impact of the internet on economic development in third world nations a researcher could conduct a meta-analysis of those research results to attempt to discover some overarching themes.
}

\longnewglossaryentry{model}
{name={model}}
{%
	A model is a representation of all or part of a system that is constructed to study that system. For example, meteorologists often create elaborate models to predict 	the path of a hurricane. 
}

\newacronym{nedv}{NEDV}{Nonequivalent Dependent Variable}

\newacronym{negd}{NEGD}{Nonequivalent Groups Design}

\longnewglossaryentry{nomothetic}
{name={nomothetic}}
{%
	An explanation for an observed phenomenon that is applicable across a wide population rather than a single example.
}

\longnewglossaryentry{nominaldata}
{name={nominal data},
	see={qualitativedata}}
{%
	Nominal data are a type of qualitative data that are grouped but with no order implied in the grouping. As an example, the gender of survey respondents is nominal data.
}

\longnewglossaryentry{nonparametric}
{name={nonparametric},
	see={nonparametric}}
{%
	Nonparametric data are data that do not conform to a distribution, are skewed, or are qualitative in nature. Statistical tests that work with nonparametric data are generally less powerful and predictive than tests that work with parametric data.
}

\longnewglossaryentry{nonprobibilitysampling}
{name={non-probability sampling},
	see={probabilitysampling}}
{%
	A type of sampling that does not involve a random selection from the population. This is called non-probability sampling since some members of the population have no probability of being selected.
}

\longnewglossaryentry{normaldistribution}
{name={normal distribution}}
{%
	A listing of all the possible values in a data set along with the number of times each value actually appears is called a ``distribution.'' In many, perhaps most, business research projects a distribution exhibits a bell shape when plotted on a graph where the values in the middle of the range are more frequent than values at the extremes of the range. This is called a ``normal distribution'' since it is so common.
}

\longnewglossaryentry{objectivism}
{name={objectivism}}
{%
	A philosophical stance that there exists an objective reality can be studied and understood.
}

\newglossaryentry{ontology}{
	name={ontology},
	description={The branch of philosophy that is concerned with the nature of reality.}
}

\longnewglossaryentry{operationalization}
{name={Operationalization}}
{%
	 The process of designing precise measures for abstract theoretical constructs. 
}

\longnewglossaryentry{ordinaldata}
{name={ordinal data},
	see={qualitativedata}}
{%
	Ordinal data are a type of qualitative data that are grouped where the groupings have an implied order. As an example, the ``satisfaction'' rating on a customer survey typically permits respondents to choose from several levels where one level is somehow better than another.
}

\longnewglossaryentry{pvalue}
{name={p-value}}
{%
	The probability that some finding reflects a true null hypothesis. In most business research, a p-value of less than $ 0.05 $ (or $ 5\% $) is desired to infer that the null hypothesis can be rejected.
}

\longnewglossaryentry{paradigm}
{name={paradigm},
	see={theory}}
{%
	A pattern or model of how things work in the world. 
}

\longnewglossaryentry{parametric}
{name={parametric},
	see={nonparametric}}
{%
	Parametric data are data that conform to a distribution, usually a normal distribution. Statistical tests that work with parametric data are generally much more powerful and predictive than tests that work with nonparametric data.
}

\longnewglossaryentry{parsimony}
	{name={parsimony}}
	{%
		A fundamental aspect of research that states if two or more competing explanations are considered then the simplest must be accepted. Thus, researcher would state that the pyramids were built by humans using known technology rather than aliens in spaceships.
	}

\longnewglossaryentry{population}
{name={population},
 see={sample}}
{%
	A set of similar items or events of interest to a researcher. For example, the set of small business owners in the United States would be a population.
}

\longnewglossaryentry{positivism}
{name={positivism}}
{%
	A philosophical system that posits that any justifiable assertion can be scientifically verified using statistics and logic. Thus, positivism rejects concepts like metaphysics and theism.
}

\longnewglossaryentry{positivist}
{name={positivist},
 see={positivism}}
{%
	A researcher who uses positivist techniques on research projects.
}

\longnewglossaryentry{postmodernism}
{name={post-modernism}}
{%
	A philosophical reaction to the assumptions and values of the ``modern'' period (roughly the 17th to 19th century). Post-modernists believe that rather than an objective reality independent of humans there is a subjective interpretation of reality so there is no such thing as a single ``Truth.''
}

\longnewglossaryentry{precision}
{name={precision}}
{%
	Research projects must precisely focus on one aspect of a problem or they will become so broad that their value will be diminished.
}

\longnewglossaryentry{pragmatism}
{name={pragmatism}}
{%
	An approach to research that values practical application over theory-building. 
}

\longnewglossaryentry{predictivevalidity}
{name={predictive validity},
	see={concurrentvalidity}}
{%
	The degree to which a measure predicts an outcome. For example, does increasing beer sales (a measure) predict increasing potato chip sales?
}

\longnewglossaryentry{probabilitysampling}
{name={probability sampling},
	see={nonprobibilitysampling}}
{%
	A type of sampling that involves a random selection from a population. It is called probability sampling since every member of the population has a probability to be selected. This is frequently called ``random sampling'' since members of the population are chosen at random. 
}

\longnewglossaryentry{proposition}
{name={proposition}}
{%
	A statement that expresses a judgment or opinion.
}

\longnewglossaryentry{qualitativedata}
{name={qualitative data},
	see={quantitativedata}}
{%
	Qualitative data approximates or describes attributes that cannot be directly measured, like employee morale, customer relationships, and management effectiveness. Typically, qualitative data attempt to answer questions like ``why'' and ``how come.'' 
}

\longnewglossaryentry{qualitativeresearch}
{name={qualitative research},
	see={quantitativeresearch}}
{%
	Qualitative research typically intends to explore observed phenomena with a goal of developing hypotheses and dive deep into a problem. Qualitative data collection involves semi-structured activities like focus groups and ethnographies.
}

\longnewglossaryentry{quantitativedata}
{name={quantitative data},
	see={qualitativedata}}
{%
	Quantitative data are numeric measurements of attributes, like the number of employees, the median value of housing, and total revenue. Quantitative data are gathered and analyzed using statistical methods.
}

\longnewglossaryentry{quantitativeresearch}
{name={quantitative research},
	see={qualitativeresearch}}
{%
	Quantitative research typically uses numerical data and statistical analysis to find patterns and generalize results to a large population. Quantitative data collection involves structured activities like surveys, interviews, and systematic observations.
}

\longnewglossaryentry{questionnaire}
{name={questionnaire}}
{%
	A type of survey research tool comprised of a written set of questions. Questionnaires are typically self-administered, that is, they are sent to respondents and completed without assistance.
}

\longnewglossaryentry{radicalhumanism}
{name={radical humanism}}
{%
	Humanism is a philosophical and ethical stance that emphasizes the value of human beings. It prefers critical thinking and evidence over dogma and superstition. Radical humanists believe that the world is constantly changing, in sometimes radical ways, with few predicable patterns. Research often involves subjectively interpreting evidence like interviews and focus groups.
}

\longnewglossaryentry{radicalstructure}
{name={radical structure}}
{%
	A structuralist believes that the world can be studied objectively and understood mathematically and scientifically without regard to subjective interpretation. Radical structuralists believe that the world is constantly changing, in sometimes radical ways, with few predictable patterns. Research often involves objectively interpreting evidence like direct measurements of populations.
}

\longnewglossaryentry{ratiodata}
{name={ratio data},
	see={quantitativedata}}
{%
	Ratio data are a type of quantitative data that are measured along a scale where each point is an equal distance from the next and there is a zero point. An example of ratio data is people's heights, which is measured along a uniform scale, \eg inches or centimeters. Because there is a true zero point, it is possible to determine that one person is twice as tall as another.
}

\longnewglossaryentry{realism}
{name={realism}}
{%
	A philosophical position that the world exists apart from human interpretation and understanding. A realist believes that research must be objective and not dependent upon the interpretation of the researcher.
}

\newacronym{rd}{RD}{Regression-Discontinuity}

\longnewglossaryentry{reliability}
	{name={reliability},
	 see={validity}}
	{%
		A descriptor for the consistency of a concept's measure. It is desirable to achieve the same, or nearly same, values for each sampling. For example, if the mean age of the people in one sample is $ 30 $ while in another is $ 50 $ then this would indicate a problem with reliability of the data.
	}

\longnewglossaryentry{replicability}
{name={replicability}}
{%
	A research project must be able to be replicated by other researchers or at other times in order to be considered sound.
}

\longnewglossaryentry{sample}
{name={sample},
	see={population}}
{%
	A subset of a population from which data are drawn in order to make inferences about the entire population.
}

\longnewglossaryentry{sampleframe}
{name={sampling frame}}
{%
	A subset of a sample that is accessible to the researcher. As an example, if the sample is high school students then the sampling frame could be the students in a specific high school or city. 
}

\longnewglossaryentry{semanticdiffscale}
{name={semantic differential}}
{%
	The Semantic Differential scale is used to determine attitudes or opinions using a sliding scale of values between two opposite pairs of adjectives. For example, respondents can be asked to choose some value between ``1-Dislike'' and ``5-Like'' for a certain snack sample in a store.
}

\newacronym{ses}{SES}{Socio-Economic Status}

\longnewglossaryentry{skew}
{name={skew}}
{%
	Skew is asymmetry in a distribution, so a graph appears distorted. A positive skew creates a longer tail on the right side of the graph.
}

\longnewglossaryentry{statisticalvalidity}
{name={statistical conclusion validity},
	see={validity}}
{%
	The degree to which the conclusions found in a research project are correct. Studies with high statistical conclusion validity minimize the two types of statistical errors: Type I (finding a correlation when there is none) and Type II (failing to find a correlation when one exists). 
}

\longnewglossaryentry{survey}
{name={survey}}
{%
	A research method involving the use of standardized questionnaires or interviews to collect data about people and their preferences, thoughts, and behaviors in a systematic manner. 
}

\longnewglossaryentry{ttest}
{name={t-test},
	see={anova}}
{%
	A test used to analyze the difference in two groups of samples that are normally distributed.
}

\longnewglossaryentry{theory}
{name={theory},
	plural={theories},
	see={paradigm}}
{%
	A system of ideas that is intended to explain phenomena. Theories that are accepted by scientists have been repeatedly tested and can be used to make accurate predictions. Unlike common usage, a scientific theory is a tested, falsifiable explanation for phenomena.
}

\longnewglossaryentry{translationalvalidity}
{name={translational validity},
	see={validity}}
{%
	The degree to which a construct has been measured by a research project. Translational validity is divided into two types: face and content.
}

\longnewglossaryentry{univariate}
{name={univariate},
	see={bivariate}}
{%
	A type of analysis involving a single variable. Univariate analysis findings include the central measure, standard deviation, and frequency distributions. Graphic tools include box plots for continuous data and bar plots for discrete data.
}

\longnewglossaryentry{validity}
{name={validity},
 see={reliability}}
{%
	A descriptor of whether a research project is measuring the variable under question. For example, if a project hypothesis is that older men tend to tip more than younger men then the validity of the study would be in question if the researcher only sampled men under the age of $ 40 $.
}

\longnewglossaryentry{variable}
{name={variable}}
{%
	In scientific research, a variable is a measurable representation of an abstract construct. For example, \textit{Intelligence Quotient} (IQ) is a construct that cannot be directly measured, but variables like verbal and mathematical acuity can be measured and are assumed to be a proxy for IQ.
}

\newacronym{yds}{YDS}{Youth Development Study}
