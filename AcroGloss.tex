% This file contains all of the acronymns and glossary entries for the book.

% Acronyms
% The ``code'' is what I use in the document to link to a definition here
% The ``name'' is what the user sees in the document
% The ``definition'' is what the user sees in the glossary
%\newacronym{code}{name}{Definition}

%\newacronym{nasa}{NASA}{National Aeronatics and Space Administration}

% Glossary
% The ``code'' is what I use in the document to link to a definition here
% The ``name'' is the phrase listed in the glossary
% The ``description'' defines the phrase
% Can also include a ``text'' entry that is how the phrase will appear in the text
% Ex: name{Latin Alphabet} uses caps in the Glossary
%     text{Latin alphabet} uses LC in the text
% Can also include a ``plural'' entry to spell non-standard plural words
% Ex: name{bravo} 
%     plural{bravi}

%\newglossaryentry{code}{name={Phrase in Glossary},
%		description={This describes the phrase.}}
% ``see'' creates a cross-link in the glossary, usually not needed
% Note that a glossary entry will not be printed if it is not in the
% main text somewhere.

%\newglossaryentry{myphone}{
%	name={George's Phone},
%	description={(520) 335-2958},
%	see={myaddr}
%}

% For long entries, the percent sign after the opening brace
% removes the CR/LF so you don't get an extra vertical space

%%%%%%%%%%%%%%%%%%%%%%%%%%%%%%%%%%%%%%%%%
% Chapter 01
%%%%%%%%%%%%%%%%%%%%%%%%%%%%%%%%%%%%%%%%%
\longnewglossaryentry{constructivism}
{name={constructivism}}
{%
	A philosophical stance that reality is a construct of the human mind and is, therefore, subjective. Normally, qualitative research methods are used by researchers who are constructivists.
}

\longnewglossaryentry{deductiveresearch}
{name={deductive research}}
{%
	A research methodology that works from a general theory to specific observations. This is sometimes called the ``theory-testing'' form of research.
}

\longnewglossaryentry{descriptiveresearch}
{name={descriptive research},
 see={exploratoryresearch}}
{%
	Research that is designed to describe observed phenomena. The goal is to improve
	understanding rather than explore new ideas.
}

\newglossaryentry{epistomology}{
	name={epistomology}, 
	description={A branch of philosophy that is concerned with the sources of knowledge.}
}

\longnewglossaryentry{explanatoryresearch}
{name={explanatory research},
	see={exploratoryresearch}}
{%
	Research that is designed to explain an observed phenomena or process. 
}

\newglossaryentry{exploratoryresearch}{
	name={exploratory research}, 
	description={Research that explores data in an effort to find new ideas.}
}

\longnewglossaryentry{falsifiability}
{name={falsifiability}}
{%
	To have credence, a hypothesis must be disprovable; that is, there must be a way to prove it wrong.
}

\longnewglossaryentry{idiographic}
{name={idiographic}}
{%
	An explanation for an observed phenomenon that explains only a single case and is not applicable to a wider population.
}

\longnewglossaryentry{inductiveresearch}
{name={inductive research}}
{%
	A research methodology that works from specific observations to a general theory. This is sometimes called the ``theory-building'' form of research.
}

\longnewglossaryentry{nomothetic}
{name={nomothetic}}
{%
	An explanation for an observed phenomenon that is applicable across a wide population rather than a single example.
}

\longnewglossaryentry{objectivism}
{name={objectivism}}
{%
	A philosophical stance that there exists an objective reality can be studied and understood.
}

\newglossaryentry{ontology}{
	name={ontology},
	description={The branch of philosophy that is concerned with the nature of reality.}
}

\newglossaryentry{paradigm}{
	name={paradigm},
	description={A pattern or model of how things work in the world.}
}

\longnewglossaryentry{parsimony}
	{name={parsimony}}
	{%
		A fundamental aspect of research that states if two or more competing explanations are considered then the simplest must be accepted. Thus, researcher would state that the pyramids were built by humans using known technology rather than aliens in spaceships.
	}

\longnewglossaryentry{precision}
	{name={precision}}
	{%
		Research projects must precisely focus on one aspect of a problem or they will become so broad that their value will be diminished.
	}

\longnewglossaryentry{qualitativeresearch}
{name={qualitative research},
	see={quantitativeresearch}}
{%
	Qualitative research typically intends to explore observed phenomena with a 
	goal of developing hypotheses and dive deep into a problem. Qualitative 
	data collection involves semi-structured activities like focus groups
	and ethnographies.
}

\longnewglossaryentry{quantitativeresearch}
{name={quantitative research},
	see={qualitativeresearch}}
{%
	Quantitative research typically uses numerical data and statistical
	analysis to find patterns and generalize results to a large population.
	Quantitative data collection involves structured activities like surveys,
	interviews, and systematic observations.
}

\longnewglossaryentry{replicability}
	{name={replicability}}
	{%
		A research project must be able to be replicated by other researchers or at other times in order to be considered sound.
	}
