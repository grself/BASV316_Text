% This file contains all of the acronymns and glossary entries for the book.

% Acronyms
% The ``code'' is what I use in the document to link to a definition here
% The ``name'' is what the user sees in the document
% The ``definition'' is what the user sees in the glossary
%\newacronym{code}{name}{Definition}

%\newacronym{nasa}{NASA}{National Aeronatics and Space Administration}

% Glossary
% The ``code'' is what I use in the document to link to a definition here
% The ``name'' is the phrase listed in the glossary
% The ``description'' defines the phrase
% Can also include a ``text'' entry that is how the phrase will appear in the text
% Ex: name{Latin Alphabet} uses caps in the Glossary
%     text{Latin alphabet} uses LC in the text
% Can also include a ``plural'' entry to spell non-standard plural words
% Ex: name{bravo} 
%     plural{bravi}

%\newglossaryentry{code}{name={Phrase in Glossary},
%		description={This describes the phrase.}}
% ``see'' creates a cross-link in the glossary, usually not needed
% Note that a glossary entry will not be printed if it is not in the
% main text somewhere.

%\newglossaryentry{myphone}{
%	name={George's Phone},
%	description={(520) 335-2958},
%	see={myaddr}
%}

% For long entries, the percent sign after the opening brace
% removes the CR/LF so you don't get an extra vertical space

\longnewglossaryentry{appliedresearch}
{name={applied research},
 see={basic research}}
{%
	Research that is intended to be applied to a situation rather than further the knowledge of some topic. For example, if a researcher completes a project designed to increase the sales of bottled water in a small town it would be considered applied research.
}

\longnewglossaryentry{basicresearch}
{name={basic research},
	see={applied research}}
{%
	Research that is intended to be further the knowledge of some topic rather than be applied to a specific situation. For example, if a researcher completes a project designed to refine some aspect of the Law of Supply and Demand it would be considered basic research.  
}

\longnewglossaryentry{boundarycondition}
{name={boundary condition}}
{%
	The assumptions about the ``who, when, and where'' in a theory. Boundary conditions govern how a theory can be applied or not applied.
}

\longnewglossaryentry{constructivism}
{name={constructivism}}
{%
	A philosophical stance that reality is a construct of the human mind and is, therefore, subjective. Normally, qualitative research methods are used by researchers who are constructivists.
}

\longnewglossaryentry{construct}
{name={construct}}
{%
	The characteristic of a person or organization to be assessed. Normally, a construct is not directly measurable so various indirect indicators must be used. Examples of constructs include employee attitude, happiness, or self-esteem. Usually, researchers operationalize a concept by defining it in measurable terms, and that becomes a construct.
}

\longnewglossaryentry{contentvalidity}
{name={content validity},
	see={validity}}
{%
	A determination of whether a measure correctly assesses the construct's content. For example, if a research project is attempting to determine the drivers for total sales in a store but only measured the price of the merchandise being sold then ignoring factors like advertising, competition, and even the general economy of the region would call into question the content validity of the study.
}

\longnewglossaryentry{concurrentvalidity}
{name={concurrent validity},
	see={predictivevalidity}}
{%
	The degree that a measure relates to an outcome that is presumed to occur simultaneously. For example, the results of a new employee attitude test would be the same as an older test if those tests have high concurrent validity.
}

\longnewglossaryentry{convergentvalidity}
{name={convergent validity},
	see={discriminantvalidity}}
{%
	The closeness that two measures relate to, ``converge on,'' a single construct. For example, if a research project measures the amount of sales of carbonated drinks, fruit juices, and bottled water in a store it would be expected that those would converge on a construct of ``drink sales.''
}

\longnewglossaryentry{deductiveresearch}
{name={deductive research}}
{%
	A research methodology that works from a general theory to specific observations. This is sometimes called the ``theory-testing'' form of research.
}

\longnewglossaryentry{descriptiveresearch}
{name={descriptive research},
 see={exploratoryresearch}}
{%
	Research that is designed to describe observed phenomena. The goal is to improve 	understanding rather than explore new ideas.
}

\longnewglossaryentry{discriminantvalidity}
{name={discriminant validity},
	see={convergentvalidity}}
{%
	The degree that a measure does not measure, ``discriminates between,'' one of two competing constructs. For example, a measure of the sale of toiletries in a department store would not be related to the construct of ``drink sales.'' 
}

\newglossaryentry{epistemology}{
	name={epistomology}, 
	description={A branch of philosophy that is concerned with the sources of knowledge.}
}

\longnewglossaryentry{explanatoryresearch}
{name={explanatory research},
	see={exploratoryresearch}}
{%
	Research that is designed to explain an observed phenomena or process. 
}

\longnewglossaryentry{explanatorypower}
{name={explanatory power}}
{%
	A theory or hypothesis has explanatory power if it accurately predicts phenomena. This can be statistically measured in quantitative research projects by calculating variance in regression analysis.
}

\newglossaryentry{exploratoryresearch}{
	name={exploratory research}, 
	description={Research that explores data in an effort to find new ideas.}
}

\longnewglossaryentry{facevalidity}
{name={face validity},
	see={validity}}
{%
	A determination of whether an indicator is a reasonable measure of an underlying construct ``on its face.'' For example, is the amount of money spent on live theater tickets a measure social class?
}

\longnewglossaryentry{falsifiability}
{name={falsifiability}}
{%
	To have credence, a theory or hypothesis must be disprovable; that is, there must be a way to prove it wrong using evidence.
}

\longnewglossaryentry{functionalism}
{name={functionalism}}
{%
	A belief in the practical application of a theory. Functionalism is more concerned with how a theory can be used in the real world than conducting	research for the sake of increasing understanding. 
}

\longnewglossaryentry{groundedtheory}
{name={grounded theory},
	plural={grounded theories}}
{%
	A theory based on observation rather than experimentation. Thus, the strength of the theory is dependent on the skill of the researcher and may not be repeatable by a different researcher or at a different time.
}

\longnewglossaryentry{hypothesis}
{name={hypothesis},
	plural={hypotheses}}
{%
	A proposed explanation for an observed phenomenon. Often, a hypothesis that may be based on incomplete information is the starting point for further investigation. As an example, if a merchant notices that eye-level shelves tend to need restocking more frequently a hypothesis may be proposed that shoppers purchase goods from eye-level shelves first. 
}


\longnewglossaryentry{idiographic}
{name={idiographic}}
{%
	An explanation for an observed phenomenon that explains only a single case and is not applicable to a wider population.
}

\longnewglossaryentry{inductiveresearch}
{name={inductive research}}
{%
	A research methodology that works from specific observations to a general theory. This is sometimes called the ``theory-building'' form of research.
}

\longnewglossaryentry{interpretivism}
{name={interpretivism}}
{%
	A research method that relies on observation and techniques like interviews to understand phenomena. 
}

\newacronym{irb}{IRB}{Institutional Review Board}

\longnewglossaryentry{logic}
{name={logic}}
{%
	A systematic set of principles that can be used to help validate a theory. Logic is a mode of reasoning that can link observations to explanations.
}

\longnewglossaryentry{logicalconsistency}
{name={logical consistency}}
{%
	A theory or hypothesis is logically consistent when all of the constructs, propositions, boundary conditions, and assumptions are congruous.
}

\longnewglossaryentry{model}
{name={model}}
{%
	A model is a representation of all or part of a system that is constructed to study that system. For example, meteorologists often create elaborate models to predict 	the path of a hurricane. 
}

\longnewglossaryentry{nomothetic}
{name={nomothetic}}
{%
	An explanation for an observed phenomenon that is applicable across a wide population rather than a single example.
}

\longnewglossaryentry{objectivism}
{name={objectivism}}
{%
	A philosophical stance that there exists an objective reality can be studied and understood.
}

\newglossaryentry{ontology}{
	name={ontology},
	description={The branch of philosophy that is concerned with the nature of reality.}
}

\longnewglossaryentry{operationalization}
{name={Operationalization}}
{%
	 The process of designing precise measures for abstract theoretical constructs. 
}

\longnewglossaryentry{paradigm}
{name={paradigm},
	see={theory}}
{%
	A pattern or model of how things work in the world. 
}

\longnewglossaryentry{parsimony}
	{name={parsimony}}
	{%
		A fundamental aspect of research that states if two or more competing explanations are considered then the simplest must be accepted. Thus, researcher would state that the pyramids were built by humans using known technology rather than aliens in spaceships.
	}

\longnewglossaryentry{precision}
	{name={precision}}
	{%
		Research projects must precisely focus on one aspect of a problem or they will become so broad that their value will be diminished.
	}

\longnewglossaryentry{pragmatism}
{name={pragmatism}}
{%
	An approach to research that values practical application over theory-building. 
}

\longnewglossaryentry{predictivevalidity}
{name={predictive validity},
	see={concurrentvalidity}}
{%
	The degree to which a measure predicts an outcome. For example, does increasing beer sales (a measure) predict increasing potato chip sales?
}

\longnewglossaryentry{positivism}
{name={positivism}}
{%
	A philosophical system that posits that any justifiable assertion can be scientifically verified using statistics and logic. Thus, positivism rejects concepts like metaphysics and theism.
}

\longnewglossaryentry{positivist}
{name={positivist},
 see={positivism}}
{%
	A researcher who uses positivist techniques on research projects.
}

\longnewglossaryentry{postmodernism}
{name={post-modernism}}
{%
	A philosophical reaction to the assumptions and values of the ``modern'' period (roughly the 17th to 19th century). Post-modernists believe that rather than an objective reality independent of humans there is a subjective interpretation of reality so there is no such thing as a single ``Truth.''
}

\longnewglossaryentry{proposition}
{name={proposition}}
{%
	A statement that expresses a judgment or opinion.
}

\longnewglossaryentry{qualitativeresearch}
{name={qualitative research},
	see={quantitativeresearch}}
{%
	Qualitative research typically intends to explore observed phenomena with a goal of developing hypotheses and dive deep into a problem. Qualitative data collection involves semi-structured activities like focus groups and ethnographies.
}

\longnewglossaryentry{quantitativeresearch}
{name={quantitative research},
	see={qualitativeresearch}}
{%
	Quantitative research typically uses numerical data and statistical analysis to find patterns and generalize results to a large population. Quantitative data collection involves structured activities like surveys, interviews, and systematic observations.
}

\longnewglossaryentry{radicalhumanism}
{name={radical humanism}}
{%
	Humanism is a philosophical and ethical stance that emphasizes the value of human beings. It prefers critical thinking and evidence over dogma and superstition. Radical humanists believe that the world is constantly changing, in sometimes radical ways, with few predicable patterns. Research often involves subjectively interpreting evidence like interviews and focus groups.
}

\longnewglossaryentry{radicalstructure}
{name={radical structure}}
{%
	A structuralist believes that the world can be studied objectively and understood mathematically and scientifically without regard to subjective interpretation. Radical structuralists believe that the world is constantly changing, in sometimes radical ways, with few predictable patterns. Research often involves objectively interpreting evidence like direct measurements of populations.
}

\longnewglossaryentry{realism}
{name={realism}}
{%
	A philosophical position that the world exists apart from human interpretation and understanding. A realist believes that research must be objective and not dependent upon the interpretation of the researcher.
}

\longnewglossaryentry{reliability}
	{name={reliability},
	 see={validity}}
	{%
		A descriptor for the consistency of a concept's measure. It is desirable to achieve the same, or nearly same, values for each sampling. For example, if the mean age of the people in one sample is $ 30 $ while in another is $ 50 $ then this would indicate a problem with reliability of the data.
	}

\longnewglossaryentry{replicability}
{name={replicability}}
{%
	A research project must be able to be replicated by other researchers or at other times in order to be considered sound.
}

\longnewglossaryentry{theory}
{name={theory},
	plural={theories},
	see={paradigm}}
{%
	A system of ideas that is intended to explain phenomena. Theories that are accepted by scientists have been repeatedly tested and can be used to make accurate predictions. Unlike common usage, a scientific theory is a tested, falsifiable explanation for phenomena.
}

\longnewglossaryentry{translationalvalidity}
{name={translational validity},
	see={validity}}
{%
	The degree to which a construct has been measured by a research project. Translational validity is divided into two types: face and content.
}

\longnewglossaryentry{validity}
{name={validity},
 see={reliability}}
{%
	A descriptor of whether a research project is measuring the variable under question. For example, if a project hypothesis is that older men tend to tip more than younger men then the validity of the study would be in question if the researcher only sampled men under the age of $ 40 $.
}

\longnewglossaryentry{variable}
{name={variable}}
{%
	In scientific research, a variable is a measurable representation of an abstract construct. For example, \textit{Intelligence Quotient} (IQ) is a construct that cannot be directly measured, but variables like verbal and mathematical acuity can be measured and are assumed to be a proxy for IQ.
}
